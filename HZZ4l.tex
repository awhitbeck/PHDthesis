\def\cm{\hbox{$\;\hbox{\rm cm}$}}
\def\eV{\hbox{$\;\hbox{\rm eV}$}}
\def\GeV{\hbox{$\;\hbox{\rm GeV}$}}
\def\MeV{\hbox{$\;\hbox{\rm MeV}$}}
\def\TeV{\hbox{$\;\hbox{\rm TeV}$}}
\def\ifb{\hbox{$fb^{-1}\rm$}}

\def\POWHEG{{\it POWHEG}}
\def\JHUGen{{\it JHUGen}}
\def\MCFM{{\it MCFM}}
\def\PYTHIA{{\it PYTHIA}}
\def\MADGRAPH{{\it MADGRAPH}}
\def\GEANTfour{{\it GEANT4}}

\newcommand{\KD}{$\mathscr{D}^{\rm kin}_{\rm bkg}$}
\newcommand{\superKD}{$\mathscr{D}_{\rm bkg}$}
\newcommand{\spinKD}{$\mathscr{D}_{J^P}$}
\newcommand{\Djet}{$\mathscr{D}_{\rm jet}$}
\newcommand{\pt}{$p_{T}$}
\newcommand{\sip}{$SIP_{3D}$}
\newcommand{\isocomb}{$ISO_{comb}$}

%\newcommand{\Lumi2011}{5.1~\ifb}
%\newcommand{\Lumi2012}{19.6~\ifb}

\section{Golden Decay Channel}
\label{sec:HZZ4l}
The $ZZ\to 4\ell$ channel, often refered to as the golden decay 
channel, is one of the most promising channels for discovering a 
Higgs like resonance over of broad range of masses because of 
the high mass resolution and low SM background rates.
Using the tools developed in chapter~\ref{sec:HiggsPhen}, it will 
be shown that this channel is also very condusive for property
measurements of resonances.

\subsection{Datasets}
\label{sec:HZZ4ldatasets}

Events used are selected either via the
double electron, double muon, or triple electron triggers.  The
double electron and muon triggers require that the transverse
momentum, $p_T$, of the leading and sub-leading leptons be 
greater than 17 and 8~GeV, respectively; the triple electron 
triggers thresholds are 15, 8, and 5~GeV, respectively.  
The efficiencies for these triggers are found to be at least 98\% 
for a SM Higgs boson with $m_H>$120~GeV.  

Monte Carlo (MC) simulations have been used to develop, optimize, 
and validate analysis strategies.  Signal samples
are generated using either \verb+POWHEG+~\cite{Frixione:2007vw} at next-to-leading 
order (NLO) in $\alpha_s$ for SM Higgs couplings or with 
\verb+JHUGen+~\cite{Gao:2010qx,Bolognesi:2012mm} at leading order 
(LO) in $\alpha_s$ for either
SM Higgs couplings or non-SM scalar couplings and higher spin
models.  For simulation of Higgs bosons produced in association
with either weak vector bosons, VH, or $t\bar{t}$ pairs, ttH, 
the event generator \verb+PYTHIA+~\cite{Sjostrand:2006za} is used.  

Since the \verb+PYTHIA+ samples do not model the interference
of final leptons for the $4\mu$ and $4e$ channels.  These samples
are reweighted using the JHUGen matrix element calculation where
appropriate.  However, the branching fractions 
$\mathscr{B}(H\to 4\ell)$ are taken from \verb+PROPHECY4F+ which 
includes both interference effects and NLO QCD/EW corrections. 
The narrow-width approximation for the $m_{4\ell}$
line shape is employed at low mass resulting in a Breit-Wigner
distribution.  At larger masses were the Higgs width become large,
the $m_{4\ell}$ line shape is reweighted to match the complex-pole
scheme described in~\cite{Passarino:2010qk,Goria:2011wa,Kauer:2013cga}.  Effects from the intereference
bewteen signal and the continuum $gg\to ZZ$ production is also 
accounted for following the prescription of~\cite{Passarino:2012ri}.
Events are weighted to account for the total production cross
section of the Higgs boson according to 
references~\cite{Anastasiou:2008tj,deFlorian:2009hc,Baglio:2010ae,Dittmaier:2011ti,Djouadi:1991tka,Dawson:1990zj,Spira:1995rr,Harlander:2002wh,Anastasiou:2002yz,Ravindran:2003um,Catani:2003zt,Actis:2008ug} 
for gluon-gluon fusion process and according to 
references~\cite{Dittmaier:2011ti,Ciccolini:2007jr,Ciccolini:2007ec,Figy:2003nv,Arnold:2008rz,Bolzoni:2010xr} 
for VBF process.

The SM continuum production of ZZ events via $q\bar{q}$
annihilation is simulated at NLO using \verb+POWHEG+ while
other diboson processes were simulated with 
\verb+MADGRAPH+\cite{Alwall:2007st}.
The gluon-gluon
fusion production of continuum ZZ events is simulated using
\verb+GG2ZZ+~\cite{Binoth:2008pr}. Other di-boson background 
as well 
as Drell-Yan events  are simulated at LO using \verb+MADGRAPH+.  
Di-boson samples produced at leading order are rescaled to match
cross sections predicted by NLO calculations while Drell-Yan 
samples are rescaled to match cross sections predicted by NNLO
calculations.  Finally $t\bar{t}$ events are simulated at NLO
with \verb+POWHEG+.

All initial-state and final-state radiation is modelled using 
\verb+PYTHIA+. Parton density function are taken from 
\verb+CTEQ6L+~\cite{Lai:2010nw} (\verb+CT10+~\cite{Lai:2010vv}) 
for LO (NLO) 
generators.  Detector effects and event reconstruction is 
simulated using \verb+GEANT4+~\cite{Allison:2006ve}.  The number 
of reconstructed vertices per collision is reweighted to match 
the distribution seen in data.  Addition energy deposited into
calorimeter from pileup interactions and from the underlying
event is subtracted using the \verb+FASTJET+ algorithm~\cite{Cacciari:2007fd,Cacciari:2008gn,Cacciari:2011ma}. 

The generators and cross sections for each of these event types 
is shown in table~\ref{table:MCsamples}

\begin{table}
\begin{center}
\begin{tabular}{c|c|c}
\hline 
\hline
Sample Name & Generator & $\sigma$  \\ 
\hline
$pp\to H\to ZZ^{(*)}\to 4\ell$ & \verb+POWHEG+ & --- \\
$gg\to H\to ZZ^{(*)}\to 4\ell$ & \verb+JHUGen+ & --- \\
$X\to ZZ^{(*)}\to 4\ell$ & \verb+JHUGen+ & --- \\
Z+X & \verb+MADGRAPH+ & --- \\
$t\bar{t}$ & \verb+POWHEG+ & --- \\
WW\&ZW & \verb+MADGRAPH+ & --- \\
$q\bar{q}\to ZZ$ & \verb+POWHEG+ & --- \\
$gg\to ZZ$ & \verb+GG2ZZ+ & --- \\
\hline
\hline
\end{tabular}
\end{center}
\caption{List of MC samples used for the $ZZ^{(*)}\to 4\ell$ analysis.
along with the event generator used to simulate them and their
respective cross sections.  Note, the meaning of cross sections
for non-SM samples is highly model-dependent.  As a results, 
cross sections are ommitted for these samples.}
\label{table:HZZ4lMCsamples}
\end{table}

\subsection{Event Selection and Categorization}
\label{sec:HZZ4lselection}

Selections are applied to increase the purity of signal events 
as much as possible.  Selections based isolation and identification
requirements are used to reduce background in which the physical
process does not produce four leptons, e.g. $Z+jet$ events, 
generally referred to as reducible backgrounds.  All 
reconstructed leptons are also required to have in impact
parameter which is sufficiently compatible with the primary 
vertex, which is quantified by the significance of the impact 
parameter, \sip. 

Events are then classified into a number of categories as 
outlined in table~\ref{table:HZZ4lCategories}.  Categories 
which make up the 
signal region always consist of events with two oppositely 
charged lepton pairs. The signal regions are then further 
subdivided into categories based on the number of jets 
which provides sensitivity to various
production mechanisms, especially VBF where at least two 
additional jets are always produced.  Events are either 
in the dijet tag category if there are at least two jets
or in the non-dijet category if there are less than two jets.  
Events are also classified according to final state lepton
flavors. Since each flavor will have a different $m_{4\ell}$
resolution, this categorization increases the overall
sensitivity to signal events. 
Control regions in which either looser ID requirements or 
same-sign leptons pairs are used to estimate the amount of 
instrumental background from data. 

%\begin{table}
%\begin{center}
%\begin{tabular}{c|c|c|c}
%\hline 
%\hline
%category & notes & resolution & $N_{exp}$ per \ifb \\
%\hline 
%\hline
%$H\to 4\mu$ & -- && \\ \hline
%$H\to 2e2\mu$& -- && \\ \hline
%$H\to 4e$   & -- &&\\ \hline \hline
%$Hjj\to 4\mu$& -- && \\ \hline 
%$Hjj\to 2e2\mu$& -- && \\ \hline 
%$Hjj\to 4e$& -- && \\ \hline
%\hline
%\hline
%\end{tabular}
%\caption{...}
%\label{table:HZZ4lCategories}
%\end{center}
%\end{table}

A minimal amount of the kinematic selections are applied to 
further reduce the continuum ZZ backgrounds.  In order to 
reduce the contamination of low-mass
resonances, such as $J/\psi$'s, all dilepton pairings are 
required to have a 
minimum invariant mass, $m_{\ell\ell}>4$\GeV.  Dilepton 
pairings whose invariant mass is closest to the Z pole-mass
is referred to as $Z_{1}$, while the other pairing is referred
to as $Z_{2}$.  The invariant mass of these dilepton pairs is 
denoted by $m_1$ and $m_2$, respectively, and are required
to satisfy $12<m_2<120$~\GeV~and $40<m_1<120$~\GeV.  The 
leading and subleading leptons are required to have
\pt$>20$ and \pt$>10$~\GeV, respectively.

\subsection{Yields and Kinematics Distributions}
\label{sec:HZZ4lyields}

The expected shape and event yields for continuum ZZ backgrounds
are taken from MC simulation.  Cross sections
for $q\bar{q}$ annihilation and gg initiated events are 
calculated at NLO using \verb+MCFM+.  Systematic variations due to 
QCD renormalization scale, factorization scale, and parton 
distribution functions are calculated as a function of $m_{4\ell}$
following the \verb+PDF4LHC+ prescription~\cite{Botje:2011sn,Alekhin:2011sk}. 
The total uncertainties from QCD and PDFs are typically 8\%.  

{\it NEED TO REWRITE}

To estimate the reducible ($Zb\bar{b}$, $t\bar{t}$) and 
instrumental ($Z+light jets$, $WZ+jets$) backgrounds, a 
$Z_1+X$ background control region, well separated from the 
signal region, is defineed.  In addition, a sample $Z_1+\ell_{reco}$,
with at least one reconstructed lepton object, is defined for the
measurement of the lepton misidentification probability - the
probability for a reconstructed objet to pass the isolation and 
identification requirements.  The contamination from WZ in these
events is suppressed by requiring the imbalance of the measured
energy deposition in the transverse plane to be below 25~\GeV. 
Then lepton misidentification probability is compared, and found
compatible, with the one derived from MC simulation.

The event rates measured in the background control region are
extrapolated to the signal region.  Two different approaches are
used.  They differ in the way the contribution from electrons
coming from photon conversions is handled.  Both start by 
relaxing the isolation and identification criteria for two 
additional reconstructed lepton objects.  The first approach 
follows from the previous CMS analysis [14].  It aims at
estimating all contributions of reducible background in one 
single step.  The additional pair of leptons is required to have
the same charge (to avoid signal contamination) and same flavour,
a reconstructed invariant mass .. 

{\it END OF REWRITE}

Systematic uncertainties are evaluated from data for trigger 
and combined lepton reconstruction, identification, and
isolation efficiencies.  Samples of $Z\to\ell\ell$, 
$\Upsilon\to\ell\ell$, and $J/\psi\to\ell\ell$ events are used
to set and validate the abolute momentum scale and momentum
resolution.  
%More details on the measurement and validation of momentum 
%scale and resolution can be found in Appendix A of~\cite{}.
Additional systematics arrise from limited statistics in 
background control regions as well as systematic differences 
between the control regions.  
%All systematics for background
%are tabulated in table~\ref{HZZ4lSystematics}.

%\begin{table}
%\begin{center}
%\begin{tabular}{l|c}
%\hline 
%\hline
%name & size \\
%\hline 
%\hline
%& --  \\ \hline
%& -- \\ \hline
%& -- \\ \hline \hline
%& -- \\ \hline 
%& -- \\ \hline 
%& --  \\ \hline
%\hline
%\hline
%\end{tabular}
%\end{center}
%\caption{Table listing all styematics for ...}
%\label{table:HZZ4lSystematics}
%\end{table}

Starting from Higgs boson production cross sections decsribed in 
section~\ref{sec:HZZ4ldatasets}, signal event yields are calculated
using MC simulations to calculate efficiencies.   Shapes of 
signal distributions are also taken from MC simulations.  
%All systematic 
%uncertainties for signal events are tabulated in 
%table~\ref{HZZ4lSystematics}.

There are a number of different measurables with which event 
likelihoods will be evaluated.  For cross section measurements,
$m_{4\ell}$, \KD, and either $\mathscr{D}_jet$ or $p_{T,4\ell}$ 
are used.  The first two
variables provide discrimination between signal and background,
while the latter two distinguish different production modes.
\KD~is a discriment built within the MELA framework presented 
in chapter~\ref{sec:HiggsPhen} and is described in 
equation~\ref{eq:Dbkg}.
\Djet~is used for events in the dijet category and is a linear 
combination of the difference in pseudorapidity,
$\Delta\eta$, and the invariant mass of the event's two leading 
jets, $m_{jj}$.  The coefficients which are used in \Djet~were 
optimized 
for maximal separation between VBF events and gluon-gluon fusion
events.  For events in the non-dijet category, $p_{T,4\ell}$ is used
to distinguish different production mechanisms.  

The signal and background 
$m_{4\ell}$ distributions are described using emperical functions, 
$\mathscr{P}_{bkg}(m_{4\ell})$ and $\mathscr{P}_{sig}(m_{4\ell};m_{H})$.  
The signal modeling is derived 
through a similar manner use in the semi-leptonic analysis and
is ultimately parameterized as a function of the Higgs mass 
hypothesis, $m_H$.
To account for the correlation between $m_{4\ell}$ and other
variables, conditional probability distributions are built, 
$\mathscr{P}($\KD$|m_{4\ell})$, $\mathscr{P}($\Djet$|m_{4\ell})$,
and $\mathscr{P}(p_{T,4\ell}|m_{4\ell})$.  
In this way,
three separate likelihoods can be constructed to describe 
each event class: using a single measureable, $m_{4\ell}$;
a 2D likelihood described by
\begin{equation}
\mathscr{L}_{2D}\sim\mathscr{P}_{bkg}(m_{4\ell})\mathscr{P}_{bkg}(\mathscr{D}_{bkg}^{kin}|m_{4\ell})+\mu\times\mathscr{P}_{sig}(m_{4\ell};m_H)\mathscr{P}_{sig}(\mathscr{D}_{bkg}^{kin}|m_{4\ell});
\end{equation}
or using all three measureables according to
\begin{equation}
\begin{split}
\mathscr{L}_{3D}(m_{4\ell},\mathscr{D}^{kin}_{bkg},\mathscr{D}_{jet})\sim\mathscr{P}_{bkg}(m_{4\ell})\mathscr{P}_{bkg}(\mathscr{D}^{kin}_{bkg}|m_{4\ell})\mathscr{P}_{bkg}(\mathscr{D}_{VBF}|m_{4\ell})+ \\
\mu\times\mathscr{P}_{sig}(m_{4\ell};m_H)\mathscr{P}_{sig}(\mathscr{D}^{kin}_{bkg}|m_{4\ell})\mathscr{P}_{sig}(\mathscr{D}_{VBF}|m_{4\ell}),
\end{split}
\end{equation}
where $\mathscr{D}_{VBF}$ is used as short hand for either $p_{T,4\ell}$ or \Djet, 
depending on which category the event belongs to. 

For certain property measurements, distributions of \spinKD~
and \superKD are used.  The \superKD~variable is an extension
of \KD~which also includes $m_{4\ell}$ information for optimal 
separation of signal and background,
\begin{equation}
\mathscr{D}_{bkg}=\left(1+\frac{\mathscr{P}^{kin}_{bkg}(m_{Z_1},m_{Z_2},\vec{\Omega}|m_{4\ell})\times\mathscr{P}^{mass}_{bkg}(m_{4\ell})}{\mathscr{P}^{kin}_{0^+}(m_{Z_1},m_{Z_2},\vec{\Omega}|m_{4\ell})\times\mathscr{P}^{mass}_{sig}(m_{4\ell}}\right)^{-1}.
\end{equation}

Although spin-0 models are
inherently production independent, spin-1 and spin-2 models
can have information of the production mechanism reflected in 
distributions of the production angles through spin correlations.
In order to be more model independent when testing alternative
signal models, discriminants can be designed such that production
angles are integrated out making the discriminant independent 
of the production mechanism.  A third set of variables which 
are production independent, $D_{bkg}^{dec}$ and $D_{J^P}^{dec}$,
will also be used to test spin-1, and spin-2 models.
In these cases, the likelihood used for spin-parity measurements 
is constructed 
from two observables, $\mathscr{L}$(\superKD,\spinKD), or 
their production independent forms. 

The input matrix element calculations used are the 
analytical descriptions discussed in section~\ref{sec:HiggsPhen}
and the \verb+JHUGen+ squared matrix element.  These calculations
were checked against each other and were found to perform the 
same in the $2e2\mu$ channel.  However, the since the analytical
calculation doesn't incorporate lepton interference in the 
$4e$ and $4\mu$ channels, the \verb+JHUGen+ calculation is used. 
This was found to provide at most a 10-15\% improvement over the 
analytical calculation. 

The expected distributions that are used to build likelihoods
are taken from MC simulation for both the signal and continuum
backgrounds.  The \KD~and \spinKD distributions for the reducible
background are found to be similar to those of the continuum ZZ
backgrounds.  Because of the lack of statistics in the control
regions, the continuum background distributions are used and then
corrected to match the average shape in the opposite sign control 
regions.  The difference between the control region shapes and the
continuum ZZ shapes are taken as a systematic uncertainty on the
reducible background.   

\subsection{Cross Section Measurement}
\label{sec:HZZ4lxsec}

The expected and observed event yields for the different event 
classes is shown in tables~\ref{table:HZZ4lYieldsLowMass},~\ref{table:HZZ4lYields}.  
The expected and observed $m_{4\ell}$ distribution is show in 
figure~\ref{fig:HZZ4lMassDistribution},
where the reducible and irreducible backrounds are shown in green
and blue respectively.  The expected shape of a 126~GeV SM Higgs
boson is shown in red.  The expected and observed distribution of 
events in the 
$m_{4\ell}-K_D$ plane are shown in figure~\ref{fig:HZZ4lMassKDdist}.
Finally, expected and observed distributions of events in the 
$m_{4\ell}-p_{T,4\ell}$ and $m_{4\ell}-$\Djet~plane are 
shown in figure~\ref{fig:HZZ4lVBFvars}. 

\begin{figure}
\begin{center}
\includegraphics[trim=50 0 20 00,clip,width=.57\linewidth]{HZZ4lPlots/ZZMass_7Plus8TeV_70-1000_3GeV.eps}
\includegraphics[trim=50 0 00 00,clip,width=.40\linewidth]{HZZ4lPlots/ZZMass_7Plus8TeV_70-180_3GeV.eps}
\caption{Invariant mass distribution of the $4\ell$ system for
events between $70<m_{4\ell}<1000$~GeV (left) and between
$100<m_{4\ell}<180$~GeV (right).  All final states have been included.  
Points with error bars represent a sum of the $\sqrt{s}=7~TeV$
and $\sqrt{s}=8~TeV$ datasets.  Solid histograms represent
background estimations.  The open red histogram represents
simulation of a SM Higgs, $m_H=126~GeV$.}
\label{fig:HZZ4lMassDistribution}
\end{center}
\end{figure}

\begin{figure}
\begin{center}
\includegraphics[trim=270 0 0 0,clip,width=.32\linewidth]{HZZ4lPlots/KD_vs_m4l_lowMass_Back.eps}
\includegraphics[trim=270 0 0 0,clip,width=.32\linewidth]{HZZ4lPlots/KD_vs_m4l_lowMass_Signal.eps}
\includegraphics[trim=270 0 0 0,clip,width=.32\linewidth]{HZZ4lPlots/KD_vs_m4l_highMass_Back.eps}
\caption{
Distribution of $m_{4\ell}$ and $K_D$ in various regions.  
Contours in the left and right plot represent the background 
expectation of continuum ZZ events.  Contours in the middle
plot represent signal plus background expectation, where signal
is a SM Higgs, $m_H=126~GeV$.  Points with error bars represent
the individual events observed in the four different final states.
Horizontal error bars represent the reconstructed mass
uncertainties.
}
\label{fig:HZZ4lMassKDdist}
\end{center}
\end{figure}


\begin{figure}
\begin{center}
\includegraphics[trim=250 30 0 0,clip,width=.49\linewidth]{HZZ4lPlots/M4l_vs_pT_VBF.eps}
\includegraphics[trim=250 30 0 0,clip,width=.49\linewidth]{HZZ4lPlots/M4l_vs_pT_ggH.eps}\\
\includegraphics[trim=250 30 0 0,clip,width=.49\linewidth]{HZZ4lPlots/M4l_vs_Fisher_VBF.eps}
\includegraphics[trim=250 30 0 0,clip,width=.49\linewidth]{HZZ4lPlots/M4l_vs_Fisher_ggH.eps}
\caption{Distribution of $p_{T,4\ell}$ in the non-dijet category (top row) and 
\Djet~in the dijet category (bottom row) for expectation of a VBF produced (left column) 
or a gluon-gluon fusion produced Higgs boson with $m_H=126$~GeV.
Points with error bar show the distribution of observed $4\mu$ (circles), $4e$ (triangles), 
and $2e2\mu$ (squares) events.}
\label{fig:HZZ4lVBFvars}
\end{center}
\end{figure}

The data show a clear excess of events around $m_{4\ell}=125$~GeV. 
Elsewhere, there are no significant deviations from the 
background only expectation found.  
Events near the signal peak also tend to be distributed closer to 
\KD=1, consistent with that of a Higgs-like signal, as demonstrated
in figure~\ref{fig:HZZ4lMassKDdist}. 
Since \KD~efficiently separates signal events from background,
selecting events at higher values of \KD~provides a subsample
of data which has a higher expected signal purity.

%Thus, 
%a convenient visualization is provided by plots of the 
%$m_{4\ell}$ distribution after cutting on \KD.  
%Three examples are shown in figure~\ref{fig:HZZ4lMassKDdistWithCut}.
%The left plot is with no cut, the middle with a cut of \KD$>0.3$, 
%and the right with a cut of \KD$>0.5$.  The observed distributions 
%of $p_{T,4\ell}$ and \Djet~are also consistent with expectation from
%a Higgs boson.  

%\begin{figure}
%\begin{center}
%\includegraphics[width=.32\linewidth]{HZZ4lPlots/ZZMass_7Plus8TeV_100-180_3GeV.eps}
%\includegraphics[width=.32\linewidth]{HZZ4lPlots/ZZMass_7Plus8TeV_100-180_3GeV_KD03.eps}
%\includegraphics[width=.32\linewidth]{HZZ4lPlots/ZZMass_7Plus8TeV_100-180_3GeV_KD05.eps}
%\caption{Distribution of $m_{4\ell}$ applying different cuts on
%$K_D$, $K_D>$0.0 (left), 0.3 (middle), 0.5 (right). All final 
%states have been included.  
%Points with error bars represent a sum of the $\sqrt{s}=7~TeV$
%and $\sqrt{s}=8~TeV$ datasets.  Solid histograms represent
%background estimations.  The open red histogram represents
%simulation of a SM Higgs, $m_H=126~GeV$.}
%\label{fig:HZZ4lMassKDdistWithCuts}
%\end{center}
%\end{figure}

To quantify the statistical significance of the observed data
with respect to either background or signal and background, 
fit are done using either the 1D, 2D, or 3D likelihood as 
described in the previous section.  

Compatibility of data with respect to the background only
hyothesis can also be quantified in terms of 95\% CL upper
limits of $\mu$.  
Figure~\ref{fig:HZZ4lUpperLimits} shows these upper limits as
a function of $m_H$.  The median expected limits are represented
by the dashed blue lines, while the 68\% and 95\% countours are 
represented by green and yellow bands, respectively.  The 
observed upper limits are represented by the solid black line. 
Aside from the significant deviation
from expectation near 125~\GeV, the observed upper limits are 
always consistent with expectation to within $2\sigma$.  The
observed upper limit imply that the current data is sufficient
to rule out SM Higgs mass hypotheses bewteen $130<m_{H}<827$~\GeV.
The large deviation from the expected limit around 126~GeV is 
a reflection of the excess of events in this region.  

\begin{figure}
\begin{center}
\includegraphics[width=.49\linewidth]{HZZ4lPlots/UpperLimit_ASCLS_7p8TeV_wholeMass_Final2_2l2tau.eps}
\caption{Expected and Observed 95\% CL upper limit on
$\sigma/\sigma_{SM}$ as a function of the hypothetical 
Higgs mass, $m_H$, in the range [110-1000].
The green and yellow bands represent
the one and two sigma bands of the expected distribution, 
respectively.}
\label{fig:HZZ4lUpperLimits}
\end{center}
\end{figure}

The excess around 126 GeV can be interpretted in terms of the 
p-value with respected to the background only hypothesis.  
The p-value scan as a function of the hypothetical Higgs mass
is shown in figure~\ref{fig:HZZ4lPvalues}.  The maximum local 
p-value 
occurs at 125.7~\GeV~and has a value of $7.3\sigma$.  This 
significant deviation from the background-only hypothesis, has 
a cross section which is compatible 
with that expected from the SM Higgs.  The ratio of the best-fit
cross section with respected to the expected SM Higgs cross
section if found to be $\mu=\sigma_{obs}/\sigma_{SM}=0.93^{+0.30}_{-0.24}$.  Figure~\ref{fig:HZZ4lfittedMu} shows the best-fit value
in both the dijet ($\mu=1.46^{+0.89}_{-0.63}$)and untagged ($\mu=0.83^{+0.31}_{-0.25}$) categories, as well as the combined
(black line).  The width of the particle is found to be 
constistent with be much smaller than the experimental resolution
of $m_{4\ell}$, consistent with the expectation of a 126~\GeV~SM 
Higgs boson.

\begin{figure}
\begin{center}
\includegraphics[width=.49\linewidth]{HZZ4lPlots/Pvals_PLP_lowMass_1D2D3D_Final3_no2l2tau_7p8sep.eps}
\includegraphics[width=.49\linewidth]{HZZ4lPlots/Pvals_PLP_wholeMass_Final3_2l2tau_7p8sep.eps}
\caption{Expected and Observed p-value with respect to the
background only hypothesis as a function of the hypothetical 
Higgs mass, $m_H$, in the range [110-180] (left) and 
[110-1000] (right).  Solid lines show the observed p-values
while dashed lines show the expected p-values, assuming a 
SM Higgs.  Green lines show p-values obtained using only
the information about $m_{4\ell}$ distributions. Red
lines show p-values obtained using $m_{4\ell}$ vs $K_D$
distributions. }
\label{fig:HZZ4lPvalues}
\end{center}
\end{figure}

\begin{figure}
\begin{center}
\includegraphics[trim=270 0 0 0,clip,width=.49\linewidth]{HZZ4lPlots/mu_bestfit_bycategory.eps}
\caption{Best fit signal strength modifier, $\mu$, is both the 
dijet and untagged categories as well the combination of all 
channels (black line).  Red bar represent the 68\% confidence 
intervals for each of the indivual measurements.  The 
green band represents the 68\% confidence interval for the
combined measurement.}
\label{fig:HZZ4lfittedMu}
\end{center}
\end{figure}



%============
\begin{table}[!hb]
\begin{center}
\begin{tabular}{lcccc}
\hline \textbf{Channel} & $4e$ & $4mu$ & $2e2\mu$ & $4\ell$ \\ 
\hline
$ZZ$ background &  1.1  $\pm$  0.1  &  2.5  $\pm$  0.2  &  3.2  $\pm$  0.2 & 6.8  $\pm$ 0.3  \\
$Z$ + X background &  0.8  $\pm$  0.2  &  0.4  $\pm$  0.2  &  1.3  $\pm$  0.3 &  2.5 $\pm$ 0.4 \\
\hline
All backgrounds            &  1.9  $\pm$  0.2   & 2.9  $\pm$ 0.2    &  4.6  $\pm$ 0.4  & 9.4  $\pm$ 0.5\\
\hline
$m_H$ =  125 \GeV &  3.0  $\pm$  0.4  &  6.4  $\pm$  0.7  &  7.7  $\pm$  0.9  & 17.1  $\pm$ 1.2 \\
$m_H$ =  126 \GeV &  3.4  $\pm$  0.5  &  7.1  $\pm$  0.8  &  8.8  $\pm$  1.0  & 19.3  $\pm$ 1.4 \\
\hline
Observed  & 4 & 8 & 13 & 25 \\ 
\hline
\end{tabular}
\caption{Expected and observed yields in the mass range $121.5<m_{4\ell}<130.5$ for different event classes.}
\label{table:HZZ4lYieldsLowMass}
\end{center}
\end{table}
%============

%============
\begin{table}[!hb]
\begin{center}

\begin{tabular}{lcccc}
\hline \textbf{Channel} & $4e$ & $4\mu$ & $2e2\mu$ & $4\ell$  \\ 
\hline
$ZZ$ background &  77.4  $\pm$ 10.7  &  119.8  $\pm$  15.6  &  191.2  $\pm$  24.6  & 388.4 $\pm$ 31.0\\
$Z$ + X  background & 7.4    $\pm$ 1.5    &  3.6      $\pm$ 1.5     &   11.5    $\pm$ 2.9     & 22.5  $\pm$ 3.6  \\
\hline
All backgrounds              & 84.8 $\pm$ 10.8   &  123.5   $\pm$ 15.6  &   202.7  $\pm$ 24.8   & 411.0 $\pm$ 31.2 \\
\hline
$m_H$ =  125 \GeV &  3.6  $\pm$  0.5  &  6.9  $\pm$  0.8  &  8.9  $\pm$  1.0  & 19.4 $\pm$ 1.4  \\
$m_H$ =  126 \GeV &  4.0  $\pm$  0.6  &  7.5  $\pm$  0.9  &  9.8  $\pm$  1.1  & 21.3 $\pm$ 1.5 \\
$m_H$ =  500 \GeV &  5.2  $\pm$  0.6  &  7.0  $\pm$  0.8  &  12.2  $\pm$  1.4 & 24.3 $\pm$ 1.7  \\
$m_H$ =  800 \GeV &  0.7  $\pm$  0.1  &  0.9  $\pm$  0.1  &  1.6  $\pm$  0.2   & 3.1  $\pm$ 0.2 \\
\hline
Observed  & 89 & 134 & 247 & 470\\ 
\hline
\end{tabular}
\caption{Expected and observed yields in the mass range $100<m_{4\ell}<1000$ for difference class of events.}
\label{table:HZZ4lYields}
\end{center}
\end{table}
%============

\subsection{Spin and Parity Measurements}
\label{sec:HZZ4lspinParity}

Assuming two basic conservation laws, electric charge and angular
momentum, one can infer that the excess of events presented 
above corresponds to a 
new chargless, bosonic resonance.  However, little else can be
concluded from the above data alone.  Furthermore, there are a
number of resonances which could, in principle, mimic this 
signal.  
Understanding whether or not this new boson is the SM Higgs, 
one of several Higgses, or even something more exotic, like 
a graviton, is one of the most promising route to searching
for physics beyond the SM.  As demonstrated in 
chapter~\ref{sec:HiggsPhen}, the MELA techniques can be employed
to perform property measurements and infer more information 
about the observed resonance. 

Hypothesis testing can be used to evaluate the compatibility 
of data with respect to either the null hypothesis, the SM
background plus a SM Higgs boson, or some alternative signal
hypothesis.  The list of alternative signal hypotheses include:
$J^P=0^-$, $0^+_h$, $q\bar{q}\to1^-$, $q\bar{q}\to1^+$, 
$gg\to2_m^+$, $gg\to2_h^+,$, $gg\to2_h-$, $q\bar{q}\to2_m^+$, 
and $gg\to2_b^+$ and are described in chapter~\ref{sec:HiggsPhen}.  
%The list of hypotheses that are tested, a breif description of 
%them, and the discriminants used for measurements is shown in 
%table~\ref{???}.  
In each case, a dedicated discriminant is built, 
\spinKD, and used to distinguish kinematics of a SM Higgs boson 
from the alternative signal hypothesis.  

%\begin{table}
%\begin{center}
%\begin{tabular}{|c|c|c|}
%\hline 
% $J^{P}$ & production & comment \\
%\hline 
%\hline
%$0^-$            & $gg\to X$ & pseudoscalar \\ \hline
%$0_h^+$          & $gg\to X$ & high dim operator \\ \hline
%$2_{m,gg}^+$      & $gg\to X$ & minimal couplings \\ \hline
%$2_{m,q\bar{q}}^+$ & $q\bar{q}\to X$ & minimal couplings \\ \hline
%$1_{-}$          & $q\bar{q}\to X$ & exotic vector \\ \hline
%$1_{+}$          & $q\bar{q}\to X$ & exotic axialvector \\ \hline
%$2_{h}^+$        & $gg\to X$ & high dim operator \\ \hline
%$2_{h}^+$        & $gg\to X$ & high dim operator \\ \hline
%$2_{b}^+$        & $gg\to X$ & tensor w/ couples to bulk SM fields \\ \hline
%\end{tabular}
%\caption{List of alternative signal models considered for spin and 
%parity measurements.}
%\label{table:HZZ4lAltSig}
%\end{center}
%\end{table}

The expected and observed $\mathscr{D}_{bkg}$ and 
$\mathscr{D}_{bkg}^{dec}$ distributions are shown in 
figure~\ref{fig:HZZ4lsuperMELA}.  
Although the $\mathscr{D}_{bkg}$ distributions of some 
alternative signals are more background 
like compared to the SM Higgs, these variations are typically 
small compared to the difference between each signal and 
background.  Thus, this variable serves as a sufficient, model 
independent way of isolating signal events.  

The distribution of each of the $D_{J^P}$ variables is shown in 
figure~\ref{fig:HZZ4ldjp} for events which satisfy
$D_{bkg}>0.5$.  Each plot shows that the SM 
Higgs tends
to be distributed more towards $D_{J^P}=1$ while the corresponding
alternative signal is distributed more towards $D_{J^P}=0$. 

\begin{figure}
\begin{center}
\includegraphics[trim=270 0 0 0,clip,width=.49\linewidth]{HZZ4lPlots/0minus_SuperKD.eps}
\includegraphics[trim=270 0 0 0,clip,width=.49\linewidth]{HZZ4lPlots/1minusProdIndep_SuperKD.eps}
\caption{Distributions of $\mathscr{D}_{bkg}$
(left) and $\mathscr{D}_{bkg}$ (right).  Expected distribution
for a 126~GeV SM Higgs boson is shown in red, the continuum ZZ
background in blue, and the reducible background in green.}
\label{fig:HZZ4lsuperMELA}
\end{center}
\end{figure}

\begin{figure}
\begin{center}
\includegraphics[trim=270 0 0 0,clip,width=.32\linewidth]{HZZ4lPlots/0minus_KD_superKDcut.eps}
\includegraphics[trim=270 0 0 0,clip,width=.32\linewidth]{HZZ4lPlots/0hplus_KD_superKDcut.eps}
\includegraphics[trim=270 0 0 0,clip,width=.32\linewidth]{HZZ4lPlots/1minus_KD_superKDcut.eps}\\
\includegraphics[trim=270 0 0 0,clip,width=.32\linewidth]{HZZ4lPlots/1plus_KD_superKDcut.eps}
\includegraphics[trim=270 0 0 0,clip,width=.32\linewidth]{HZZ4lPlots/2mplus_gg_KD_superKDcut.eps}
\includegraphics[trim=270 0 0 0,clip,width=.32\linewidth]{HZZ4lPlots/2mplus_qqbar_KD_superKDcut.eps}\\
\includegraphics[trim=270 0 0 0,clip,width=.32\linewidth]{HZZ4lPlots/2hplus_KD_superKDcut.eps}
\includegraphics[trim=270 0 0 0,clip,width=.32\linewidth]{HZZ4lPlots/2hminus_KD_superKDcut.eps}
\includegraphics[trim=270 0 0 0,clip,width=.32\linewidth]{HZZ4lPlots/2bplus_KD_superKDcut.eps}\\
\includegraphics[trim=270 0 0 0,clip,width=.32\linewidth]{HZZ4lPlots/2mplusProdIndep_KD_superKDcut.eps}
\includegraphics[trim=270 0 0 0,clip,width=.32\linewidth]{HZZ4lPlots/1plusProdIndep_KD_superKDcut.eps}
\includegraphics[trim=270 0 0 0,clip,width=.32\linewidth]{HZZ4lPlots/1minusProdIndep_KD_superKDcut.eps}
\caption{Distributions of $\mathscr{D}_{J^P}$ for $J^P=0^-$, 
$0_h^+$, and $1^-$ (first row), $J^P=1^+$, $2_m^+(gg)$, and 
$2_m^+(q\bar{q})$ (second row), $J^P=2_h^+$, $2_h^-$, and $2_b^+$ 
(third row), and production independent tests of $J^P=1^-$, $1^+$,
and $2_m^+$ (fourth row).  Expected shapes for a 125~GeV SM Higgs
boson is shown in red, the continuum background in blue, the
reducible background in green, and observed data in the point
with error bars.}
\label{fig:HZZ4ldjp}
\end{center}
\end{figure}

The effect of different couplings on the ZZ branching 
ratios as well as different relative efficiencies is accounted
for by calculating scale factors for each of the six different 
channels comparing SM Higgs against alternative $J^P$ samples 
with \verb+JHUGen+.  Tables~\ref{table:HZZ4lyieldcorr_spin0},
\ref{table:HZZ4lyieldcorr_spin1},
\ref{table:HZZ4lyieldcorr_spin2_min},
and \ref{table:HZZ4lyieldcorr_spin2_HD} show
 each of these scale
factors for all alternative signals in all channels.  The large
difference in the $q\bar{q}$ initiated samples are due to the
more forward rapidity distributions of these samples relative to 
the $gg$ initiated samples.  

 %%%%%%%%%%%%%%%%%%%%%%%%%%%%%%
\begin{table}[b]
\centering
\caption{
Table with correction factors and event yields in the different
channels of the alternative spin-0 hypotheses arising
due to lepton interference and detector effects.}
\centering % used for centering table
\begin{tabular}{c c c c c c c c} % centered columns (8 columns)
\hline \hline

 \multicolumn{8}{|c|}{$0^{+}_{m}$ $\sqrt{s}=7$~Tev} \\ \hline 

channel & $f_{i}^{J^P}$ & $\alpha_{\rm ideal} (i)$ & $\epsilon_{\rm reco} (i)$ & $\alpha_{\rm exp} (i)$ & $N^{J^P}_{\rm exp} (i)$ & $\alpha_{\rm norm} (i)$ & $N^{J^P}_{\rm norm} (i)$\\ \hline 
4e & 0.2592 &  1.0  & 0.254878 &  1.0  & 0.681158 &  1.0  & 0.681158 \\ \hline 
4mu & 0.2592 &  1.0  & 0.390734 &  1.0  & 1.05786 &  1.0  & 1.05786 \\ \hline 
2mu2e & 0.4816 &  1.0  & 0.305464 &  1.0  & 1.5215 &  1.0  & 1.5215 \\ \hline \hline 

 \multicolumn{8}{|c|}{$0^{+}_{m}$ $\sqrt{s}=8$~Tev} \\ \hline 

4e & 0.2592 &  1.0  & 0.209051 &  1.0  & 2.83281 &  1.0  & 2.83281 \\ \hline 
4mu & 0.2592 &  1.0  & 0.384041 &  1.0  & 5.20253 &  1.0  & 5.20253 \\ \hline 
2mu2e & 0.4816 &  1.0  & 0.279299 &  1.0  & 7.02377 &  1.0  & 7.02377 \\ \hline \hline  

\multicolumn{8}{|c|}{$0^{-}$ $\sqrt{s}=7$~Tev} \\ \hline

channel & $f_{i}^{J^P}$ & $\alpha_{\rm ideal} (i)$ & $\epsilon_{\rm reco} (i)$ & $\alpha_{\rm exp} (i)$ & $N^{J^P}_{\rm exp} (i)$ & $\alpha_{\rm norm} (i)$ & $N^{J^P}_{\rm norm} (i)$\\ \hline 
4e & 0.2382 & 0.845266 & 0.21946 & 0.730505
 & 0.497589% N^{Higg} = 18.3196
% N_{exp}^{JP} = 0.497589 + 0.858759 + 1.48305 + 2.08753 + 4.10322 + 6.76086 = 15.791
 & 0.847481 & 0.577268 \\ \hline 
4mu & 0.2382 & 0.845266 & 0.375617 & 0.811788
 & 0.858759% N^{Higg} = 18.3196
% N_{exp}^{JP} = 0.497589 + 0.858759 + 1.48305 + 2.08753 + 4.10322 + 6.76086 = 15.791
 & 0.94178 & 0.996272 \\ \hline 
2mu2e & 0.5236 & 1.0  & 0.298035 & 0.974732
 & 1.48305% N^{Higg} = 18.3196
% N_{exp}^{JP} = 0.497589 + 0.858759 + 1.48305 + 2.08753 + 4.10322 + 6.76086 = 15.791
 & 1.13082 & 1.72054 \\ \hline \hline 

\multicolumn{8}{|c|}{$0^{-}$ $\sqrt{s}=8$~Tev} \\ \hline 

4e & 0.2382 & 0.845266 & 0.182517 & 0.736911
 & 2.08753% N^{Higg} = 18.3196
% N_{exp}^{JP} = 0.497589 + 0.858759 + 1.48305 + 2.08753 + 4.10322 + 6.76086 = 15.791
 & 0.854913 & 2.4218 \\ \hline 
4mu & 0.2382 & 0.845266 & 0.358533 & 0.788697
 & 4.10322% N^{Higg} = 18.3196
% N_{exp}^{JP} = 0.497589 + 0.858759 + 1.48305 + 2.08753 + 4.10322 + 6.76086 = 15.791
 & 0.914991 & 4.76026 \\ \hline 
2mu2e & 0.5236 & 1.0  & 0.268579 & 0.962568
 & 6.76086% N^{Higg} = 18.3196
% N_{exp}^{JP} = 0.497589 + 0.858759 + 1.48305 + 2.08753 + 4.10322 + 6.76086 = 15.791
 & 1.1167 & 7.84348 \\ \hline \hline 
% Sum N_{norm}^{JP} = 0.577268 + 0.996272 + 1.72054 + 2.4218 + 4.76026 + 7.84348 = 18.3196

 \multicolumn{8}{|c|}{$0^{+}_{h}$ $\sqrt{s}=7$~Tev} \\ \hline 

channel & $f_{i}^{J^P}$ & $\alpha_{\rm ideal} (i)$ & $\epsilon_{\rm reco} (i)$ & $\alpha_{\rm exp} (i)$ & $N^{J^P}_{\rm exp} (i)$ & $\alpha_{\rm norm} (i)$ & $N^{J^P}_{\rm norm} (i)$\\ \hline 
4e & 0.2458 & 0.898313 & 0.271464 & 0.958688
 & 0.653018% N^{Higg} = 18.3196
% N_{exp}^{JP} = 0.653018 + 1.00605 + 1.70679 + 2.749 + 5.01137 + 7.67655 = 18.8028
 & 0.934054 & 0.636238 \\ \hline 
4mu & 0.2458 & 0.898313 & 0.42079 & 0.951022
 & 1.00605% N^{Higg} = 18.3196
% N_{exp}^{JP} = 0.653018 + 1.00605 + 1.70679 + 2.749 + 5.01137 + 7.67655 = 18.8028
 & 0.926585 & 0.980197 \\ \hline 
2mu2e & 0.5084 & 1.0  & 0.340119 & 1.12178
 & 1.70679% N^{Higg} = 18.3196
% N_{exp}^{JP} = 0.653018 + 1.00605 + 1.70679 + 2.749 + 5.01137 + 7.67655 = 18.8028
 & 1.09296 & 1.66294 \\ \hline \hline 

 \multicolumn{8}{|c|}{$0^{+}_{h}$ $\sqrt{s}=8$~Tev} \\ \hline 

4e & 0.2458 & 0.898313 & 0.223834 & 0.970414
 & 2.749% N^{Higg} = 18.3196
% N_{exp}^{JP} = 0.653018 + 1.00605 + 1.70679 + 2.749 + 5.01137 + 7.67655 = 18.8028
 & 0.945478 & 2.67836 \\ \hline 
4mu & 0.2458 & 0.898313 & 0.412882 & 0.963257
 & 5.01137% N^{Higg} = 18.3196
% N_{exp}^{JP} = 0.653018 + 1.00605 + 1.70679 + 2.749 + 5.01137 + 7.67655 = 18.8028
 & 0.938505 & 4.8826 \\ \hline 
2mu2e & 0.5084 & 1.0  & 0.306175 & 1.09294
 & 7.67655% N^{Higg} = 18.3196
% N_{exp}^{JP} = 0.653018 + 1.00605 + 1.70679 + 2.749 + 5.01137 + 7.67655 = 18.8028
 & 1.06486 & 7.4793 \\ \hline \hline 
% Sum N_{norm}^{JP} = 0.636238 + 0.980197 + 1.66294 + 2.67836 + 4.8826 + 7.4793 = 18.3196
\end{tabular}
\label{table:HZZ4lyieldcorr_spin0}
\end{table}

 %%%%%%%%%%%%%%%%%%%%%%%%%%%%%%

\begin{table}[b]
\centering
\caption{
Table with correction factors and event yields in the different
channels of the alternative spin-1 hypotheses arising
due to lepton interference and detector effects.}
\centering % used for centering table
\begin{tabular}{c c c c c c c c} % centered columns (8 columns)
\hline \hline

 \multicolumn{8}{|c|}{$1^{-}$ $\sqrt{s}=7$~Tev} \\ \hline 

channel & $f_{i}^{J^P}$ & $\alpha_{\rm ideal} (i)$ & $\epsilon_{\rm reco} (i)$ & $\alpha_{\rm exp} (i)$ & $N^{J^P}_{\rm exp} (i)$ & $\alpha_{\rm norm} (i)$ & $N^{J^P}_{\rm norm} (i)$\\ \hline 
4e & 0.2395 & 0.854121 & 0.127888 & 0.429419
 & 0.292502% N^{Higg} = 18.3196
% N_{exp}^{JP} = 0.292502 + 0.47399 + 0.837269 + 1.15378 + 2.34693 + 3.71105 = 8.81552
 & 0.89238 & 0.607852 \\ \hline 
4mu & 0.2395 & 0.854121 & 0.207372 & 0.448064
 & 0.47399% N^{Higg} = 18.3196
% N_{exp}^{JP} = 0.292502 + 0.47399 + 0.837269 + 1.15378 + 2.34693 + 3.71105 = 8.81552
 & 0.931127 & 0.985002 \\ \hline 
2mu2e & 0.521 & 1.0  & 0.167307 & 0.550292
 & 0.837269% N^{Higg} = 18.3196
% N_{exp}^{JP} = 0.292502 + 0.47399 + 0.837269 + 1.15378 + 2.34693 + 3.71105 = 8.81552
 & 1.14357 & 1.73994 \\ \hline \hline 

 \multicolumn{8}{|c|}{$1^{-}$ $\sqrt{s}=8$~Tev} \\ \hline 

4e & 0.2395 & 0.854121 & 0.100312 & 0.407292
 & 1.15378% N^{Higg} = 18.3196
% N_{exp}^{JP} = 0.292502 + 0.47399 + 0.837269 + 1.15378 + 2.34693 + 3.71105 = 8.81552
 & 0.846397 & 2.39768 \\ \hline 
4mu & 0.2395 & 0.854121 & 0.202707 & 0.451114
 & 2.34693% N^{Higg} = 18.3196
% N_{exp}^{JP} = 0.292502 + 0.47399 + 0.837269 + 1.15378 + 2.34693 + 3.71105 = 8.81552
 & 0.937464 & 4.87718 \\ \hline 
2mu2e & 0.521 & 1.0  & 0.147179 & 0.528356
 & 3.71105% N^{Higg} = 18.3196
% N_{exp}^{JP} = 0.292502 + 0.47399 + 0.837269 + 1.15378 + 2.34693 + 3.71105 = 8.81552
 & 1.09798 & 7.71197 \\ \hline \hline 
% Sum N_{norm}^{JP} = 0.607852 + 0.985002 + 1.73994 + 2.39768 + 4.87718 + 7.71197 = 18.3196

 \multicolumn{8}{|c|}{$1^{+}$ $\sqrt{s}=7$~Tev} \\ \hline 

channel & $f_{i}^{J^P}$ & $\alpha_{\rm ideal} (i)$ & $\epsilon_{\rm reco} (i)$ & $\alpha_{\rm exp} (i)$ & $N^{J^P}_{\rm exp} (i)$ & $\alpha_{\rm norm} (i)$ & $N^{J^P}_{\rm norm} (i)$\\ \hline 
4e & 0.2466 & 0.904082 & 0.151964 & 0.538705
 & 0.366943% N^{Higg} = 18.3196
% N_{exp}^{JP} = 0.366943 + 0.61119 + 0.990764 + 1.47037 + 2.97901 + 4.45948 = 10.8778
 & 0.907252 & 0.617982 \\ \hline 
4mu & 0.2466 & 0.904082 & 0.251755 & 0.57776
 & 0.61119% N^{Higg} = 18.3196
% N_{exp}^{JP} = 0.366943 + 0.61119 + 0.990764 + 1.47037 + 2.97901 + 4.45948 = 10.8778
 & 0.973026 & 1.02933 \\ \hline 
2mu2e & 0.5068 & 1.0  & 0.198025 & 0.651177
 & 0.990764% N^{Higg} = 18.3196
% N_{exp}^{JP} = 0.366943 + 0.61119 + 0.990764 + 1.47037 + 2.97901 + 4.45948 = 10.8778
 & 1.09667 & 1.66858 \\ \hline \hline 

 \multicolumn{8}{|c|}{$1^{+}$ $\sqrt{s}=8$~Tev} \\ \hline 

4e & 0.2466 & 0.904082 & 0.119758 & 0.519051
 & 1.47037% N^{Higg} = 18.3196
% N_{exp}^{JP} = 0.366943 + 0.61119 + 0.990764 + 1.47037 + 2.97901 + 4.45948 = 10.8778
 & 0.874151 & 2.4763 \\ \hline 
4mu & 0.2466 & 0.904082 & 0.242716 & 0.572609
 & 2.97901% N^{Higg} = 18.3196
% N_{exp}^{JP} = 0.366943 + 0.61119 + 0.990764 + 1.47037 + 2.97901 + 4.45948 = 10.8778
 & 0.964351 & 5.01706 \\ \hline 
2mu2e & 0.5068 & 1.0  & 0.177697 & 0.634913
 & 4.45948% N^{Higg} = 18.3196
% N_{exp}^{JP} = 0.366943 + 0.61119 + 0.990764 + 1.47037 + 2.97901 + 4.45948 = 10.8778
 & 1.06928 & 7.51037 \\ \hline \hline 
% Sum N_{norm}^{JP} = 0.617982 + 1.02933 + 1.66858 + 2.4763 + 5.01706 + 7.51037 = 18.3196
\end{tabular}
\label{table:HZZ4lyieldcorr_spin1}
\end{table}

 %%%%%%%%%%%%%%%%%%%%%%%%%%%%%%
\begin{table}[b]
\centering
\caption{
Table with correction factors and event yields in the different
channels of the alternative spin-2 hypotheses with minimal
couplings arising due to lepton interference and detector effects.}
\centering % used for centering table
\begin{tabular}{c c c c c c c c} % centered columns (8 columns)
\hline \hline

 \multicolumn{8}{|c|}{$2^{+}_{m} (gg)$ $\sqrt{s}=7$~TeV} \\ \hline 

channel & $f_{i}^{J^P}$ & $\alpha_{\rm ideal} (i)$ & $\epsilon_{\rm reco} (i)$ & $\alpha_{\rm exp} (i)$ & $N^{J^P}_{\rm exp} (i)$ & $\alpha_{\rm norm} (i)$ & $N^{J^P}_{\rm norm} (i)$\\ \hline 
4e & 0.2368 & 0.835494 & 0.22689 & 0.745966
 & 0.508121% N^{Higg} = 18.3196
% N_{exp}^{JP} = 0.508121 + 0.830746 + 1.47894 + 2.11001 + 4.08398 + 6.76734 = 15.7791
 & 0.866069 & 0.58993 \\ \hline 4mu & 0.2368 & 0.835494 & 0.368471 & 0.785308
 & 0.830746% N^{Higg} = 18.3196
% N_{exp}^{JP} = 0.508121 + 0.830746 + 1.47894 + 2.11001 + 4.08398 + 6.76734 = 15.7791
 & 0.911745 & 0.964499 \\ \hline 2mu2e & 0.5265 & 1.0  & 0.296789 & 0.97203
 & 1.47894% N^{Higg} = 18.3196
% N_{exp}^{JP} = 0.508121 + 0.830746 + 1.47894 + 2.11001 + 4.08398 + 6.76734 = 15.7791
 & 1.12853 & 1.71706 \\ \hline \hline 

 \multicolumn{8}{|c|}{$2^{+}_{m} (gg)$ $\sqrt{s}=8$~TeV} \\ \hline 

4e & 0.2368 & 0.835494 & 0.18665 & 0.744846
 & 2.11001% N^{Higg} = 18.3196
% N_{exp}^{JP} = 0.508121 + 0.830746 + 1.47894 + 2.11001 + 4.08398 + 6.76734 = 15.7791
 & 0.864769 & 2.44972 \\ \hline 
4mu & 0.2368 & 0.835494 & 0.361526 & 0.784999
 & 4.08398% N^{Higg} = 18.3196
% N_{exp}^{JP} = 0.508121 + 0.830746 + 1.47894 + 2.11001 + 4.08398 + 6.76734 = 15.7791
 & 0.911387 & 4.74151 \\ \hline 
2mu2e & 0.5265 & 1.0  & 0.268665 & 0.96349
 & 6.76734% N^{Higg} = 18.3196
% N_{exp}^{JP} = 0.508121 + 0.830746 + 1.47894 + 2.11001 + 4.08398 + 6.76734 = 15.7791
 & 1.11862 & 7.8569 \\ \hline \hline 
% Sum N_{norm}^{JP} = 0.58993 + 0.964499 + 1.71706 + 2.44972 + 4.74151 + 7.8569 = 18.3196

 \multicolumn{8}{|c|}{$2^{+}_{m} (q\bar{q})$ $\sqrt{s}=7$~TeV} \\ \hline 

channel & $f_{i}^{J^P}$ & $\alpha_{\rm ideal} (i)$ & $\epsilon_{\rm reco} (i)$ & $\alpha_{\rm exp} (i)$ & $N^{J^P}_{\rm exp} (i)$ & $\alpha_{\rm norm} (i)$ & $N^{J^P}_{\rm norm} (i)$\\ \hline 
4e & 0.2368 & 0.835494 & 0.180851 & 0.593713
 & 0.404413% N^{Higg} = 18.3196
% N_{exp}^{JP} = 0.404413 + 0.673168 + 1.21801 + 1.70669 + 3.2149 + 5.50743 = 12.7246
 & 0.854769 & 0.582233 \\ \hline 
4mu & 0.2368 & 0.835494 & 0.298801 & 0.636349
 & 0.673168% N^{Higg} = 18.3196
% N_{exp}^{JP} = 0.404413 + 0.673168 + 1.21801 + 1.70669 + 3.2149 + 5.50743 = 12.7246
 & 0.916151 & 0.969161 \\ \hline 
2mu2e & 0.5265 & 1.0  & 0.24418 & 0.800531
 & 1.21801% N^{Higg} = 18.3196
% N_{exp}^{JP} = 0.404413 + 0.673168 + 1.21801 + 1.70669 + 3.2149 + 5.50743 = 12.7246
 & 1.15253 & 1.75357 \\ \hline \hline 

 \multicolumn{8}{|c|}{$2^{+}_{m} (q\bar{q})$ $\sqrt{s}=8$~TeV} \\ \hline 

4e & 0.2368 & 0.835494 & 0.150986 & 0.602471
 & 1.70669% N^{Higg} = 18.3196
% N_{exp}^{JP} = 0.404413 + 0.673168 + 1.21801 + 1.70669 + 3.2149 + 5.50743 = 12.7246
 & 0.867378 & 2.45712 \\ \hline 
4mu & 0.2368 & 0.835494 & 0.284727 & 0.61795
 & 3.2149% N^{Higg} = 18.3196
% N_{exp}^{JP} = 0.404413 + 0.673168 + 1.21801 + 1.70669 + 3.2149 + 5.50743 = 12.7246
 & 0.889664 & 4.6285 \\ \hline 
2mu2e & 0.5265 & 1.0  & 0.218591 & 0.784113
 & 5.50743% N^{Higg} = 18.3196
% N_{exp}^{JP} = 0.404413 + 0.673168 + 1.21801 + 1.70669 + 3.2149 + 5.50743 = 12.7246
 & 1.12889 & 7.92905 \\ \hline \hline 
% Sum N_{norm}^{JP} = 0.582233 + 0.969161 + 1.75357 + 2.45712 + 4.6285 + 7.92905 = 18.3196

 \multicolumn{8}{|c|}{$2^{+}_{b}$ $\sqrt{s}=$8~TeV} \\ \hline 

channel & $f_{i}^{J^P}$ & $\alpha_{\rm ideal} (i)$ & $\epsilon_{\rm reco} (i)$ & $\alpha_{\rm exp} (i)$ & $N^{J^P}_{\rm exp} (i)$ & $\alpha_{\rm norm} (i)$ & $N^{J^P}_{\rm norm} (i)$\\ \hline 
4e & 0.234 & 0.81758 & 0.222087 & 0.725251
 & 0.494011% N^{Higg} = 18.3196
% N_{exp}^{JP} = 0.494011 + 0.786165 + 1.45677 + 2.09386 + 3.80295 + 6.64082 = 15.2746
 & 0.869832 & 0.592493 \\ \hline 
4mu & 0.234 & 0.81758 & 0.35873 & 0.743164
 & 0.786165% N^{Higg} = 18.3196
% N_{exp}^{JP} = 0.494011 + 0.786165 + 1.45677 + 2.09386 + 3.80295 + 6.64082 = 15.2746
 & 0.891317 & 0.942889 \\ \hline 
2mu2e & 0.5319 & 1.0  & 0.293403 & 0.957458
 & 1.45677% N^{Higg} = 18.3196
% N_{exp}^{JP} = 0.494011 + 0.786165 + 1.45677 + 2.09386 + 3.80295 + 6.64082 = 15.2746
 & 1.14833 & 1.74718 \\ \hline \hline 

 \multicolumn{8}{|c|}{$2^{+}_{b}$ $\sqrt{s}=$8~TeV} \\ \hline 

4e & 0.234 & 0.81758 & 0.185353 & 0.739147
 & 2.09386% N^{Higg} = 18.3196
% N_{exp}^{JP} = 0.494011 + 0.786165 + 1.45677 + 2.09386 + 3.80295 + 6.64082 = 15.2746
 & 0.886499 & 2.51128 \\ \hline 
4mu & 0.234 & 0.81758 & 0.346648 & 0.730982
 & 3.80295% N^{Higg} = 18.3196
% N_{exp}^{JP} = 0.494011 + 0.786165 + 1.45677 + 2.09386 + 3.80295 + 6.64082 = 15.2746
 & 0.876706 & 4.56109 \\ \hline 
2mu2e & 0.5319 & 1.0  & 0.265235 & 0.945478
 & 6.64082% N^{Higg} = 18.3196
% N_{exp}^{JP} = 0.494011 + 0.786165 + 1.45677 + 2.09386 + 3.80295 + 6.64082 = 15.2746
 & 1.13396 & 7.96469 \\ \hline \hline 
% Sum N_{norm}^{JP} = 0.592493 + 0.942889 + 1.74718 + 2.51128 + 4.56109 + 7.96469 = 18.3196
\end{tabular}
\label{table:HZZ4lyieldcorr_spin2_min}
\end{table}


 %%%%%%%%%%%%%%%%%%%%%%%%%%%%%%
\begin{table}[b]
\centering
\caption{
Table with correction factors and event yields in the different
channels of the alternative spin-2 hypotheses with high dimensional
couplings arising due to lepton interference and detector effects.}
\begin{tabular}{c c c c c c c c} % centered columns (8 columns)
\hline \hline

\multicolumn{8}{|c|}{$2^{+}_{h}$ $\sqrt{s}=7$~TeV} \\ \hline 

channel & $f_{i}^{J^P}$ & $\alpha_{\rm ideal} (i)$ & $\epsilon_{\rm reco} (i)$ & $\alpha_{\rm exp} (i)$ & $N^{J^P}_{\rm exp} (i)$ & $\alpha_{\rm norm} (i)$ & $N^{J^P}_{\rm norm} (i)$\\ \hline 
4e & 0.2453 & 0.894726 & 0.223832 & 0.791281
 & 0.538988% N^{Higg} = 18.3196
% N_{exp}^{JP} = 0.538988 + 0.845455 + 1.44081 + 2.2683 + 4.12916 + 6.56791 = 15.7906
 & 0.918012 & 0.625311 \\ \hline 
4mu & 0.2453 & 0.894726 & 0.357244 & 0.799212
 & 0.845455% N^{Higg} = 18.3196
% N_{exp}^{JP} = 0.538988 + 0.845455 + 1.44081 + 2.2683 + 4.12916 + 6.56791 = 15.7906
 & 0.927213 & 0.980862 \\ \hline 
2mu2e & 0.5094 & 1.0  & 0.286971 & 0.946968
 & 1.44081% N^{Higg} = 18.3196
% N_{exp}^{JP} = 0.538988 + 0.845455 + 1.44081 + 2.2683 + 4.12916 + 6.56791 = 15.7906
 & 1.09863 & 1.67157 \\ \hline \hline 

\multicolumn{8}{|c|}{$2^{+}_{h}$ $\sqrt{s}=8$~TeV} \\ \hline 

4e & 0.2453 & 0.894726 & 0.188832 & 0.800725
 & 2.2683% N^{Higg} = 18.3196
% N_{exp}^{JP} = 0.538988 + 0.845455 + 1.44081 + 2.2683 + 4.12916 + 6.56791 = 15.7906
 & 0.928968 & 2.63159 \\ \hline 
4mu & 0.2453 & 0.894726 & 0.343297 & 0.793683
 & 4.12916% N^{Higg} = 18.3196
% N_{exp}^{JP} = 0.538988 + 0.845455 + 1.44081 + 2.2683 + 4.12916 + 6.56791 = 15.7906
 & 0.920798 & 4.79048 \\ \hline 
2mu2e & 0.5094 & 1.0  & 0.259049 & 0.935098
 & 6.56791% N^{Higg} = 18.3196
% N_{exp}^{JP} = 0.538988 + 0.845455 + 1.44081 + 2.2683 + 4.12916 + 6.56791 = 15.7906
 & 1.08486 & 7.61982 \\ \hline \hline 
% Sum N_{norm}^{JP} = 0.625311 + 0.980862 + 1.67157 + 2.63159 + 4.79048 + 7.61982 = 18.3196

 \multicolumn{8}{|c|}{$2^{-}_{h}$ $\sqrt{s}=7$~TeV} \\ \hline 

channel & $f_{i}^{J^P}$ & $\alpha_{\rm ideal} (i)$ & $\epsilon_{\rm reco} (i)$ & $\alpha_{\rm exp} (i)$ & $N^{J^P}_{\rm exp} (i)$ & $\alpha_{\rm norm} (i)$ & $N^{J^P}_{\rm norm} (i)$\\ \hline 
4e & 0.2426 & 0.875596 & 0.205982 & 0.715726
 & 0.487522% N^{Higg} = 18.3196
% N_{exp}^{JP} = 0.487522 + 0.792493 + 1.29849 + 2.08139 + 3.90183 + 5.95518 = 14.5169
 & 0.903211 & 0.615229 \\ \hline 
4mu & 0.2426 & 0.875596 & 0.336909 & 0.749146
 & 0.792493% N^{Higg} = 18.3196
% N_{exp}^{JP} = 0.487522 + 0.792493 + 1.29849 + 2.08139 + 3.90183 + 5.95518 = 14.5169
 & 0.945386 & 1.00009 \\ \hline 
2mu2e & 0.5148 & 1.0  & 0.26108 & 0.853431
 & 1.29849% N^{Higg} = 18.3196
% N_{exp}^{JP} = 0.487522 + 0.792493 + 1.29849 + 2.08139 + 3.90183 + 5.95518 = 14.5169
 & 1.07699 & 1.63864 \\ \hline \hline 

 \multicolumn{8}{|c|}{$2^{-}_{h}$ $\sqrt{s}=8$~TeV} \\ \hline 

4e & 0.2426 & 0.875596 & 0.172541 & 0.734743
 & 2.08139% N^{Higg} = 18.3196
% N_{exp}^{JP} = 0.487522 + 0.792493 + 1.29849 + 2.08139 + 3.90183 + 5.95518 = 14.5169
 & 0.927209 & 2.62661 \\ \hline 
4mu & 0.2426 & 0.875596 & 0.330978 & 0.749988
 & 3.90183% N^{Higg} = 18.3196
% N_{exp}^{JP} = 0.487522 + 0.792493 + 1.29849 + 2.08139 + 3.90183 + 5.95518 = 14.5169
 & 0.946448 & 4.92392 \\ \hline 
2mu2e & 0.5148 & 1.0  & 0.237978 & 0.847861
 & 5.95518% N^{Higg} = 18.3196
% N_{exp}^{JP} = 0.487522 + 0.792493 + 1.29849 + 2.08139 + 3.90183 + 5.95518 = 14.5169
 & 1.06996 & 7.51514 \\ \hline \hline 
% Sum N_{norm}^{JP} = 0.615229 + 1.00009 + 1.63864 + 2.62661 + 4.92392 + 7.51514 = 18.3196
\end{tabular}
\label{table:HZZ4lyieldcorr_spin2_HD}
\end{table}

The test statistic used to distinguish the null hypothesis from
the alternative hypothesis is a log-likelihood ratio,  
$q=ln(\mathscr{L}_{SM}/\mathscr{L}_{J^P})$. Expected results are
obtained in two different ways, generating pseudoexperiments using
the SM Higgs cross section for each hypothesis, 
or using the best-fit signal strength modifying, $\mu$, for each
hypothesis individually.  Since the expected production cross 
section for alternative signal 
models is highly model independent, using the best-fit signal
strength for generating toys allows for a more model independent 
interpretation.

Results are shown in 
table~\ref{table:HZZ4lhypothTests} where observed $0^+$ ($J^P$) 
refers to the p-value 
of the observed test statistic, represented by the red arrow,
calculated according to the SM (alternative signal) toy 
distribution, shown in yellow (blue), converted to normal 
quantiles.  A $CL_s$ criterion is built from the p-values 
according to:
\begin{equation}
CL_s = P(q>q_0|SM)/P(q>q_0|J^P)
\end{equation}  
All results show that data is more consistent with 
the SM expectation and disfavor the alternative hypothesis
at a level of 8.1\% or better.  

Several results show large observed significance with respect to 
the expected, namely the $1^+$, $1^-$, and $2_{m,q\bar{q}}^+$ tests.
Each of these cases have $m_{Z}$ and $m_{Z*}$ distributions which 
are quite distinct from a SM Higgs boson.  As a result of a 
statistical
fluctuation observed in data in the tails of these distributions,
these models all have large q-values.  This is one of the driving
factors to why the discovery significance is larger than expected.
However, it is important
to note that not only are these results correlated as a result
of this statistical fluctuation.

\begin{figure}
\begin{center}
\includegraphics[width=.32\linewidth]{HZZ4lPlots/separation_0mp.eps}
\includegraphics[width=.32\linewidth]{HZZ4lPlots/separation_1minus.eps}
\includegraphics[width=.32\linewidth]{HZZ4lPlots/separation_gg2pmin.eps}
\includegraphics[width=.32\linewidth]{HZZ4lPlots/separation_0hp.eps}
\includegraphics[width=.32\linewidth]{HZZ4lPlots/separation_1p.eps}
\includegraphics[width=.32\linewidth]{HZZ4lPlots/separation_qq2pmin.eps}
\includegraphics[width=.32\linewidth]{HZZ4lPlots/separation_2hplus.eps}
\includegraphics[width=.32\linewidth]{HZZ4lPlots/separation_2hminus.eps}
\includegraphics[width=.32\linewidth]{HZZ4lPlots/separation_2bplus.eps}
\includegraphics[width=.32\linewidth]{HZZ4lPlots/separation_1minus_prodind.eps}
\includegraphics[width=.32\linewidth]{HZZ4lPlots/separation_1plus_prodind.eps}
\includegraphics[width=.32\linewidth]{HZZ4lPlots/separation_2mplus_prodind.eps}
\caption{Distribution of expected and observed test statistics 
for various hypothesis test.  Orange histograms represent toys
generated under the null hypothesis, SM background plus a SM 
Higgs boson.  Blue histograms represent toys generated under the
alternative hypothesis.  The red arrow shows the value of the 
observed test statistic.  All resonances are assumed to have a 
mass of 126~GeV.}
\label{fig:HZZ4lhypothTests}
\end{center}
\end{figure}

%-----------

\begin{table}[h]
\centering
\begin{tabular}{|c|c|c|c|c|c|} 
\hline%--------------------------------------------------------------------------------
 $J^P$ model & $J^P$ production & expect ($\mu$=1) &  obs. $0^+$  & obs. $J^P$ & CL$_s$  \\
\hline%---------------------------------------------------------------------------------
$0^-$     & any         &  2.6$\sigma$ (2.7$\sigma$)  & -0.9$\sigma$ & +3.8$\sigma$  &  0.04\%  \\
$0_h^+$  & any         &  1.8$\sigma$ (1.9$\sigma$)  & +0.3$\sigma$ &  +1.6$\sigma$  &  10.0\% \\
$1^-$     & $q \bar q\to X$         &  2.9$\sigma$ (3.1$\sigma$)  & -1.5$\sigma$ &  $>5.0\sigma$  &  $<$0.1\% \\
$1^-$   & any       &  2.7$\sigma$ (3.0$\sigma$)  & -1.8$\sigma$ &  $>5\sigma$    &  $<$0.1\% \\
 $1^+$    & $q \bar q\to X$          &  2.4$\sigma$ (2.6$\sigma$)  & -1.4$\sigma$  & +4.5$\sigma$  &  0.004\% \\
 $1^+$   & any           &  2.3$\sigma$ (2.4$\sigma$)  & -2.1$\sigma$ &  $>5\sigma$    &  $<$0.1\% \\
$2_{~m}^+$   & $gg \to X$  &  1.8$\sigma$ (1.9$\sigma$)  & -0.7$\sigma$ & +2.7$\sigma$  &  1.5\% \\
$2_{~m}^+$  & $q \bar q\to X$ &  1.8$\sigma$ (1.9$\sigma$)  & -1.6$\sigma$ & +3.8$\sigma$  &  0.16\% \\
 $2_{\rm m}^+$  & any     &  1.5$\sigma$ (1.6$\sigma$)  & -1.5$\sigma$ & +3.3$\sigma$  &  0.8\% \\
$2_{~b}^+$      & $gg \to X$     &  1.8$\sigma$ (1.9$\sigma$)  & -1.1$\sigma$ & +3.2$\sigma$  &  0.5\% \\
$2_{~h}^+$    & $gg \to X$       &  3.9$\sigma$ (4.1$\sigma$)  & +2.0$\sigma$ & +1.9$\sigma$  &  3.0\% \\
$2_{~h}^-$     & $gg \to X$      &  4.4$\sigma$ (4.8$\sigma$)  & +1.1$\sigma$ & +3.3$\sigma$  &  0.05\% \\
\hline%----------------------------------------------------------------------------------
\end{tabular}
\caption{ List of models used in analysis of spin-parity hypotheses
  corresponding to the pure states of the type noted.  The expected
  separation is quoted for two scenarios, when the signal strength for
  each hypothesis is pre-determined from the fit to data and when
  events are generated with SM expectation for the signal yield
  ($\mu$=1).  The observed separation quotes consistency of the
  observation with the $0^+$ model or $J^P$ model, and corresponds to
  the scenario when the signal strength is pre-determined from the fit
  to data.  The last column quotes CL$_s$ criterion for the $J^P$
  model.  }
\label{table:HZZ4lhypothTests}
\end{table}

Of the 8 results presented, one of the most interesting is the 
$0^-$ hypothesis separation.  Since one of the simplest extensions
to the SM, the class of two-Higgs-doublet models (2HDMs), 
predicts the presence of a pure pseudo-scalar, ruling out the 
pure $0^-$ model suggests that the observed resonance is very 
unlikely to be a pure pseudoscalar. 

\subsection{Constraining CP-violation}
\label{sec:HZZ4lcpViolation}

As discussed in chapter~\ref{sec:intro}, one solution to 
fine tuning, SUSY, also predicts an additional scalar doublet, 
resulting in four new 
scalar resonances, two charge and two neutral, one of which 
is CP-odd.  
The CP-even and CP-odd
scalars that are predicted by SUSY and the more generic class
of 2HDM models can also mix to produce a parity-violating 
resonance.  
It has been recently argued~\cite{Shu:2013uua} that
the additional amount of CP violation which is still allowed by
experimental data could explaining the observed baryon assymetry
in the universe.  Thus, constraining CP-violation in the HZZ 
amplitude is one of the most promising ways of probing
new physics of the SM which could help to explain not only 
theoretical problems the SM is thought to suffer from, e.g. 
fine tuning, but emperical facts the SM is currently thought to
be insufficient to explain.  

The parameter $f_{a3}$, described in section~\ref{sec:HiggsPhen},
is a natural gauge of CP-violation in the HZZ amplitude.  Given 
that $f_{a3}=1$
has been ruled out through hypothesis testing in favor of the
SM Higgs hypothesis at the level of $3.3\sigma$,  measuring any
non-zero value of $f_{a3}$ would be direct evidence of CP-violation
given that $f_{a2}$ is zero.  Furthermore, the $D_{0^-}$ variable 
used for hypothesis testing in section~\ref{sec:HZZ4lspinParity} 
is suitable for measuring the value of $f_{a3}$ using the 
simplified model for a mixed-CP state describe in 
equation~\ref{eq:fa3}

Using this model, a two parameter fit for $\mu$ and $f_{a3}$ was 
performed.  
Figure~\ref{fig:HZZ4lfa3Fit2D} shows the $\ln\mathscr{L}$ scan as a 
function of the two parameters.  The best-fit values are 
found to be $\mu=X.XX$, $f_{a3}=0.0$.  Profiling $\mu$, we 
arrive at the 1D $\ln\mathscr{L}$ scan versus $f_{a3}$ in 
figure~\ref{fig:HZZ4lfa3Fit1D}.  The expected 68\% and 95\% CL 
intervals, from fitting the asimov dataset, are found to be 
[0.0,0.39] and [0.0,0.69], respectively. 
The observed  68\% and 95\% CL 
intervals are found to be [0.00,0.16] and [0.00,0.49], 
respectively. 

\begin{figure}
\begin{center}
\includegraphics[width=0.49\linewidth]{HZZ4lPlots/can_scan2D_x.eps}
\caption{Distribution of $-2\ln\mathscr{L}$ versus ($\mu$,$f_{a3}$),
Blue and teal band represent the 68\% and 95\% cofidence level
contours, respectively.  The point represents the location of the
maximum likelihood.}
\label{fig:HZZ4lfa3Fit2D}
\end{center}
\end{figure}

\begin{figure}
\begin{center}
\includegraphics[width=0.49\linewidth]{HZZ4lPlots/can_scan1D_x.eps}
\caption{Distribution of $-2\ln\mathscr{L}$ versus $f_{a3}$.
The black line in the right plots 
represents the expected distribution calculated from fitting 
the asimov dataset; the blue line represents the observed 
distribution. The signal strength, $\mu$, has been profiled.}
\label{fig:HZZ4lfa3Fits1D}
\end{center}
\end{figure}


\section{Summary}

A search for a SM Higgs boson decaying into two Z boson which
subsequently decay into to quark jets and two leptons has been 
presented. The data used in this analysis constitute $4.6~fb^{-1}$ 
of integrate luminosity.  No significant excess of events was 
found and upper limits have been measure in the context of the 
SM and SM4 models.  Although only a small portion of Higgs masses 
have been excluded at 95\% CL, more data should allow sensitivity 
which is sufficient for excluding almost the entire range between 
200 and 600~GeV with this channel alone.  

A search for a SM Higgs boson decaying into two Z boson which
subsequently decay into 4 lepton has been presented.  The data
used in this analysis constitute $5.1~fb^{-1}$ and $19.8~fb^{-1}$ 
at $\sqrt{s}=7$ and 8~TeV, respectively.  An excess of events
has been observed around 126~GeV.  The properties of these 
events have been analyzed in the context of the mass and angular
distributions of the final state product using the MELA techniques
outlined in chapter~\ref{sec:HiggsPhen}.  Hypothesis testing 
shows that data favors the SM Higgs boson hypothesis over all 
others tested, although results for the $2_h^+$ are largely
inconclusive.  Measurement of the scalar model parameter $f_{a3}$
has also been presented and found to be consistent with zero.  
The 95\% confidence interval is [0.0,0.49], thus providing a 
direct constraint on the level of CP-violation in the HZZ
amplitude. At higher values of $m_{4\ell}$, the data is consistent
with the background only hypothesis.  In light of this, limits
have been set on $\sigma/\sigma_{SM}$ and SM Higgs boson masses 
in the range [114.5,119] and [129,800] have been ruled out.  
