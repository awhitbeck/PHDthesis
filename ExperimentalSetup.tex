\newcommand{\microns}{$\mu m$}


\chapter{Experimental Setup}
\label{sec:ExpSetup}
\chaptermark{Experimental Setup}


\section{The Large Hadron Collider}
\label{sec:LHC}
The Large Hadron Collider was designed to accelerates two beams of protons to up to 
energies of 7 TeV using a 27 km storage ring and  1232 individual 8.33 T dipole magnets. 
Although it is also capable of accelerating heavier nuclei up to energies of 
2.76 TeV, heavy ion physics is outside the scope of this work.  The proton
energies accessible to the LHC are a factor of seven times higher than its
most advanced predicessor, the Tevatron.  These energies are not only 
important for accessing new particles which might exist at large invariant
mass, on the order of serveral TeV, they are also necessary for efficient 
production of moderately heavy particles, like the Higgs boson or the top 
quark.  For a 125~GeV Higgs boson these energies provides a 
factor of XXX in total cross section over the production cross section at
the Tevatron.  

The LHC also has the capability to collide bunches of $1\times10^{11}$
protons every 25 ns at $\beta*=.55$ and $\sigma*=16.7$.  These
parameters and others, summarized in table~\ref{table:LHCparameters},
combine to allow the LHC to produce instantaneous luminosities 
of up to $10^{34} cm^{-2}s^{-1}$ accoding to
\begin{equation}
\mathscr{L} = \frac{\gamma f k_B N_p^2}{4\pi \epsilon_n\beta^*}F,
\end{equation}
where $\gamma$ is the Lorentz factor, f is the revolution frequency,
$k_B$ is the number of protons per bunch, $\epsilon_n$ is the betatron
function at the interaction point, and F is the reduction factor
due to the crossing angle.  This translates to roughly 1 billion proton
proton interactions per second and up to 50 collisions per bunch crossing, 
known as pileup.  

\begin{table}
\begin{center}
\begin{tabular}{l|c|c}
\hline 
\hline
Energy per nucleon           & $E$           & 7 TeV    \\
Dipole field at 7 TeV        & $B$           & 8.33 T   \\
Design Luminosity            & $\mathscr{L}$ & $10^{34}~cm^{-2}s^{-1}$\\
Bunch separation             &               & 25 ns    \\
No. of bunches               & $k_B$         & 2808      \\
No. of particles/bunch       & $N_p$         & $1.15\times 10^{11}$\\ \hline
\multicolumn{3}{l}{{\bf Collisions}} \\ \hline
$\beta$-value at IP          & $\beta^*$     & 0.55 m   \\
RMS beam radius at IP        & $\sigma^*$    & $16.7~\mu m$  \\
Luminosity lifetime          & $\tau_L$      & 15 hr         \\
Number of collisions/crossing& $n_c$         & $\equiv20$      \\
\hline 
\hline
\end{tabular}
\caption{Table listing relevant operational parameters of the LHC.}
\label{table:LHCparameters}
\end{center}
\end{table}

These conditions provide unprecidented conditions to probe the SM and 
discovering new particles but also extreme conditions for reconstructing
particle paths and energy deposits with a high degree of accuracy and efficiency.
The inclusive proton-proton
cross section at 14~TeV is approximately 100~mb.  At design luminosity, this
corresponds to an event rate of $10^9$~Hz.  
The large number of proton-proton
collisions produce a considerable amount of background noise which can 
either produce extra particles from secondary interactions, also known as pileup, as well an overall
increase in the energy deposited in the calorimeters.  The  high rate of the 
LHC far exceed the capabilities of the Data Aquisition (DAQ) system.  As a 
result it is necessary to use fast hardware logic to filter the vast majority of 
events. 

The short time between bunch crossings also puts significant constraints on 
detectors since subdetectors should have fast response times and low occupancy. 
High grainularity tracking will be necessary for high precision vertexing in
order to mitigate the effects of pileup.

\section{The Compact Muon Solenoid}
\label{sec:CMS}

The Compact Muon Solenoid (CMS), is a general purpose particle detector.  It
was designed to not only have a broad scope of discovery potential but also
to mitigate the extreme conditions created by the LHC.  CMS is made up of
several different types of apparati designed to improve identification of 
particles and measure their properties.  There is a two-stage trigger system
to filter the extreme rates comming from the LHC.  There is an all silicon 
tracking system at the center to carefully record the positions of charged 
particles passing through the detector.  There is a 4 Tesla magnet to bend 
charge particles 
providing the tracker and muon system sensitivity to the momentum of charged 
particles.  There are two calorimeters designed to induce particle showers 
which can then be used to measure energy deposits.  Finally, there is a Muon 
system 
at the edge of the detector to detect semi-stable, charged particles with long 
interaction lengths, e.g. the muon. This chapter describes these detectors in
detail. 

\subsection{Magnet}
\label{sec:Magnet}

CMS employs a 4~T superconducting aluminum solenoid magnet to bend tracks for 
both charge identification and momentum resolution.  The field was
chosen to have good momentum resolution, $\Delta p/p\equiv10\%$ at 
$p=1~TeV/c$.  The magnet has an inner bore of 5.9~m, large enough to house
the tracker and both calorimeters, and a length of 12.9~m.  Drawing a 
current of 19.5~kA, the magnets total stored energy is 2.7~GJ.  Making it
one of the largest magnets in the world.  
The outer return yolk of the magnet concetrates the magnetic field in the
region near the muon system, which is placed outside of the solenoid.  

\subsection{ElectroMagnetic Calorimeter}
\label{sec:ECal}

The Electromagnetic calorimeter (ECal) is a high grainularity calorimeter
intended to induce electromagnetic showers which are collected by either
avalanche photodiodes (barrel) or vacuum phototriodes (endcap).  The 
material used is scintillating lead tungstate crystal
which was chosen its: short radiation ($X_0$=0.89~cm) and Moliere (2.2~cm) 
lengths; the time scale in which showers occur, 80\% of light is emmitted
in 25~ns; and the radiation hardness.  The ECal is divided into barrel (EB)
and endcap (EE) regions.

The EB region has an inner radius of 129~cm and is constructed from 36
identical {\it supermodules}, each covering half of the barrel in the 
z-direction (1.479 unit of pseudorapidity).  Each individual crystal 
1 degree in both $\Delta\phi$ and $\Delta\eta$, corresponding to a cross
sectional area of $22\times22~mm^2$, and are 230~mm, corresponding to 
25.8~$X_0$.

The EE region is located at a distrance of 314~cm along the z-direction
and covering the pseudorapidity range $1.479<|\eta|<3.0$.  The crystals
clustered into $5\times5$ {\it supercrystals} which are combined to form
semi-circle structures.  Each crystal has a cross sectional area of 
$28.6\times28.6~mm^2$ and are 220 mm (24.7~$X_0$) in length.  The endcap 
region is also preceded by a preshower which consists of a lead absorber
whos thickness is 2-3~$X_0$ followed by 2 planes of silicon strip detectors.

The energy response of the ECal was measured in test beams.  The energy 
resolution was parameterized according to
\begin{equation}
\left(\frac{\sigma}{E}\right)^2 = \left(\frac{S}{\sqrt{E}}\right)^2 + \left(\frac{N}{E}\right)^2 + C^2,
\end{equation}
where S, N and C represent the stochastic, noise, and constant contributions.
The coefficients measurements for two energy clustering schemes are shown
if figure~\ref{fig:???}.

\subsection{Hadronic Calorimeter}
\label{sec:HCal}

The hadronic calorimeter (HCal) consists of brass absorbers and plastic 
scintillators
in which light is collected from using wavelength-shifting (WLS) fibers. 
Fiber cables transmit light into hybrid photodiodes.  The HCal is separated
into four regions, the barrel (HB), the outer (HO), the endcap (HE), 
and the forward (HF) region.

The HB is made up of 32 towers which cover the pseudorapidity region 
$|\eta|<1.4$, resulting in 2304 towers with a segmetation of 
$\Delta\eta\time\Delta\phi=0.087\time0.087$.  There are 15 brass plates, 
each 5~cm thick and two steel plates for structural stability. Particles 
entering the HCal barrel region fist impinge upon a scillating layer that
is 9~mm thick, instead of the typical 3.7~mm for other scintillating layers.
More details of the HB design and test beam performance can be found
elsewhere~\cite{??}.

The HO region contains 10~mm thick scintillators.  Each scintillating tile
matches the segmentation pattern of the muon stystem's Drift tubes.  
The purpose of the HO is to catch hadronic showers leaking through the 
HB region. This makes the effective length of the barrel region 10~$X_0$
and improves missing transverse energy $E_T^{miss}$ resolution.

The HE region consists of 14 $\eta$ towers with 5 degree segmentation in 
$\phi$ and covers the region between $1.3<|\eta|<3.0$. {\it ... more about the
geometry of the HE...} There are 2304 HE towers in total.  more details of
the design and test beam performance of the HE can be found 
elsewhere~\cite{??}.

The HF region extends between $3.0<|\eta|<5.0$ and is made from steel absober
and quartz fibers.  The fibers are intended to measure Cherenkov radiation.  
The HF will mainly be used for detecting very forward jets and real-time
luminosity measurements.  

\subsection{Muon System}
\label{sec:Muon System}

The Muon system plays an important role in identifying muons.  However, 
because of the vast distance from the interaction point and the muon 
chambers, resolution of low energy muons is dominated by energy loss 
due to multiple scattering in the inner detector.  In this region, it 
is found that the tracker dominates the momentum resolution.  However, 
for high energy tracks, the combination of the tracker and muon system
provide superior energy resolution to either system alone.  This effect
is shown in figure~\ref{fig:???} for the barrel and endcap region.  Thus
for high momentum muons, the muon system plays a major role in momentum 
resolution. 

The muon system employs three different gaseous detectors, drift tube (DT)
chambers, cathode strip chambers (CSC), and resistive plate chambers (RPC).
The DT are used in the barrel region, $|\eta|<1.2$, where the magnetic 
field is low.  The CSC detectors are used in the endcaps, 
$1.2<|\eta|<2.4$, where the rates of both muons and the neutron induced
background are high as well and high magnetic field.  The RPC detectors
are used both in barrel and endcaps.  

The RPCs are fast response and have
good timing resolution, while not as sensitive to spatial measurements
as the DTs and CSCs.  Thus, RPC provide the necessary input to distinguish
which bunch crossing a particles should be identified with, which is critical
for triggering.  All three sub-systems participate in level-1 triggering
and provide a key element to level-1 triggering.

The DTs are arranged in four layers of wheels made up of 12 segements
each covering 30 azimuthal degrees.  The outermost layer has 1 extra 
segment in the top an bottom, totaling 14.  Each DT is pair with either
one or two RPCs, two on either side in the first two layers and one on
the inner most edge in the second two layers.  A high-$p_T$ track can 
cross up to 6 RPCs and 4 DTs, providing 44 measurements for momentum 
measurements. 

The CSC are trapesoidal chambers containing 6 gas gaps, each with a 
corresponding cathode strips running radially and anode wires running
azimuthally.  Charge from iozined gas is collected on strips and wires.
Signals on the wires are fast and can be used for level-1 triggering,
while cathodes provide a better measurement of position, on the order
of 200~\microns.

\subsection{Trigger and data aquisition}

The event rate delivered to CMS is approximately $10^9$~Hz.  
However, only about 100~Hz can be processed by CMS.  
This requires a large, yet efficient, rejection scheme.  
CMS employs a two level system to make fast decisions on 
which events to keep.  The level-1 system consists of 
custom electronics which monitor the activity in the calorimeters
and the muon system.   Decisions are based off of raw energy
and momentum thresholds.  The level-1 system reduces the
event rate down to roughly 100~kHz while the High-Level 
trigger (HLT), an online
processing farm which executes reconstruction software, 
further reduces the rate to 100~HZ.  Customized HLT selections
are designed to ensure high efficiencies for different 
physics signatures.  

%\subsection{Track Reconstruction}
%\label{sec:trackRECO}
%
%\subsection{Muon Reconstruction}
%\label{sec:muonRECO}
%
%
%Muon tracks are reconstructed using the Kalman filter technique
%starting with RecHits at the inner most radius.
%In the barrel region, where the DTs provide the most sensitive 
%information about particle position, primitive track segments
%are reconstructed and fed to the Kalman filter.
%In the endcap, 3D hits are constructed from wire and strip 
%information which is then pass to the Kalman filter. In both
%cases, the RecHits from the RPCs are used.  Track states are 
%propagated through each layer taking into account the effects
%of energy loss in material, multiple scattering and inhomogeneities
%in the magentic field.  Once the outer most measurement is reached,
%the Kalman filter is then applied in reverse to define the final
%set of track parameters.  After the track is extrapolated to the 
%to the nominal interaction point, the beam-spot, a vertex-constrained
%fit to the track parameters is performed.
%Muons can also be reconstructed using the information from the 
%silicon tracker.  In this case, tracker hits which are compatible
%with the muon track are added to the Kalman filter.
%
%Muon reconstruction typically makes use of other subdetectors, such
%as the calorimeters.  Calorimeter cells which are compatible with 
%the extrapolated muon track should have energy deposits which are 
%consistent with the presence of a minimum ionizing particle.  
%Since events from pile-up can contribute the HO can be efficiently
%used to discriminate against pile-up activity.  
%
%Muon are also typically required to be well isolated.  An isolation
%cone is defined around the muon direction and a maximum threshold of 
%$E_T$ is allowed from calorimeter cells whithin this corn, or a maximum
%sum $p_T$ is allowed from tracks within this cone.  These thresholds are
%typically set as a fraction of the candidate muons $p_T$.
%
%The details of muon identification and isolation typically depend on
%the specific use case.  Thus, more detailed explainations will follow
%in conjunction with descriptions of analyses. 
%
%\subsection{Electron Reconstruction}
%\label{sec:electronRECO}
%
%Electron produce tracks in the silicon track and induce showers 
%in the ECal.  In beam test, typically, more than 90\% of the incident 
%energy of a single electron is contained in a $3\times3$ array of 
%crystals and more than 95\% is contained in a $5\times5$ array.  
%Within CMS, showing patterns are broadened by bremsstrahlung 
%induced by the presence material within the ECal and by the
%presence of a strong magnetic field.  In order to better identify 
%the energy deposits from electrons clustering algorithms are 
%used to combine energy measurements from individual crystals which
%take into account effects of bremsstrahlung and the magnetic field.  
%
%Seed crystals are identified which contain energy above some predefined
%threshold.  Nearby crystal are then combined to for a cluster.  In 
%an analogous procedure, cluster are combined into superclusters 
%starting from seed clusters.  Supercluster energies are then 
%corrected for systematic effects based on shower shape and the
%location of the supercluster.  The position of the reconstructed
%electron candidate is then measured from the energy-weighted mean
%position of each of the crystals.
%
%Superclusters are matched to pixel hits in the tracker which serve 
%as seed for electron tracks to be built.  Tracks are the built using
%a modified Kalman filter known as the Gaussian Sum Filter (GSF) which
%is a nonlinear filter which makes use of guassian mixtures to model
%errors.  The GSF algorithm allows for momentum at either end of the 
%tracker to be measured reliably and allows for the amount of brem to 
%be accurately estimated.  
%
%E-p combination...
%
%electron isolation ...
% 
%electron identification ...
%
%\subsection{Photon Reconstruction}
%\label{sec:photonRECO}
%
%Photon energy is clustered according to either the hybrid (EB) or the 
%island (EE) algorithm, similar to electrons.  For unconverted photons,
%most of the energy will be contained in a $3\times3$ array of crystals.
%In contrast, photons which get converted into $e^+e^-$ pairs can 
%cause the supercluster to spread over more crystal.  
%
%The R9 variable provides a powerful discriminator for converted photons.
%R9 is defined as the ratio of the energy in a $3\time3$ array, centered 
%on the highest energy crystal, to the total energy in a supercluster.
%Values of R9 close to 1 are indicitive of unconverted photon
%
%photon isolation...
%
%\subsection{Jet Reconstruction}
%\label{sec:jetRECO}

\subsection{Tracker}
\label{sec:Tracker}

The CMS tracker is an all silicon detector that consists of more than 16,000
individual silicon modules.  These modules are of two basic variaties, pixels
which provide a 2-dimensional measurement of particle positions and strips
which provide 1-dimensional measurements of particles positions within the plane of the module.  The tracker
is the closest sub-detector to the interaction point.  As such, it is exposed 
to the highest radiation flux and must be radiation hard to survive the extreme
conditions of the LHC.  As such, the design of the tracker barrel has been 
broken into three distinct regions in order to optimise occupancy against
signal-to-noise (S/N).  The most inner region is the pixel barrel (PXB), then the tracker
inner barrel (TIB), and finally the tracker outer barrel (TOB).  The latter
two regions constist of silicon microstrip detectors. 

\subsubsection{Pixel Modules}

The pixel modules are exposed to the highest particle flux, roughly $10^7$~Hz
at $r=10~cm$.  As a result, small pixels, $100\times150~\mu m^2$, are used,
giving an occupancy of about $10^{-4}$ per pixel per bunch crossing. Three
layers make up the pixel barrel at radii $r=4.4,~7.3,$ and $10.2~cm$ consisting of 768 pixel modules.  There are also two endcap disks on either side of 
the pixel barrel made of 672 pixel modules arranged in a turbine fashion. 
The layout of the pixel modules is shown in figure~\ref{fig:TrackerGeometry}.
In total, there are 66 million pixels which provide precise hit measurements.

\subsubsection{Strip Modules}

The strip modules are arranged into four regions, inner barrel (TIB), 
outer barrel (TOB), inner disks (TID), and end caps (TEC). 

The TIB is divided into 4 layers which extend out to $|z|<65~cm$, totalling
2724 strip modules.  The microstrip sensors on each module have a thickness
of 320~\microns~and a pitch of 80-120~\microns. The inner layers of the TIB
have stereo modules offset by an angle of 100 mrad, providing 2D
measurements.  The resolution of these
modules ranges from 23-34~\microns~in $r-\phi$ and 230~\microns~in 
the z-direction.

The TOB is divided into 6 layers extending out to $|z|<65~cm$, totalling
5208 strip modules.  Each microstrip sensor has a thickness of 500~\microns~
and a pitch ranging from 120-180~\microns.  Since the radii of the strip
layers is large, strips can be thicker in order to have better S/N while 
still have low occupancy.  Similar to the TIB, the first two layers of the
TOB have stereo modules offset by 100~mrad so that the single point resolution
in $r-\phi$ is 35-52~\microns~ while it is 530~\microns~ in the z-direction.

The TID is divided into 3 disks, the fist two of which are stereo, arranged
at various distances bewteen $120<|z|<280~cm$.  Modules
are arranged in wheels around the beam axis.  Each microstrip sensor has a 
thickness of 320~\microns.  Similarly, the TEC has 9 disks, the first two and 
the fifth of which are stereo.  The thickness of each microstrip sensor
is 500~\microns. 

\begin{figure}
\begin{center}
\hspace{-2.5cm}
\includegraphics[width=.8\linewidth]{ExperimentalSetupPlots/las.eps}
\caption{Quarter slice of the CMS tracker.  Single-sided silicon strip modules
are indicated as solid light (purple) lines, double-sided strip modules as 
open (blue) lines, and pixel modules as solid dark (blue) lines.}
\label{fig:TrackerGeometry}
\end{center}
\end{figure}

\subsubsection{Tracking Performance \& Alignment}

The tracker provides high precision measurements of track parameters for
all charged particles, this includes both the momentum and direction
of tracks.  These track parameters can be used to 
better understand resonance properties, as will be shown in 
chapters~\ref{sec:HiggsPhen} and~\ref{sec:HZZsearches}.  Thus, the 
tracker will be one of the most important tools in searching for new 
resonances, such as the Higgs boson, and understanding their role in
nature.  

The tracker is also the only detector which can reconstruct vertices, either
displaced or not.  Vertexing provides critical information to help mitigate
the effects of pile-up as well as tagging b-jets.  Since pile-up will be
a continuing challenge at the LHC, continued performance of the tracker
will be critical.  The use of the tracker in b-tagging will also play 
a central role in physics measurements, since b-jets provide a distinct 
signature which is relevant to many beyond the SM physics as well of Higgs
physics.  

In order to ensure the high quality performance, uncertainties of module 
positions, taken to refer to both the location and orientation which are
depicted in figure~\ref{fig:ModuleCoordinates},
should be reduced to within the precision of each module.  For the
pixel modules, this precision is around 10~\microns, for the strips, this 
precision can be as large as 30~\microns.  Because of changing enviromental
conditions of the detector, the tracker geometry can be time dependent. 
In order to efficiently determine module positions through run periods, 
offline track-based alignment algorithms must be employed.  

\begin{figure}
\begin{center}
\includegraphics[width=.49\linewidth]{ExperimentalSetupPlots/coordinates.eps}
\caption{Diagram of module positions variables, u, v, w, and module orientation
variables, $\alpha$, $\beta$, $\gamma$.}
\label{fig:ModuleCoordinates}
\end{center}
\end{figure}

Track-based alignments are intended to determine the position of each module
in the tracker from a large collection of reconstructed tracks.  Each track
is built from a set of charge deposition sites, or hits, on a given module
which are used to produce a piece-wise helical trajectory using the
Combinatorial Track Finder (CTF) algorithm~\cite{Borrello:2010zz}.  
Alignment of each module position can be performed by minimizing
\begin{equation}
\chi^2(\vec{p},\vec{q}) = \sum_{j}^{tracks}\sum_{i}^{hits}\vec{r}_{ij}^{T}(\vec{p},\vec{q}_j)V^{-1}_{ij}\vec{r}_{ij}(\vec{p},\vec{q}_j),
\end{equation}
where $\vec{p}$ is the position correction, $\vec{q}_j$ is the set of 
track parameters for the j tracks, $\vec{r}_{ij}$ are the track 
residuals, and $\vec{V}_{ij}$ is the covariance matrix.  The residuals are
defined as $\vec{r}_{ij} = \vec{m}_{ij} - \vec{f}_{ij}(\vec{p},\vec{q}_j)$, 
where $\vec{m}_{ij}$ are the measured hit positions and $\vec{f}_{ij}$ are
the track trajectory impact point in the plane of the module.  The 
$\chi^2$ function is then minimized with respect to the module position 
corrections, $\vec{p}$.

Since there are more than 16,000 modules with 6 parameters to be determined,
tracker alignment is a 
extremely difficult to solve exactly, which would require inverting
$16,000\cdot6\times16,000\cdot6$ matrix.  As a result, approximations 
must be employed.   One such approximation is to minimize the $\chi^2$ for
each module individually, ignoring the correlation between the change in 
parameters between different modules.  The correlation is then recovered 
by recalculating $f_{ij}$ and iterating the procedure many times.  Solving 
for each individual module position is then reduce to a six-dimensional 
matrix equation, 
\begin{equation}
\chi^2_m(\vec{p})=\sum_i^{hits}\vec{r}^T_i\vec{V}_i^{-1}\vec{r}_i(\vec{p}).
\end{equation}
This local iterative algorithm, described in detail 
elsewhere~\cite{Karimaki:2006az,Brown:2008ccb}, was employed to produce 
the first geometry using minimum bias collision tracks.  

Validations of tracker geometries are critical to understanding that 
the output of alignment algorithms produce improved measurements.  
Several validation are important for showing improvements in the tracker 
geometry, the primary vertex validation and the cosmic splitting validation.
Both of these validation provide a direction connection between the 
tracker geometry and measurements relevant for physics analyses.   

The cosmic splitting validation makes use of cosmic tracks recorded 
during interfills.  Cosmic tracks have the unique feature that the 
tracks can pass through all layers of the tracker.  As a result,
a cosmic track is qualitatively similar to two collision track
produced back to back.  This feature can be taken advantage of by dividing
each cosmic track into subset of hits and reconstructing these hits 
into {\it split} tracks which are reconstructed independently.  The 
track parameters of the 
split tracks should, by construction, have the same track parameters.  
Thus, by comparing the track parameters, track parameter resolution
and biases can be test.  

%% describe the slew of track parameters that can be studied.
The resolution of individual track parameters can be quantified
and compared between different tracker geometries.  This is
represented by the distribution of the difference of a given 
track parameters between the two split tracks.  This difference
can also be compared in slices of other track parameters in 
order to quantify systematic misalignments.

To demonstrate the precision on each of the track quantities, the
difference of 5 track parameters, $\Delta d_{xy}$, $\Delta d_z$, 
$\Delta \eta$, $\Delta\phi$, and $\Delta p_T$ are shown in 
figure~\ref{fig:trackSplittingMC}.  
Figure~\ref{fig:trackSplitting2012A} shows
the same set of validation plots for cosmics collected during
2012 run A.  Three geometries are compared, the ideal geometry, the 
prompt geometry (before alignment) and the Re-RECO geometry (after 
alignment).  Improvements are found over the prompt geometry and
in some cases, the aligned geometry is found to be consistent 
with the ideal geometry tested on MC simulations.  

\begin{figure}
\begin{center}
\includegraphics[width=.32\linewidth]{ExperimentalSetupPlots/histDelta_dxy.eps}
\includegraphics[width=.32\linewidth]{ExperimentalSetupPlots/histDelta_dz.eps}
\includegraphics[width=.32\linewidth]{ExperimentalSetupPlots/histDelta_eta.eps}
\includegraphics[width=.32\linewidth]{ExperimentalSetupPlots/histDelta_phi.eps}
\includegraphics[width=.32\linewidth]{ExperimentalSetupPlots/histDelta_pt.eps}
\caption{Resolution of 5 track parameters from track splitting validation 
using MC and ideal geomtry. }
\label{fig:trackSplittingMC}
\end{center}
\end{figure}

\begin{figure}
\begin{center}
\includegraphics[width=.32\linewidth]{ExperimentalSetupPlots/2012A/histDelta_dxy.eps}
\includegraphics[width=.32\linewidth]{ExperimentalSetupPlots/2012A/histDelta_dz.eps}
\includegraphics[width=.32\linewidth]{ExperimentalSetupPlots/2012A/histDelta_eta.eps}
\includegraphics[width=.32\linewidth]{ExperimentalSetupPlots/2012A/histDelta_phi.eps}
\includegraphics[width=.32\linewidth]{ExperimentalSetupPlots/2012A/histDelta_pt.eps}
\caption{Resolution of 5 track parameters from track splitting validation 
using three geometries, ideal (blue), prompt geometry (black), and the
aligned geometry (red).  Cosmic track recording during the 2012 run A
period were used.}
\label{fig:trackSplitting2012A}
\end{center}
\end{figure}

From figures~\ref{fig:trackSplittingMC} and~\ref{fig:trackSplitting2012A} we can see 
that the average errors of the impact parameters are 25\microns~(42\microns) for the 
transverse (longitudinal) directions with respect to the beam line.  The angular
variables are found to have extremely good precision, on the level of the $3.2\times10^{-4}$~radians 
for the azimuthal angle, $\phi$, and ($4.6\times10^{-4}$) for pseudorapidity, $\eta$.

The transverse momentum, $p_T$, has a relative precision of 1\%.  Since the $p_T$
distribution of cosmic tracks is dominated by low $p_T$ tracks, in order to 
understand the resolution on $p_T$ for the high momentum tracks which are relevant
to physics searches, the width of the $\Delta p_T$ distribution is plotted in 
bins of $p_T$, shown in the right plot of figure~\ref{fig:trackSplittingProfiles}.  
The relative resolution on $p_T$ varies from .1~GeV to .45~GeV for tracks with
$p_T$ between 10 and 100 GeV.

The track parameter errors can also be check in bins of other
variables.  The left plot of figure~\ref{fig:trackSplittingProfiles} shows 
$d_{xy}$ in bins of $\phi$ for the 2012 run A cosmic data.  
There is a significant improvement Between the prompt and re-RECO geometry.  
The structure seen cannot be noticed in the simple
resolution plots and is indicitive of a systematic deformation present in
the geometry.  

\begin{figure}
\begin{center}
\includegraphics[width=.49\linewidth]{ExperimentalSetupPlots/2012A/profiledxy_orgDelta_phi.eps}
\includegraphics[width=.49\linewidth]{ExperimentalSetupPlots/2012A/resolutionpt_orgDelta_pt.eps}
\caption{Profile plots of several reference geometries using cosmic track
taken during the 2012 run A period.  The left plot shows the difference 
in $d_{xy}$ between the two split tracks, $\Delta d_{xy}$ vs $\phi$.  The 
right plot shows the width of the $\Delta p_T$ distribution, $\sigma(p_T)$, 
vs $p_T$.}
\label{fig:trackSplittingProfiles}
\end{center}
\end{figure}

Profile plots, like those seen in figure~\ref{fig:trackSplittingProfiles}, 
are good probes of weak modes which could be present in
tracker geometry.  The modes correspond to systematic deformations which 
are $\chi^2$ invariant.  Some examples include a systematic shift of modules
in the r-$\phi$ direction which is a function of $\phi$ itself.  This 
type of deformation would result in the structure that is seen in the 
left plot of figure~\ref{fig:trackSplittingProfiles} in the prompt geometry.
In this case, the deformation is not a weak mode since the alignment 
procedure is sensitive to it and corrects the module positions accordingly.  However, understanding similar defformations is important for assessing uncertainties in physics measurements.  

%% discuss systematic misalignments and their effects on validations

%% discuss primary vertex validations

The primary vertex validation uses the position of primary vertices as
an estimator of the true impact parameters of an individual track.  
Residuals can be constructed from the difference between the primary 
vertex and a tracks fitted impact parameter as demonstrated in 
figure~\ref{fig:PVcartoon}.    If tracks truly originate from 
the vertex, then on average the above assumption
will be true.  However, individual tracks which pass through poorly 
aligned regions of the tracker will give larger residuals.  Thus providing
a self consistent probe of the tracker geometry.  

\begin{figure}
\begin{center}
\includegraphics[width=.69\linewidth]{ExperimentalSetupPlots/PVcartoon.eps}
\caption{}
\label{fig:PVcartoon}
\end{center}
\end{figure}

Distributions of impact parameter residuals are sensitive to changes in the
pixel modules.
Figure~\ref{fig:???}
shows a number of residual distributions in various bins of $\eta$ and
$\phi$.  Each bin represents tracks from a specific region of pixel 
module.  The mean and RMS of these distributions, which are measured
using double gaussian fits, can provide useful
information about systematic misalignments of the pixel barrel.  In
particular, this validation is sensitive to observed separation of the
pixel half barrels, depicted in figure~\ref{??}, which tend to move from 
time to time. 

To quantify the separation of the pixel half barrels, the mean and width
of the residual distributions are plotted as a function of phi.  If a 
separation between the two half barrels is present, it will cause a discontinuity
at zero.  Figure~\ref{fig:dzResidVsPhi} shows an example plot of this
using MC tracks with either the ideal geometry or a geomtry in which the 
two half barrels have been purposefully shifted.  The size of the
discontinuity is directly corresponds to the size of the physical separation.  

\begin{figure}
\begin{center}
\includegraphics[width=.69\linewidth]{ExperimentalSetupPlots/PVValidation_dzResidVsPhi_paper.eps}
\caption{Distribution of mean and width of transverse impact parameter residuals 
in bins of the probe tracks azimuthal angle, $\phi$, for an ideal geometry (black), 
ideal geometry plus 40~\microns~separation between the pixel half barrels (red),
and the 2011 candidate geometry (blue).}
\label{fig:dzResidVsPhi}
\end{center}
\end{figure}

The presence of a shift can have significant impact on vertex measurements,
which can effect either efficiency of associating tracks with the primary
vertex or efficiency of b-tagging. Thus monitoring and correcting
these deformations in time is critical.  Figure~\ref{fig:DzVStime} shows the 
measured separation of the pixel half barrles versus time before (black)
and after (red) alignment parameters were determined.  This procedure
was critical for determining an effective alignment procedure by defining
run ranges to perform independent alignments of large structures in order to 
correct the time dependence seen.  The red points in figure~\ref{fig:DzVStime}
show that most of the time dependence is reduced to below 5-10~\microns.

\begin{figure}
\begin{center}
\includegraphics[width=.49\linewidth]{ExperimentalSetupPlots/plotForAlignmentPaper.eps}
\caption{Measured separation between pixel half barrels versus time
before and after alignment.}
\label{fig:DzVStime}
\end{center}
\end{figure}

\section{summary}

The necessary but challenging environment provided by the LHC has produced higher energies than have ever previously been attained.  This is critical for 
producing heavy resonances as well as increasing the phase space
for producing intermediate mass resonances such as the Higgs boson.  
The design of CMS has allowed for high quality data collecting even in the midst 
of the high rates and high pileup environment produced by the LHC.  Offline validation,
calibration, and alignment of the various subdetectors is a critical 
aspect of the success of CMS.  

The continued monitoring and adjustment of the tracker geometry using offline
track-based alignment algorithms is critical to producing high precision
track measurements.  These measurements will be critical to physics measurements,
especially Higgs searches.  Since angular distributions of the final state 
particles of resonances can be exploited for property measurements, to be
discussed in chapters~\ref{sec:HiggsPhen} and~\ref{sec:HZZsearches}, it is 
important to have tools, like those mentioned above to monitor tracker 
performance using either collision tracks or cosmic tracks.
