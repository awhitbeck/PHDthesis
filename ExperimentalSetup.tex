\newcommand{\microns}{$\mu m$}


\chapter{Experimental Setup}
\label{sec:ExpSetup}
\chaptermark{Experimental Setup}


\section{The Large Hadron Collider}
\label{sec:LHC}
The Large Hadron Collider was designed to accelerates protons to up to 
energies of 7 TeV using a 26 km storage ring and  1232  8.33 T dipole magnets. 
Although it is also capable of accelerating heavier nuclei up to energies of 
2.76 TeV, heavy ion physics is outside the scope of this work.  The proton
energies accessible to the LHC are a factor of seven times higher than its
most advanced predicessor, the Tevatron.  These energies are not only 
important for accessing new particles which might exist at large invariant
mass, on the order of serveral TeV, they are also necessary for efficient 
production of moderately heavy particles, like the Higgs boson or the top 
quark.  In the more favorable region of the SM 
Higgs mass,  $m_{H}$, parameter space, these energies provides a 
factor of XX in total cross section over the production cross section at
the Tevatron.  

The LHC also has the capability to collide bunches of $1\times10^{11}$
protons every 25 ns at $\beta*=.55$ and $\sigma*=16.7$.  These
parameters and others, summarized in table~\ref{table:LHCparameters},
combine to allow the LHC to produce instantaneous luminosities 
of up to $10^{34} cm^{-2}s^{-1}$ accoding to
\begin{equation}
\mathscr{L} = \frac{\gamma f k_B N_p^2}{4\pi \epsilon_n\beta^*}F,
\end{equation}
where $\gamma$ is the Lorentz factor, f is the revolution frequency,
$k_B$ is the number of protons per bunch, $\epsilon_n$ is the betatron
function at the interaction point (IP), and F is the reduction factor
due to the crossing angle.  This translates to roughly 1 billion proton
proton interactions per second and up to 50 collisions per bunch crossing, 
known as pileup.  

\begin{table}
\begin{center}
\begin{tabular}{l|c|c}
\hline 
\hline
Energy per nucleon           & $E$           & 7 TeV    \\
Dipole field at 7 TeV        & $B$           & 8.33 T   \\
Design Luminosity            & $\mathscr{L}$ & $10^{34}~cm^{-2}s^{-1}$\\
Bunch separation             &               & 25 ns    \\
No. of bunches               & $k_B$         & 2808      \\
No. of particles/bunch       & $N_p$         & $1.15\times 10^{11}$\\ \hline
\multicolumn{3}{l}{{\bf Collisions}} \\ \hline
$\beta$-value at IP          & $\beta^*$     & 0.55 m   \\
RMS beam radius at IP        & $\sigma^*$    & $16.7~\mu m$  \\
Luminosity lifetime          & $\tau_L$      & 15 hr         \\
Number of collisions/crossing& $n_c$         & $\equiv20$      \\
\hline 
\hline
\end{tabular}
\label{table:LHCparameters}
\caption{Table listing relevant operational parameters of the LHC.}
\end{center}
\end{table}

These conditions provide unprecidented conditions to probe the SM and 
discovering new particles but also extreme conditions for reconstructing
objects with a high degree of efficiency.  
The inclusive proton-proton
collision at 14~TeV is approximately 100~mb.  At design luminosity, this
corresponds to an event rate of $10^9/s$.  
The large number of proton-proton
collisions produce a considerable amount of background noise which can 
either produce extra objects from secondary interactions as well an overall
increase in the energy deposited in the calorimeters.  The  high rate of the 
LHC far exceed the capabilities of the Data Aquisition (DAQ) system.  As a 
result it is necessary to use fast hardware logic to filter the majority of 
events. 

The short time between bunch crossings also puts significant constraints on 
detectors since subdetectors should have fast response times and low occupancy. 
High grainularity tracking will be necessary for high precision vertexing in
order to mitigate the effects of pileup.

\section{The Compact Muon Solenoid}
\label{sec:CMS}

The Compact Muon Solenoid (CMS), is a general purpose particle detector.  It
was designed to not only have a broad scope of discovery potential but also
to mitigate the extreme conditions created by the LHC.  CMS is made up of
several different types of apparati designed to improve identification of 
particles and measure their properties.  There is a two-stage trigger system
to filter the extreme rates comming from the LHC.  There is an all silicon 
tracking system at the center to carefully record the positions of charged 
particles passing through the detector.  There is a 4 Tesla magnet to bend 
charge particles 
providing the tracker and muon system sensitivity to the momentum of charged 
particles.  There are two calorimeters designed to induce particle showers 
which can then be used to measure energy deposits.  Finally, there is a Muon 
system 
at the edge of the detector to detect semi-stable, charged particles with long 
interaction lengths, e.g. the muon. This chapter describes these detectors in
detail. 

\subsection{Tracker}
\label{sec:Tracker}

The CMS tracker is an all silicon detector that consists of more than 16,000
individual silicon modules.  These modules are of two basic variaties, pixels
which provide a 2-dimensional measurement of particle positions and strips
which provide 1-dimensional measurements of particles positions.  The tracker
is the closest sub-detector to the interaction point.  As such, it is exposed 
to the highest radiation flux and must be radiation hard to survive the extreme
conditions of the LHC.  As such, the design of the tracker barrel has been 
broken into three distinct regions in order to optimise occupancy against
S/N.  The most central region is the pixel barrel (PXB), then the tracker
inner barrel (TIB), and finally the tracker outer barrel (TOB).  The latter
two regions constist of silicon microstrip detectors. 

\subsubsection{Pixel Modules}

The pixel modules are exposed to the highest particle flux, roughly $10^7/s$
at $r=10~cm$.  As a result, small pixels, $100\times150~\mu m^2$, are used,
giving an occupancy of about $10^{-4}$ per pixel per bunch crossing. Three
layers make up the pixel barrel at radii $r=4.4,~7.3,$ and $10.2~cm$ made
up of 768 pixel modules.  There are also two endcap disks on either side of 
the pixel barrel made of 672 pixel modules arranged in a turbine fashion. 
The layout of the pixel modules is shown in figure~\ref{fig:???}.
In total, there are 66 million pixels which provide precise hit measurements.

\subsubsection{Strip Modules}

The strip modules are arranged into four regions, inner barrel (TIB), 
outer barrel (TOB), inner disks (TID), and end caps (TEC). 

The TIB is divided into 4 layers which extend out to $|z|<65~cm$, totalling
2724 strip modules.  The microstrip sensors on each module have a thickness
of 320~\microns~and a pitch of 80-120~\microns. The inner layers of the TIB
have stereo modules offset by an angle of 100 mrad.  The resolution of these
modules ranges from 23-34~\microns~in $r-\phi$ and 230~\microns~in 
the z-direction.

The TOB is divided into 6 layers extending out to $|z|<65~cm$, totalling
5208 strip modules.  Each microstrip sensor has a thickness of 500~\microns~
and a pitch ranging from 120-180~\microns.  Since the radii of the strip
layers is large, strips can be thicker in order to have better S/N while 
still have low occupancy.  Similar to the TIB, the first two layers of the
TOB have stereo modules offset by 100~mrad so that the single point resolution
in $r-\phi$ is 35-52~\microns~ while it is 530~\microns~ in the z-direction.

The TID is divided into 3 disks, the fist two of which are stereo, arranged
at various distances bewteen $120<|z|<280~cm$.  Modules
are arranged in wheels around the beam axis.  Each microstrip sensor has a 
thickness of 320~\microns.  Similarly, the TEC has 9 disks, the first two and 
the fifth of which are stereo.  The thickness of each microstrip sensor
is 500~\microns. 

\subsection{Magnet}
\label{sec:Magnet}

CMS employs a 4~T superconducting aluminum solenoid magnet to bend tracks for 
both charge identification and momentum resolution.  The field was
chosen to have good momentum resolution, $\Delta p/p\equiv10\%$ at 
$p=1~TeV/c$.  The magnet has an inner bore of 5.9~m, large enough to house
the tracker and both calorimeters, and a length of 12.9~m.  Drawing a 
current of 19.5~kA, the magnets total stored energy is 2.7~GJ.  Making it
one of the largest magnets in the world.  
The outer return yolk of the magnet concetrates the magnetic field in the
region near the muon system, which is placed outside of the solenoid.  

\subsection{ElectroMagnetic Calorimeter}
\label{sec:ECal}

The Electromagnetic calorimeter (ECal) is a high grainularity calorimeter
intended to induce electromagnetic showers which are collected by either
avalanche photodiodes (barrel) or vacuum phototriodes (endcap).  The 
material used is scintillating lead tungstate crystal
which was chosen its: short radiation ($X_0$=0.89~cm) and Moliere (2.2~cm) 
lengths; the time scale in which showers occur, 80\% of light is emmitted
in 25~ns; and the radiation hardness.  The ECal is divided into barrel (EB)
and endcap (EE) regions.

The EB region has an inner radius of 129~cm and is constructed from 36
identical {\it supermodules}, each covering half of the barrel in the 
z-direction (1.479 unit of pseudorapidity).  Each individual crystal 
1 degree in both $\Delta\phi$ and $\Delta\eta$, corresponding to a cross
sectional area of $22\times22~mm^2$, and are 230~mm, corresponding to 
25.8~$X_0$.

The EE region is located at a distrance of 314~cm along the z-direction
and covering the pseudorapidity range $1.479<|\eta|<3.0$.  The crystals
clustered into $5\times5$ {\it supercrystals} which are combined to form
semi-circle structures.  Each crystal has a cross sectional area of 
$28.6\times28.6~mm^2$ and are 220 mm (24.7~$X_0$) in length.  The endcap 
region is also preceded by a preshower which consists of a lead absorber
whos thickness is 2-3~$X_0$ followed by 2 planes of silicon strip detectors.

The energy response of the ECal was measured in test beams.  The energy 
resolution was parameterized according to
\begin{equation}
\left(\frac{\sigma}{E}\right)^2 = \left(\frac{S}{\sqrt{E}}\right)^2 + \left(\frac{N}{E}\right)^2 + C^2,
\end{equation}
where S, N and C represent the stochastic, noise, and constant contributions.
The coefficients measurements for two energy clustering schemes are shown
if figure~\ref{fig:???}.

\subsection{Hadronic Calorimeter}
\label{sec:HCal}

The HCal ...

\subsection{Muon System}
\label{sec:Muon System}

The Muon System ...

\subsection{Electron Reconstruction and Identification}
\label{sec:HZZ4lelectrons}

\subsection{Muon Reconstruction and Identification}
\label{sec:HZZ4lmuons}

\subsection{Photon Reconstruction and Identification}
\label{sec:HZZ4lphotons}

\subsection{Jet Reconstruction and Identification}
\label{sec:HZZ4ljets}
