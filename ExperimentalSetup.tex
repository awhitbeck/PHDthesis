\newcommand{\microns}{$\mu m$}


\chapter{Experimental Setup}
\label{sec:ExpSetup}
\chaptermark{Experimental Setup}


\section{The Large Hadron Collider}
\label{sec:LHC}
The Large Hadron Collider was designed to accelerates protons to up to 
energies of 7 TeV using a 26 km storage ring and  1232  8.33 T dipole magnets. 
Although it is also capable of accelerating heavier nuclei up to energies of 
2.76 TeV, heavy ion physics is outside the scope of this work.  The proton
energies accessible to the LHC are a factor of seven times higher than its
most advanced predicessor, the Tevatron.  These energies are not only 
important for accessing new particles which might exist at large invariant
mass, on the order of serveral TeV, they are also necessary for efficient 
production of moderately heavy particles, like the Higgs boson or the top 
quark.  In the more favorable region of the SM 
Higgs mass,  $m_{H}$, parameter space, these energies provides a 
factor of XX in total cross section over the production cross section at
the Tevatron.  

The LHC also has the capability to collide bunches of $1\times10^{11}$
protons every 25 ns at $\beta*=.55$ and $\sigma*=16.7$.  These
parameters and others, summarized in table~\ref{table:LHCparameters},
combine to allow the LHC to produce instantaneous luminosities 
of up to $10^{34} cm^{-2}s^{-1}$ accoding to
\begin{equation}
\mathscr{L} = \frac{\gamma f k_B N_p^2}{4\pi \epsilon_n\beta^*}F,
\end{equation}
where $\gamma$ is the Lorentz factor, f is the revolution frequency,
$k_B$ is the number of protons per bunch, $\epsilon_n$ is the betatron
function at the interaction point (IP), and F is the reduction factor
due to the crossing angle.  This translates to roughly 1 billion proton
proton interactions per second and up to 50 collisions per bunch crossing, 
known as pileup.  

\begin{table}
\begin{center}
\begin{tabular}{l|c|c}
\hline 
\hline
Energy per nucleon           & $E$           & 7 TeV    \\
Dipole field at 7 TeV        & $B$           & 8.33 T   \\
Design Luminosity            & $\mathscr{L}$ & $10^{34}~cm^{-2}s^{-1}$\\
Bunch separation             &               & 25 ns    \\
No. of bunches               & $k_B$         & 2808      \\
No. of particles/bunch       & $N_p$         & $1.15\times 10^{11}$\\ \hline
\multicolumn{3}{l}{{\bf Collisions}} \\ \hline
$\beta$-value at IP          & $\beta^*$     & 0.55 m   \\
RMS beam radius at IP        & $\sigma^*$    & $16.7~\mu m$  \\
Luminosity lifetime          & $\tau_L$      & 15 hr         \\
Number of collisions/crossing& $n_c$         & $\equiv20$      \\
\hline 
\hline
\end{tabular}
\label{table:LHCparameters}
\caption{Table listing relevant operational parameters of the LHC.}
\end{center}
\end{table}

These conditions provide unprecidented conditions to probe the SM and 
discovering new particles but also extreme conditions for reconstructing
objects with a high degree of efficiency.  
The inclusive proton-proton
collision at 14~TeV is approximately 100~mb.  At design luminosity, this
corresponds to an event rate of $10^9/s$.  
The large number of proton-proton
collisions produce a considerable amount of background noise which can 
either produce extra objects from secondary interactions as well an overall
increase in the energy deposited in the calorimeters.  The  high rate of the 
LHC far exceed the capabilities of the Data Aquisition (DAQ) system.  As a 
result it is necessary to use fast hardware logic to filter the majority of 
events. 

The short time between bunch crossings also puts significant constraints on 
detectors since subdetectors should have fast response times and low occupancy. 
High grainularity tracking will be necessary for high precision vertexing in
order to mitigate the effects of pileup.

\section{The Compact Muon Solenoid}
\label{sec:CMS}

The Compact Muon Solenoid (CMS), is a general purpose particle detector.  It
was designed to not only have a broad scope of discovery potential but also
to mitigate the extreme conditions created by the LHC.  CMS is made up of
several different types of apparati designed to improve identification of 
particles and measure their properties.  There is a two-stage trigger system
to filter the extreme rates comming from the LHC.  There is an all silicon 
tracking system at the center to carefully record the positions of charged 
particles passing through the detector.  There is a 4 Tesla magnet to bend 
charge particles 
providing the tracker and muon system sensitivity to the momentum of charged 
particles.  There are two calorimeters designed to induce particle showers 
which can then be used to measure energy deposits.  Finally, there is a Muon 
system 
at the edge of the detector to detect semi-stable, charged particles with long 
interaction lengths, e.g. the muon. This chapter describes these detectors in
detail. 

\subsection{Tracker}
\label{sec:Tracker}

The CMS tracker is an all silicon detector that consists of more than 16,000
individual silicon modules.  These modules are of two basic variaties, pixels
which provide a 2-dimensional measurement of particle positions and strips
which provide 1-dimensional measurements of particles positions.  The tracker
is the closest sub-detector to the interaction point.  As such, it is exposed 
to the highest radiation flux and must be radiation hard to survive the extreme
conditions of the LHC.  As such, the design of the tracker barrel has been 
broken into three distinct regions in order to optimise occupancy against
S/N.  The most central region is the pixel barrel (PXB), then the tracker
inner barrel (TIB), and finally the tracker outer barrel (TOB).  The latter
two regions constist of silicon microstrip detectors. 

\subsubsection{Pixel Modules}

The pixel modules are exposed to the highest particle flux, roughly $10^7/s$
at $r=10~cm$.  As a result, small pixels, $100\times150~\mu m^2$, are used,
giving an occupancy of about $10^{-4}$ per pixel per bunch crossing. Three
layers make up the pixel barrel at radii $r=4.4,~7.3,$ and $10.2~cm$ made
up of 768 pixel modules.  There are also two endcap disks on either side of 
the pixel barrel made of 672 pixel modules arranged in a turbine fashion. 
The layout of the pixel modules is shown in figure~\ref{fig:???}.
In total, there are 66 million pixels which provide precise hit measurements.

\subsubsection{Strip Modules}

The strip modules are arranged into four regions, inner barrel (TIB), 
outer barrel (TOB), inner disks (TID), and end caps (TEC). 

The TIB is divided into 4 layers which extend out to $|z|<65~cm$, totalling
2724 strip modules.  The microstrip sensors on each module have a thickness
of 320~\microns~and a pitch of 80-120~\microns. The inner layers of the TIB
have stereo modules offset by an angle of 100 mrad.  The resolution of these
modules ranges from 23-34~\microns~in $r-\phi$ and 230~\microns~in 
the z-direction.

The TOB is divided into 6 layers extending out to $|z|<65~cm$, totalling
5208 strip modules.  Each microstrip sensor has a thickness of 500~\microns~
and a pitch ranging from 120-180~\microns.  Since the radii of the strip
layers is large, strips can be thicker in order to have better S/N while 
still have low occupancy.  Similar to the TIB, the first two layers of the
TOB have stereo modules offset by 100~mrad so that the single point resolution
in $r-\phi$ is 35-52~\microns~ while it is 530~\microns~ in the z-direction.

The TID is divided into 3 disks, the fist two of which are stereo, arranged
at various distances bewteen $120<|z|<280~cm$.  Modules
are arranged in wheels around the beam axis.  Each microstrip sensor has a 
thickness of 320~\microns.  Similarly, the TEC has 9 disks, the first two and 
the fifth of which are stereo.  The thickness of each microstrip sensor
is 500~\microns. 

\subsubsection{Tracking Performance \& Alignment}

The tracker provides high precision measurements of track parameters for
all charged particles, this includes both the momentum and direction
of tracks.  It is the only detector which can reconstruct vertices, both
displaced or not.  A key piece to have high quality measurements is the 
tracker alignment.  

Alignment deals with understanding where with what 
orientation pixel and strip modules should be.  In order to maximize the 
performance of the tracker, the position of each module should be known to
better than the intrinsic resolution of the pixel (10~\microns) or strip 
(50~\microns) modules.  Although information of
module positions was initially provided by optical surveys, knowing the 
position of each module to within 10~\microns can only be acheive by done
track-based alignments.  

Track-based alignments are done looking at the sum of track-hit residuals
and minimizing a $\xi^2$ with respect to the module positions and 
orientations.  Since there are over 
16,000 modules with 6 degrees of freedom each, tracker alignment is a 
major challenge to solve exactly, which would require inverting
$16,000\cdot6\times16,000\cdot6$ matrix.  As a results, approximations 
must be employed.   One such approximation is to mimize the $\xi^2$ for
each module individually, ignoring the correlation between the change in 
parameters between different modules.  The correlation is then recovered 
by iterating the procedure many times.  This procedure is known as the 
HIP algorithm and was employed to produce the first geometry using minimum
bias collision tracks.  

{\it ... talk more about the 2010 geometry ...}

Validations of tracker geometries are critical to understanding that 
alignment algorithms produced improved geometries.  Several validation
are important for showing improvements in the tracker geometry, the 
primary vertex validation and the cosmic splitting validation. 

The cosmic splitting validation makes use of cosmic tracks tracking 
during interfills.  Cosmic tracks have the unique feature that the 
tracks can pass through all layers of the tracker.  As a result,
a cosmic track is qualitatively similar to two collision track
produced back to back.  This feature can be taken advantage of by dividing
each cosmic track into subset of hits and reconstructing these hits 
into {\it split} tracks independently.  The track parameters of the 
split tracks should, by construction have the same track parameters.  
Thus, by comparing the track parameters, track parameter resolution
and biases can be test.  

%% describe the slew of track parameters that can be studied.

To quantify the effect of misalignments on the cosmic splitting 
validation, 5 track parameters are compare, $d_{xy}$, $d_z$,
$\eta$, $\phi$, and $p_T$. Figure~\ref{fig:trackSplittingMC}
shows the difference between the two plit tracks for each of the
7 track parameters, thus providing a measurement of the track
parameters resolution.  Figure~\ref{fig:trackSplitting2012A} shows
the same set of validation plots for cosmics collected during
2012 run A.  Two geometry are compared, the ideal geometry, the 
prompt geometry (before alignment) and the ReRECO geometry (after 
alignment).  Improvements are found over the prompt geometry and
in some cases, the aligned geometry is found to be consistent 
with the ideal geometry.  

\begin{figure}
\begin{center}
\includegraphics[width=.32\linewidth]{ExperimentalSetupPlots/histDelta_dxy.pdf}
\includegraphics[width=.32\linewidth]{ExperimentalSetupPlots/histDelta_dz.pdf}
\includegraphics[width=.32\linewidth]{ExperimentalSetupPlots/histDelta_eta.pdf}
\includegraphics[width=.32\linewidth]{ExperimentalSetupPlots/histDelta_phi.pdf}
\includegraphics[width=.32\linewidth]{ExperimentalSetupPlots/histDelta_pt.pdf}
\label{fig:trackSplittingMC}
\caption{Resolution of 5 track parameters from track splitting validation 
using MC and ideal geomtry. }
\end{center}
\end{figure}

\begin{figure}
\begin{center}
\includegraphics[width=.32\linewidth]{ExperimentalSetupPlots/2012A/histDelta_dxy.pdf}
\includegraphics[width=.32\linewidth]{ExperimentalSetupPlots/2012A/histDelta_dz.pdf}
\includegraphics[width=.32\linewidth]{ExperimentalSetupPlots/2012A/histDelta_eta.pdf}
\includegraphics[width=.32\linewidth]{ExperimentalSetupPlots/2012A/histDelta_phi.pdf}
\includegraphics[width=.32\linewidth]{ExperimentalSetupPlots/2012A/histDelta_pt.pdf}
\label{fig:trackSplitting2012A}
\caption{Resolution of 5 track parameters from track splitting validation 
using three geometries, ideal (blue), prompt geometry (black), and the
aligned geometry (red).  Cosmic track recording during the 2012 run A
period were used.}
\end{center}
\end{figure}

The difference of track parameters can also be check in bins of other
variables.  One of the most useful application of this is to measure
the $p_T$ dependence of the normalized resolution on $p_T$ itself, 
$\sigma(\Delta p_T)$. Figure~\ref{fig:trackSplittingProfiles} shows 
several example plots of these profile plots for the 2012 run A cosmic 
data.  While there are only minor improvments to the $\sigma(p_T)$ vs
$p_T$ distribution, there is a significant improvement to in the $\Delta\phi$
vs $d_{xy}$ plot.  The structure seen cannot be noticed in the simple
resolution plots and is indicitive of a system deformation present in
the geometry.  

\begin{figure}
\begin{center}
\includegraphics[width=.49\linewidth]{ExperimentalSetupPlots/2012A/profiledxy_orgDelta_phi.pdf}
\includegraphics[width=.49\linewidth]{ExperimentalSetupPlots/2012A/resolutionpt_orgDelta_pt.pdf}
\label{fig:trackSplittingProfiles}
\caption{Profile plots of several reference geometries using cosmic track
taken during the 2012 run A period.  The left plot shows the difference 
in $d_{xy}$ between the two split tracks, $\Delta d_{xy}$ vs $\phi$.  The 
right plot shows the width of the $\Delta p_T$ distribution, $\sigma(p_T)$, 
vs $p_T$.}
\end{center}
\end{figure}

Profiles plots, like those seen in figure~\ref{fig:trackSplittingProfiles}, 
are good probes of weak modes which could be present in
tracker geometry.  The modes correspond to systematic deformations which 
are $\xi^2$ invariant.  Some examples include a systematic shift of modules
in the r-$\phi$ direction which is a function of $\phi$ itself.  This 
type of deformation would result in the structure that is seen in the 
left plot of figure~\ref{fig:trackSplittingProfiles} in the prompt geometry.
In this case the deformation is not a weak mode since the alignment 
procedure is sensitive to it and corrects the module positions accordingly.
A systematic study of a collection of systematic misalignment scenarios
can be done to see which cosmic splitting plots are sensitive and which 
structures are seen in data.  ... 

%% discuss systematic misalignments and their effects on validations

%% discuss primary vertex validations

In order to ensure high quality data over time, it 
is critical to monitor and improve upon the tracker geometry over time.  
The primary vertex validation is suitable for such time dependent studies
and has been used in the past to help define appropriate alignment 
strategies. 

The primary vertex validation using the position of primary vertices as
an estimator of the true impact parameters of an individual track.  
Residuals can be constructed from the difference between the primary 
vertex and a tracks fitted impact parameter.    If tracks are truly from 
the vertex, then on average the above assumption
will be true.  However, individual tracks which pass through poorly 
aligned regions of the tracker will give larger residuals.  Thus providing
a self consistent probe of the tracker geometry.  

Distributions of PV-IP residuals can be used to provide more precise 
measurements for a given region of the tracker.  Figure~\ref{fig:???}
shows a number of residual distributions in various bins of $\eta$ and
$\phi$.  The mean and RMS of these distributions can provide useful
information about systematic misalignments of the pixel barrel.  In
particular, this validation is sensitive to observed separation of the
pixel half barrels, which tend to move from time to time. 

To quantify the separation of the pixel half barrels, the mean and width
of the residual distributions are plotted as a function of phi.  If a 
separation between the two half barrels is present, a discontinuity will
be present at zero.  Figure~\ref{fig:???} shows an example plot of this
using MC tracks with either the ideal geometry or a geomtry in which the 
two half barrels have been purposefully shifted.  The size of the
discontinuity is size of the physical separation.  

The presence of a shift can have significant impact on vertex measurements,
either through b-tagging or SIP values.  Thus monitoring and correcting
these deformations in time is critical.  ... talk about define IOV 
boundaries. 

\subsection{Magnet}
\label{sec:Magnet}

CMS employs a 4~T superconducting aluminum solenoid magnet to bend tracks for 
both charge identification and momentum resolution.  The field was
chosen to have good momentum resolution, $\Delta p/p\equiv10\%$ at 
$p=1~TeV/c$.  The magnet has an inner bore of 5.9~m, large enough to house
the tracker and both calorimeters, and a length of 12.9~m.  Drawing a 
current of 19.5~kA, the magnets total stored energy is 2.7~GJ.  Making it
one of the largest magnets in the world.  
The outer return yolk of the magnet concetrates the magnetic field in the
region near the muon system, which is placed outside of the solenoid.  

\subsection{ElectroMagnetic Calorimeter}
\label{sec:ECal}

The Electromagnetic calorimeter (ECal) is a high grainularity calorimeter
intended to induce electromagnetic showers which are collected by either
avalanche photodiodes (barrel) or vacuum phototriodes (endcap).  The 
material used is scintillating lead tungstate crystal
which was chosen its: short radiation ($X_0$=0.89~cm) and Moliere (2.2~cm) 
lengths; the time scale in which showers occur, 80\% of light is emmitted
in 25~ns; and the radiation hardness.  The ECal is divided into barrel (EB)
and endcap (EE) regions.

The EB region has an inner radius of 129~cm and is constructed from 36
identical {\it supermodules}, each covering half of the barrel in the 
z-direction (1.479 unit of pseudorapidity).  Each individual crystal 
1 degree in both $\Delta\phi$ and $\Delta\eta$, corresponding to a cross
sectional area of $22\times22~mm^2$, and are 230~mm, corresponding to 
25.8~$X_0$.

The EE region is located at a distrance of 314~cm along the z-direction
and covering the pseudorapidity range $1.479<|\eta|<3.0$.  The crystals
clustered into $5\times5$ {\it supercrystals} which are combined to form
semi-circle structures.  Each crystal has a cross sectional area of 
$28.6\times28.6~mm^2$ and are 220 mm (24.7~$X_0$) in length.  The endcap 
region is also preceded by a preshower which consists of a lead absorber
whos thickness is 2-3~$X_0$ followed by 2 planes of silicon strip detectors.

The energy response of the ECal was measured in test beams.  The energy 
resolution was parameterized according to
\begin{equation}
\left(\frac{\sigma}{E}\right)^2 = \left(\frac{S}{\sqrt{E}}\right)^2 + \left(\frac{N}{E}\right)^2 + C^2,
\end{equation}
where S, N and C represent the stochastic, noise, and constant contributions.
The coefficients measurements for two energy clustering schemes are shown
if figure~\ref{fig:???}.

\subsection{Hadronic Calorimeter}
\label{sec:HCal}

The hadronic calorimeter (HCal) consists of brass absorbers and plastic 
scintillators
in which light is collected from using wavelength-shifting (WLS) fibers. 
Fiber cables transmit light into hybrid photodiodes.  The HCal is separated
into four regions, the barrel (HB), the outer (HO), the endcap (HE), 
and the forward (HF) region.

The HB is made up of 32 towers which cover the pseudorapidity region 
$|\eta|<1.4$, resulting in 2304 towers with a segmetation of 
$\Delta\eta\time\Delta\phi=0.087\time0.087$.  There are 15 brass plates, 
each 5~cm thick and two steel plates for structural stability. Particles 
entering the HCal barrel region fist impinge upon a scillating layer that
is 9~mm thick, instead of the typical 3.7~mm for other scintillating layers.
More details of the HB design and test beam performance can be found
elsewhere~\cite{??}.

The HO region contains 10~mm thick scintillators.  Each scintillating tile
matches the segmentation pattern of the muon stystem's Drift tubes.  
The purpose of the HO is to catch hadronic showers leaking through the 
HB region. This makes the effective length of the barrel region 10~$X_0$
and improves missing transverse energy $E_T^{miss}$ resolution.

The HE region consists of 14 $\eta$ towers with 5 degree segmentation in 
$\phi$ and covers the region between $1.3<|\eta|<3.0$. {\it ... more about the
geometry of the HE...} There are 2304 HE towers in total.  more details of
the design and test beam performance of the HE can be found 
elsewhere~\cite{??}.

The HF region extends between $3.0<|\eta|<5.0$ and is made from steel absober
and quartz fibers.  The fibers are intended to measure Cherenkov radiation.  
The HF will mainly be used for detecting very forward jets and real-time
luminosity measurements.  

\subsection{Muon System}
\label{sec:Muon System}

The Muon system plays an important role in identifying muons.  However, 
because of the vast distance from the interaction point and the muon 
chambers, resolution of low energy muons is dominated by energy loss 
due to multiple scattering in the inner detector.  In this region, it 
is found that the tracker dominates the momentum resolution.  However, 
for high energy tracks, the combination of the tracker and muon system
provide superior energy resolution to either system alone.  This effect
is shown in figure~\ref{fig:???} for the barrel and endcap region.  Thus
for high momentum muons, the muon system plays a major role in momentum 
resolution. 

The muon system employs three different gaseous detectors, drift tube (DT)
chambers, cathode strip chambers (CSC), and resistive plate chambers (RPC).
The DT are used in the barrel region, $|\eta|<1.2$, where the magnetic 
field is low.  The CSC detectors are used in the endcaps, 
$1.2<|\eta|<2.4$, where the rates of both muons and the neutron induced
background are high as well and high magnetic field.  The RPC detectors
are used both in barrel and endcaps.  

The RPCs are fast response and have
good timing resolution, while not as sensitive to spatial measurements
as the DTs and CSCs.  Thus, RPC provide the necessary input to distinguish
which bunch crossing a particles should be identified with, which is critical
for triggering.  All three sub-systems participate in level-1 triggering
and provide a key element to level-1 triggering.

The DTs are arranged in four layers of wheels made up of 12 segements
each covering 30 azimuthal degrees.  The outermost layer has 1 extra 
segment in the top an bottom, totaling 14.  Each DT is pair with either
one or two RPCs, two on either side in the first two layers and one on
the inner most edge in the second two layers.  A high-$p_T$ track can 
cross up to 6 RPCs and 4 DTs, providing 44 measurements for momentum 
measurements. 

The CSC are trapesoidal chambers containing 6 gas gaps, each with a 
corresponding cathode strips running radially and anode wires running
azimuthally.  Charge from iozined gas is collected on strips and wires.
Signals on the wires are fast and can be used for level-1 triggering,
while cathodes provide a better measurement of position, on the order
of 200~\microns.

\subsection{Trigger and data aquisition}



\subsection{Track Reconstruction}
\label{sec:trackRECO}



\subsection{Muon Reconstruction}
\label{sec:muonRECO}


Muon tracks are reconstructed using the Kalman filter technique
starting with RecHits at the inner most radius.
In the barrel region, where the DTs provide the most sensitive 
information about particle position, primitive track segments
are reconstructed and fed to the Kalman filter.
In the endcap, 3D hits are constructed from wire and strip 
information which is then pass to the Kalman filter. In both
cases, the RecHits from the RPCs are used.  Track states are 
propagated through each layer taking into account the effects
of energy loss in material, multiple scattering and inhomogeneities
in the magentic field.  Once the outer most measurement is reached,
the Kalman filter is then applied in reverse to define the final
set of track parameters.  After the track is extrapolated to the 
to the nominal interaction point, the beam-spot, a vertex-constrained
fit to the track parameters is performed.
Muons can also be reconstructed using the information from the 
silicon tracker.  In this case, tracker hits which are compatible
with the muon track are added to the Kalman filter.

Muon reconstruction typically makes use of other subdetectors, such
as the calorimeters.  Calorimeter cells which are compatible with 
the extrapolated muon track should have energy deposits which are 
consistent with the presence of a minimum ionizing particle.  
Since events from pile-up can contribute the HO can be efficiently
used to discriminate against pile-up activity.  

Muon are also typically required to be well isolated.  An isolation
cone is defined around the muon direction and a maximum threshold of 
$E_T$ is allowed from calorimeter cells whithin this corn, or a maximum
sum $p_T$ is allowed from tracks within this cone.  These thresholds are
typically set as a fraction of the candidate muons $p_T$.

The details of muon identification and isolation typically depend on
the specific use case.  Thus, more detailed explainations will follow
in conjunction with descriptions of analyses. 

\subsection{Electron Reconstruction}
\label{sec:electronRECO}

Electron produce tracks in the silicon track and induce showers 
in the ECal.  In beam test, typically, more than 90\% of the incident 
energy of a single electron is contained in a $3\times3$ array of 
crystals and more than 95\% is contained in a $5\times5$ array.  
Within CMS, showing patterns are broadened by bremsstrahlung 
induced by the presence material within the ECal and by the
presence of a strong magnetic field.  In order to better identify 
the energy deposits from electrons clustering algorithms are 
used to combine energy measurements from individual crystals which
take into account effects of bremsstrahlung and the magnetic field.  

Seed crystals are identified which contain energy above some predefined
threshold.  Nearby crystal are then combined to for a cluster.  In 
an analogous procedure, cluster are combined into superclusters 
starting from seed clusters.  Supercluster energies are then 
corrected for systematic effects based on shower shape and the
location of the supercluster.  The position of the reconstructed
electron candidate is then measured from the energy-weighted mean
position of each of the crystals.

Superclusters are matched to pixel hits in the tracker which serve 
as seed for electron tracks to be built.  Tracks are the built using
a modified Kalman filter known as the Gaussian Sum Filter (GSF) which
is a nonlinear filter which makes use of guassian mixtures to model
errors.  The GSF algorithm allows for momentum at either end of the 
tracker to be measured reliably and allows for the amount of brem to 
be accurately estimated.  

E-p combination...

electron isolation ...
 
electron identification ...

\subsection{Photon Reconstruction}
\label{sec:photonRECO}

Photon energy is clustered according to either the hybrid (EB) or the 
island (EE) algorithm, similar to electrons.  For unconverted photons,
most of the energy will be contained in a $3\times3$ array of crystals.
In contrast, photons which get converted into $e^+e^-$ pairs can 
cause the supercluster to spread over more crystal.  

The R9 variable provides a powerful discriminator for converted photons.
R9 is defined as the ratio of the energy in a $3\time3$ array, centered 
on the highest energy crystal, to the total energy in a supercluster.
Values of R9 close to 1 are indicitive of unconverted photon

photon isolation...

\subsection{Jet Reconstruction}
\label{sec:jetRECO}

