\chapter{Conclusions}
\label{sec:Conclusions}
\chaptermark{Conclusions}

\section{Interpreting Higgs properties}

Give an argument of why the data looks a lot like 
something that is breaking electroweak symmetry:

\begin{itemize}
  \item Relative rates bewteen ZZ, $\gamma\gamma$, Z$\gamma$
  \item Show the best-fit $\mu$ in different channels
  \item Discuss the fact that spin-1 models and the most 
    likely spin-2 model has been excluded
  \item Discuss $f_3$ measurement
\end{itemize}

All of the above points to the fact that resonance observed
at the LHC is, in fact, a scalar resonance that breaks electroweak 
symmetry.  Furthermore this scalar is likely to couple to 
ZZ through a CP-even effective vertex.  However, the possibility
of small $f_3$ values can't yet be strongly constrained.  Because 
CP-violation in the Higgs sector could help explain 
baryogenesis, constraining $f_3$ is an important probe for physics
beyond the standard model which may also help us to solve a major
puzzle of the observable universe.  
A CP-violating Higgs sector could also be the first evidence of 
SUSY in which there are very natural candidates for dark matter. 
Measuring $f_2$ could also
provide hints to whether there is structure to the Higgs and
possibly point to new forces yet to be discovered.  

\section{Future work}

In light of this the discover of Higgs provides us with a new 
window in which precision measurements might hint to new physics
beyond the standard model.  Because the Higgs was only observed
one year ago, since this document was written, many of the tools
for measuring properties have yet to be fully refined, or in some
cases developed at all.  The tools discussed in 
chapter~\ref{sec:HiggsPhen} will continue to serve as a probe to 
constrain the certain properties of the Higgs, especially in the
measurement of $f_3$.  However, new tools will be necessary to 
fully map out the structure of the HZZ amplitude.  There is 
also the possibility that other processes or even a new collider 
might serve as a better tool to probe CP-violation in the HZZ 
amplitude.  

As a point of reference, it is instructive to estimate the
expected sensitivity that CMS can reach using the current method
for measuring $f_3$.  Generating and fitting an asimov dataset
from the model used in section~\ref{sec:HZZ4l}, one can gauge
the expected sensitivity to $f_3$ at various luminosities.  
Two benchmarks have been tested, $\mathscr{L}_{int}=300~fb^{-1}$
 and $\mathscr{L}_{int}=3000~fb^{-1}$ with $\sqrt{s}$, the latter
being the expected lifetime integrated luminosity of the LHC. 
figure~\ref{fig:???} show the 1D likelihood scan of the asimov
dataset for these two scenarios.  Assuming a pure CP conserving 
amplitude, $f_3=0$, the 95\% CL upper limit on $f_3$ is found to
be X.XX (Y.YY) for the 300 (3000)~$fb^{-1}$ scenario.  

\begin{figure}
\begin{center}
%\includegraphics[width=.49\linewidth]{ConclusionPlots/}
%\includegraphics[width=.49\linewidth]{ConclusionPlots/}
\label{fig:fa3Projections}
\caption{Projection of $f_3$ fits using asimov dataset for 
300 and 3000~$fb^{-1}$ at $\sqrt{s}=14~TeV$.}
\end{center}
\end{figure}

Similar measurments can be made with other processes such as
$q\bar{q}\to Z^*\to ZH$, $e^+e^-\to Z^*\to ZH$, or
$q\bar{q}\to H+q\bar{q}$ (VBF).  Note that the equivalent 
$f_3$ parameter for these processes will have slightly different 
meaning.  For example, table~\ref{table:fa3Conversion} summarizes
how the value for $f_3$ of the $H\to ZZ^*$ process can be 
translated.  The numbers in this table reflect the fact that 
the ration $\sigma_1/\sigma_3$, as define in 
section~\ref{sec:HiggsPhen}, can vary by orders of magnitude
between different processes.  As a results, the
sensitivity to CP-violation in terms $f_3(H\to ZZ^*)$ is
expected to be much larger for other processes.  

\begin{table}
\begin{center}
\begin{tabular}{c|cccc}

 &\begin{rotate}{60}$f_3(H\to ZZ^*)$ \end{rotate} 
 &\begin{rotate}{60}$f_3(q\bar{q}\to Z^*\to ZH)$ \end{rotate}
 &\begin{rotate}{60}$f_3(e^+e^-\to Z^*\to ZH$ \end{rotate}
 &\begin{rotate}{60}$f_3(e^+e^-\to Z^*\to ZH$ \end{rotate} \\ \hline

$f_3(H\to ZZ^*)$             & 0.5 & -- & .98 & .99 \\
$f_3(q\bar{q}\to Z^*\to ZH)$ & -- & 0.5 & -- & -- \\
$f_3(e^+e^-\to Z^*\to ZH$    & 1.8$\times10^{-3}$ & -- & 0.5 & .69 \\
$f_3(e^+e^-\to Z^*\to ZH$    & 8.4$\times10^{-3}$ & -- & .31 & 0.5 \\
\hline
\hline

\end{tabular}
\end{center}
\label{table:fa3Conversion}
\caption{Demonstration of $f_3$ values for various processes.}
\end{table}

The diagrams in figure~\ref{fig:HZZprocesses} demonstrate that 
these 
processes are all equivalent, except they probe different 
regions of phase space.  Thus, the differential cross sections
presented in section~\ref{sec:HiggsPhen} are all still applicable. 
Consider the process $e^+e^-\to Z^*\to ZH$.  In this case, the
Z and H are both on-shell and can be approximated as constant, 
leaving three non-trivial angular distributions which describe 
the kinematics of this process. Figure~\ref{fig:ILCprojections}
shows
these distributions for several scalar models, SM Higgs, a 
pseudoscalar, and two mixed parity scalar models with 
phases $\phi_3=0,~\pi/2$.  

\begin{figure}
\begin{center}
\includegraphics[width=.32\linewidth]{ConclusionPlots/angles-ZZHbb_snowmass_3.pdf}
\includegraphics[width=.32\linewidth]{ConclusionPlots/angles-ZZHbb_snowmass_cms_4.pdf}
\includegraphics[width=.32\linewidth]{ConclusionPlots/angles-HZZ4l_snowmass.pdf}
\label{fig:HZZprocesses}
\caption{Diagrams showing the different processes produced
via the HZZ ... }
\end{center}
\end{figure}

\begin{figure}
\begin{center}
\includegraphics[width=.32\linewidth]{ConclusionPlots/ILCshapes_costheta1.pdf}
\includegraphics[width=.32\linewidth]{ConclusionPlots/ILCshapes_costheta2.pdf}
\includegraphics[width=.32\linewidth]{ConclusionPlots/ILCshapes_phi.pdf}
\label{fig:ILCprojections}
\caption{Angular distributions, $\cos\theta_1$ (left), 
$\cos\theta_2$ (middle), and $\Phi$ (right), of four different 
scalar models of the process $e^+e^-\to Z^*\to ZH$.  Markers
show angular distributions from JHUGen simulations while
lines show projections of the angular distributions presented
in section~\ref{sec:HiggsPhen}. Red line/circles represent a 
SM Higgs, blue lines/diamonds represent a pseudoscalar, green
lines/squares and purple lines/solid circles represent a 
mixed parity scalar ($f_3$=0.1) with various phases.}
\end{center}
\end{figure}

Similar to the $H\to ZZ*$ 
analysis, a kinematic discriminant built according to
equation~\ref{eq:KD} can be used to measure 
$f_3$ according to equation~\ref{eq:fa3}.  
Figure~\ref{fig:???} Shows this discriminant for SM Higgs
events, pseudoscalar events, and a mixed parity scalar
corresponding to $f_3=0.5$.  To justify that there are no
biases introduced by approximation in equation~\ref{eq:fa3},
toys studies have been performed for various values of 
$f_3$.  These studies are summarized in figure~\ref{fig:???};
no biases are observed.  Projections for a future $e^+e^-$
colliders can be estimated assuming a collision energy  
of 250~GeV and an integrated luminosity of 250~$fb^{-1}$.
The production cross section times branching ratio for 
this process is roughly ...

\section{Conclusions}



