\chapter{Conclusions}
\label{sec:Conclusions}
\chaptermark{Conclusions}

A set of analysis tools which can be used to enhance the
sensitivity of diboson signatures as well as study resonance
properties has been developed.  Two specific implementations 
of these tools have been presented in the context of searches
for a Higgs boson. 

A search for a SM Higgs boson using $ZZ^{(*)}\to2\ell2q$ events
was presented.  
Drawing on the ideas presented in Chapter~\ref{sec:HiggsPhen},
a novel discriminant was used to reduce the dominant SM
background. 
Techniques for measuring expected background
shapes and event yields using data control regions were used.
No significant deviation from the background 
only hypothesis was found and upper limits were set.
Standard model Higgs boson mass hypotheses between 340 and 390~GeV
were ruled out at 95\% confidence level.

A search for a SM Higgs boson using $ZZ^{(*)}\to4\ell$ events
was presented.  Again, ideas from Chapter~\ref{sec:HiggsPhen} 
were used to build discriminants to further enhance sensitivity
to signal events.  A local excess of events near 126~\GeV was
observed.  The MELA technique was found to provide a
large increase of the observed significance of this 
excess which is found to be $6.8\sigma$ at 125.7~GeV.
At other masses, no significant excesses was observed 
and Higgs mass hypotheses in the range [114.5,119] and [129-800]
were ruled out at 95\% confidence level. 

Other discriminants were designed to test the
compatibility of excess in data with respect to either
a SM Higgs or
a number of alternative hypotheses using event kinematics.  All 
test show that data prefers the SM Higgs hypotheses over
the alternative hypotheses.  In the specific case of a
pseudoscalar alternative hypothesis, data disfavors this
hypothesis at the level of 0.04\%.  These property
measurements are summarized in Figure~\ref{fig:spinParitySummary}.
%have helped to shape our understanding of the
%observed resonance in that they strongly suggest that 
%this resonance is likely to be a scalar.
The contribution of CP-violating 
interactions were constrained through the measurement of $f_{3}$.
The best-fit
value of this parameter is found to be $f_{3}=0.00^{+0.16}_{-0.00}$
which is consistent with SM expectation.  The 95\% confidence 
interval of this parameter is found to be [0.00,0.49].

\begin{figure}
\begin{center}
\includegraphics[width=.89\linewidth]{ConclusionPlots/JP_SummaryPlot.eps}
\caption{Distribution of test statistics for SM Higgs toys (blue),
 alternative $J^P$ signals toys (orange), and the observed test
statistic (points).}
\label{fig:spinParitySummary}

\end{center}
\end{figure}

Hypothesis separations measurements
were performed using WW events to access the compatibility 
of data with either a SM Higgs boson or a minimal coupling graviton.
This result has also been combined with the ZZ result by performing
simultaneous fits in both channels\cite{CMS:yva}.  The results of
this is shown in Figure~\ref{fig:2mp_spinCombination}.  The median
toy of the SM Higgs distribution
has a $CL_s$ value of 1.25\%, corresponding to an average
separation of $3.0\sigma$.  The data is found to disfavor the
minimal coupling graviton with a $CL_s$ value of 0.6\%, compared
to the observed $CL_s$ of 1.3\%\footnote{Note that this results
corresponds to an earlier version of the analysis~\cite{CMS:xwa}}
and 6.8\% using the ZZ and WW channels alone.  Other measurements
performed by the ATLAS collaboration~\cite{ATLAS:2013mla,ATLAS:2013nma} using the same tools developed in Chapter~\ref{sec:HiggsPhen}
are consistent with those presented in Section~\ref{sec:HZZ4l}.

\begin{figure}
\begin{center}
\includegraphics[width=.49\linewidth]{ConclusionPlots/hvv_a-posteriori.qvals.root_2pmgg.eps}
\caption{Distributions of the test statistic comparing the 
SM Higgs hypothesis against the $J^P=2_m^+$ hypothesis using a
simultaneous fit of the signal strength in the ZZ and WW channels.
The orange distribution represents the SM Higgs toys, the blue 
distribution represents the $2_m^+$ hypothesis.  The red arrow
show the observed test statistic. }
\label{fig:2mp_spinCombination}
\end{center}
\end{figure}

Cross section measurements in other channels also support the 
SM Higgs hypothesis~\cite{CMS:yva}.  The left plot of
Figure~\ref{fig:crossSectionsByChannel} shows the best-fit signal 
strength of each decay channel separately.  The best-fit signal
for different production mechanism is shown
in the right plot of Figure~\ref{fig:crossSectionsByChannel}.
All are statistically consistent with the the SM Higgs hypothesis,
$\mu=1$.  
As described in Chapter~\ref{sec:HiggsPhen} it is expected that
the fermionic couplings to the Higgs field will scale with the
mass of the fermion and the boson couplings to the Higgs field
will scale with the square of the vector boson's mass.  
Figure~\ref{fig:couplingMeasurements} where the best-fit
fermionic coupling and the square-root of the HVV couplings
divided by twice the Higgs vacuum expectation value are plotted.
All couplings measured thus far are consistent with a linear
correlation between the couplings and masses.  

\begin{figure}
\begin{center}
\includegraphics[width=.49\linewidth]{ConclusionPlots/sqr_mlz_ccc_mH1257_decay.eps}
\includegraphics[width=.49\linewidth]{ConclusionPlots/sqr_mlz_ccc_mH1257_prod.eps}
\caption{Best-fit signal strength modify, $\mu$, for various 
production and decay modes.  Red error bar represent the 68\%
confidence interval of the individual measurements.  Black lines
represent the combined measurement of all channels (production and
decay); the green band represents the the 68\% confidence interval.
All fits are done for a fixed mass hypothesis, $m_H=125.7$~GeV, 
which correspond to the combined best-fit value.}
\label{fig:crossSectionsByChannel}
\end{center}
\end{figure}

\begin{figure}
\begin{center}
\includegraphics[width=.49\linewidth]{ConclusionPlots/sqr_m6summary_run.eps}
\caption{Summary of the fits for deviations in the coupling for the generic five-parameter model not effective loop couplings, expressed as function of the particle mass. For the fermions, the values of the fitted Yukawa couplings hff are shown, while for vector bosons the square-root of the coupling for the hVV vertex divided by twice the vacuum expectation value of the Higgs boson field. Particle masses for leptons and weak boson, and the vacuum expectation value of the Higgs boson are taken from the PDG. For the top quark the same mass used in theoretical calculations is used (172.5 GeV) and for the bottom quark the running mass $m_b$($m_H=125.7$~GeV)=2.763~GeV is used.}
\label{fig:couplingMeasurements}
\end{center}
\end{figure}

The measurements discussed above strongly suggest that the 
resonance observed is a scalar which participates in
electroweak symmetry breaking.  Extensions to the SM which 
fall under the generic 2HDM class provide an interesting
framework to further study the Higgs sector.  These models
predict two more neutral scalar bosons and could lead to 
CP-violating interactions.  As discussed in Chapter~\ref{sec:intro}
this could either help to explain the baryon asymmetry problem
or even dark matter, if the specific for of 2HDM turns out
to be SUSY.

Although CMS measurements have begun to constrain the presence
of CP-violating interactions by setting limits on $f_{a3}$, 
these measurements still have large errors.  However, the same
tools which are currently being used in the $H\to ZZ$ process 
could be applied to other processes at either the LHC or a
future $e^+e^-$ collider.    Projected sensitivities were
estimated for high luminosity LHC scenarios
and future colliders in Chapter~\ref{sec:FutureMeasurements}.
These projections suggest that other Higgs processes, such as 
$q\bar{q}\to ZH$ or $q\bar{q}\to Hq\bar{q}$, will play 
an important role in the campaign for precision measurements
of Higgs properties.   

Other mechanisms for electroweak symmetry breaking include 
models in which the Higgs is composite.  Measuring all of
the HZZ amplitude parameters may one day provide
hints of compositeness.  However, it is necessay to use 
more advanced techniques in order to measure all parameters.
The natural evolution of the MELA techniques were discussed in 
Chapter~\ref{sec:FutureMeasurements}.  
The advantages and disadvantages of using either the MELA 
technique or multidimensional fits were compared.
  
The MELA techniques have provided emmense utility
to high energy physics community.  These tools have been use to
not only help discover but characterize the 125~GeV Higgs-like
resonance both at CMS and ATLAS~\cite{ATLAS:2013nma}.  The property
measurements made have helped to shape our understanding of the
role this resonance plays in nature and whether new physics is
involved in its interaction with the SM fields.  Even in the next 
generation of experiments, the MELA techniques will
continue to provide a framework for performing high precision
measurement and, hopefully, one day help us to better understand
the universe we live in.



