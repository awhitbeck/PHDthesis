



\chapter{Conclusions}
\label{sec:Conclusions}
\chaptermark{Conclusions}

\section{Interpreting Higgs properties}

Give an argument of why the data looks a lot like 
something that is breaking electroweak symmetry:

\begin{itemize}
  \item Relative rates bewteen ZZ, $\gamma\gamma$, Z$\gamma$
  \item Show the best-fit $\mu$ in different channels
  \item Discuss the fact that spin-1 models and the most 
    likely spin-2 model has been excluded
  \item Discuss $f_3$ measurement
\end{itemize}

All of the above points to the fact that resonance observed
at the LHC is, in fact, a scalar resonance that breaks electroweak 
symmetry.  


Furthermore this scalar is likely to couple to 
ZZ through a CP-even effective vertex.  However, the possibility
of small $f_3$ values can't yet be strongly constrained.  Because 
CP-violation in the Higgs sector could help explain 
baryogenesis, constraining $f_3$ is an important probe for physics
beyond the standard model which may also help us to solve a major
puzzle of the observable universe.  
A CP-violating Higgs sector could also be the first evidence of 
SUSY in which there are very natural candidates for dark matter. 
Measuring $f_2$ could also
provide hints to whether there is structure to the Higgs and
possibly point to new forces yet to be discovered.  

\section{Conclusions}

...

