\chapter{Higgs Phenomenology at the LHC}
\label{sec:HiggsPhen}
\chaptermark{Higgs Phenomenology at the LHC}

In the simplest incarnation of the Higgs Mechanism, there is only one free
parameter, the Higgs Mass.  Given, the mass of the Higgs, the partial widths
for both production and decay can be calculated.  Generally, the Higgs 
couples most strongly to the most massive particles in the standard model.  
However, the mechanism for which the weak gauge bosons aquire mass and the 
fermions acquire mass is different.  Thus, the coupling of the Higgs to 
fermions is proportional to the mass of the fermion while the coupling of the 
higgs to the weak gauge bosons is proportional to the square of the gauge 
boson's mass. These features as well as the structure functions of the proton 
combine to produce the predictions show in figure [??] for the production 
cross-section and branching fraction of the Higgs.  In this chapter, the 
terminology of the different production and decay channels is introduced as 
well as the experimental signatures for each.  

\section{Gluon-gluon Fusion}
\label{sec:ggHiggs}

The gluon-gluon fusion production mechanism is responsible for $\sim 90\%$ of 
Higgs events produced at the LHC.  This is large due to the fact that the gluon
cross-section is fairly large compared to the anti-quark cross-section.  
Because the Higgs cannot couple to the gluon directly, the interaction must
be mediated through a loop, show in figure [??].  The dominant contributions 
come from the heavy quarks, top and bottom.  The production cross-section for
this process varies from XXXX fb$^{-1}$ to YYYY fb $^{-1}$ for Higgs masses 
between 110 and 800 GeV.   

\section{Weak Vector Boson Fusion}
\label{sec:VBFHiggs}

The Weak Vector Boson Fusion (VBF) production mechanism, has the next to
largest cross section at the LHC.  The fynman diagram for this process is 
shown in figure [??].  The signature of this production mechanism is two 
high energy jets at high values of pseudo-rapidity.  Because of next-to-leading
order (NLO) and next-to-NLO (NNLO) QCD effects, gluon-gluon fusion events can
also have this same signature.  At such, event class which attempt to 
distinguish the VBF production mechanism tend to have a large contamination 
from gluon-gluon fusion.  

Some discussion on the kinematics of VBF events??

\section{Other Production Mechanisms}
\label{sec:VHiggs}

Other production mechanisms include Higgs produced in association with either
a weak gauge boson, depicted in figures [??], and top pairs, depicted in 
figure [??].  In these cases either W or Z can be tagged or the presence of 
b-jets can be included.  However, these processes only make up ~5\% of the 
total Higgs production cross section.  As such, having significant sensitivity
to these production mechanisms requires very high amount of integrated luminosity (of the order of 100 fb$^{-1}$) and thus will not be discussed further.  

\section{Decay Channels}
\label{sec:HiggsDecays}

The partial decay width of the higgs, just as with productions, typically 
scale with either the mass of the decay product or the mass squared, 
depending on whether the decay is bosonic or fermionic.  As such, at low 
mass, where the production of weak gauge bosons is suppressed from 
phase-space effect, b-quarks are the dominant decay, making up $60\%$ of 
the events.  At high mass, the leading decays are to W and Z pairs.  There 
are also several decays which are mediated through loops of either quarks 
or gauge bosons.  These decays are the di-photon, di-gluon, and Z-photon 
channels.  Becuase of the distinct signature of the di-photon final state,
this channel is one of the most sensitive channels for discoverying the Higgs.  
Aside from this, the ZZ and WW decay channels are the next most sensitive, 
especially at high mass and especially in the ZZ fully leptonic and WW 
semi-leptonic final states. 

\section{Kinematics of scalar resonances}
\label{sec:Kinematics of scalar resonances}

The simplest incarnation of the Higgs mechanism predicts one extra scalar 
boson with the simplest coupling to the SM fields.  However, there are 
models which go beyond the minimal Higgs model and predict also other scalars
which would couple differently to the SM fields.  To generalize this, one 
can write down the most general way in which a scalar can couple to the SM
feilds.  For bosonic decays one gets the amplitude show in equation [??].  
For fermionic decays, the amplitude is shown in equation [??]. Each of these
terms results in distinct kinematic distributions of the final state particles.
For example, the SM Higgs couples to the W's and Z's only through the a1 term,
while it couples to the photons through a combination of the a1 and a2 terms.

By parameterizing kinematic distributions in terms of these couplings, one
can also use these ideas to measure properties of any new resonance observed 
at the LHC.  It turns out that one of the best channels for measuring Higgs
properties is the ZZ decay where both Z's decay leptonically (either 
electrons or muons).  This is due to the complex topology of the decay.  In
contrast, if one considers Higgs to di-photon decays, there is only one angle 
to describe these events and it has the same distribution for all scalar 
resonances.  This leave the di-photon channel to only be sensitive to the spin 
of the resonance if one is restricted to only use kinematic information of the
two final state photons\footnote{A similar technique can be used look at
the kinematics of tagging jets in VBF or ttH events.}.

Becuase of the ZZ channels sensitivity over a broad range of Higgs

\subsection{Kinematics of vectors and tensors resonances}
\label{sec:Kinematics of vector and tensor resonances} 

