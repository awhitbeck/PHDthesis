\chapter{Higgs Phenomenology at the LHC}
\label{sec:HiggsPhen}
\chaptermark{Higgs Phenomenology at the LHC}

In the simplest incarnation of the Higgs Mechanism, there is only one free
parameter, the Higgs Mass.  Given, the mass of the Higgs, the production
cross section, branching fractions, and width can be calculated. 
Generally, the Higgs 
couples most strongly to the most massive particles in the standard model.  
However, the mechanism for which the weak gauge bosons aquire mass and the 
fermions acquire mass is different.  Thus, the coupling of the Higgs to 
fermions is proportional to the mass of the fermion while the coupling of the 
higgs to the weak gauge bosons is proportional to the square of the gauge 
boson's mass. These features as well as the structure functions of the proton 
combine to produce the predictions show in figure~\ref{fig:HiggsProdXS} 
for the production 
cross-section and branching fraction of the Higgs.  
\begin{figure}
\begin{center}
\includegraphics[width=.49\linewidth]{HiggsPhenPlots/Higgs_XS_8TeV.eps}
\includegraphics[width=.49\linewidth]{HiggsPhenPlots/Higgs_BR.eps}
\label{fig:HiggsProdXS}
\caption{left: Higgs production cross section vs $m_H$ for 
different processes at $\sqrt{s}=8~TeV$. right: Higgs branching
ratios vs $m_H$.  Both calculations are taking from the LHC Higgs
cross section working group. }
\end{center}
\end{figure}
The general structure of the branching ratios and production cross 
sections in different channels is one of the key features of the Higgs 
boson.  In particular,the feature that the $\gamma\gamma$ and $Z\gamma$
branching ratios are much smaller than the $ZZ$ or $WW$ is a result 
of the Higgs not coupling directly to massless particles.  Thus, these
process are required to procede through loops which would contain massive
particles, usually either top quarks or W bosons.  This is one of the 
most distinguishing features which results in a large suppression of 
the $\gamma\gamma$ and $Z\gamma$ channels with respect to the ZZ and WW
channels.

In this chapter, the 
terminology of the different production and decay channels is introduced as 
well as the experimental signatures for each.  Kinematics of spin-0, spin-1, 
and spin-2 resonances decaying to two vector bosons is introduced.  
Techniques for using decay kinematics for increasing signal sensitivity
and property measurements are presented.  

\section{Higgs Signatures}

\subsection{Gluon-gluon Fusion}
\label{sec:ggHiggs}

The gluon-gluon fusion production mechanism is responsible for $\sim 50\%$ of 
Higgs events produced at the LHC.  This is large due to the fact that the 
gluon-gluon cross section at large invariant masses is large 
compared to the quark-anti-quark cross section.  
However, because the Higgs cannot couple to gluons directly, the interaction 
must be mediated through a loop, show in figure~\ref{fig:ggFusion}.  The dominant 
contributions 
come from the heavy quarks, top and bottom, which couple strongly to both
gluons and the Higgs.  The production cross section for
this process varies from $3\times10^{-2}~pb$ to $40~pb$  for Higgs masses 
between 80 and 1000 GeV and $\sqrt{s}=8~TeV$.

\begin{figure}
\begin{center}
\unitlength=1mm
\begin{fmffile}{ggFusion}

\begin{fmfgraph*}(60,40) \fmfpen{thick}
  \fmfleft{i1,i2} \fmfright{o1}
  \fmf{gluon}{i1,v1} \fmf{gluon}{i2,v2}
  \fmf{fermion,label=$t$}{v1,v2}
  \fmf{fermion,label=$\bar{t}$}{v3,v1}
  \fmf{fermion,label=$\bar{t}$}{v3,v2}
  \fmf{dashes,label=$H$}{v3,o1}
\end{fmfgraph*}

\end{fmffile}
\end{center}
\label{fig:ggFusion}
\caption{Feynman diagram depicting gluon-gluon fusion production
of a Higgs boson.}
\end{figure}

\subsection{Weak Vector Boson Fusion}
\label{sec:VBFHiggs}

The Weak Vector Boson Fusion (VBF) production mechanism has the next to
largest cross section at the LHC, depicted in figure~\ref{fig:VBF}.  The 
signature of this production mechanism is two energetic jets at high values 
of pseudorapidity.  Because of next-to-leading
order (NLO) and next-to-NLO (NNLO) QCD effects, gluon-gluon fusion events can
also have this same signature.  As such, event classes which attempt to 
distinguish the VBF production mechanism tend to have a large contamination 
from gluon-gluon fusion.  Usually the kinematics of the spectator jets
can be used to further isolate VBF-like events. 

\begin{figure}
\begin{center}
\unitlength=1mm
\begin{fmffile}{VBF}

\begin{fmfgraph*}(40,30) \fmfpen{thick}
  \fmfleft{i1,i2} \fmfright{sp1,H,sp2}
  \fmf{fermion,label=$q$}{i1,v1,sp1} 
  \fmf{fermion,label=$\bar{q}$}{i2,v2,sp2}
  \fmf{photon,label=$V$}{v1,v3}
  \fmf{photon,label=$V$}{v2,v3}
  \fmf{dashes,label=$H$}{v3,H}
\end{fmfgraph*}

\end{fmffile}
\end{center}
\label{fig:VBF}
\caption{Feynman diagram depicting weak vector boson fusion production
of a Higgs boson.}
\end{figure}

\subsection{Other Production Mechanisms}
\label{sec:VHiggs}

Other production mechanisms include Higgs produced in association with either
a weak gauge boson, depicted in figures [??], and top pairs, depicted in 
figure~\ref{fig:VHttH}.  In these cases either W or Z can be tagged or the presence of 
b-jets can be included.  However, these processes only make up ~5\% of the 
total Higgs production cross section at the LHC.  As such, having significant 
sensitivity to these production mechanisms requires very high amount of integrated 
luminosity (of the order of 100 fb$^{-1}$). 

\begin{figure}
\begin{center}
\unitlength=1mm
\begin{fmffile}{VH_and_ttH}

\begin{fmfgraph*}(40,20) \fmfpen{thick}
  \fmfleft{i1,i2} \fmfright{V,H}
  \fmf{fermion,label=$q$}{i1,v1}
  \fmf{fermion,label=$\bar{q}$}{i2,v1}
  \fmf{photon,label=$V^{*}$}{v1,v2}
  \fmf{photon,label=$V$}{v2,V}
  \fmf{dashes,label=$H$}{v2,H}
\end{fmfgraph*}
~
\begin{fmfgraph*}(40,20) \fmfpen{thick}
  \fmfleft{i1,i2} \fmfright{sp1,H,sp2}
  \fmf{fermion,label=$g$}{i1,v1}
  \fmf{fermion,label=$g$}{i2,v2}
  \fmf{photon,label=$t$}{sp1,v1,v3}
  \fmf{photon,label=$\bar{t}$}{sp2,v2,v3}
  \fmf{dashes,label=$H$}{v3,H}
\end{fmfgraph*}

\end{fmffile}
\end{center}
\label{fig:VHttH}
\caption{Feynman diagram depicting associated production (left) and $t\bar{t}$ 
fusion production of a Higgs boson.}
\end{figure}

\subsection{Decay Channels}
\label{sec:HiggsDecays}

The partial decay width of the higgs, just as with productions, typically 
scale with either the mass of the decay product or the mass squared, 
depending on whether the decay is bosonic or fermionic.  As such, at low 
mass, where the production of weak gauge bosons is suppressed from 
phase-space effects, b-quarks are the dominant decay, making up $60\%$ of 
the events.  At high mass, the leading decays are to W and Z pairs.  The 
di-photon, di-gluon, and $Z\gamma$ decay channels are mediated through loops 
of either quarks or gauge bosons.  Because of the distinct signature of ZZ,
WW, and $\gamma\gamma$ decays, these channels are of the most sensitive for 
discoverying a Higgs-like resonance.  The $4\ell$ final state of the ZZ 
channel is especially promising because it is a high resolution, fully 
reconstrubible channel.  

\section{Kinematics of scalar resonances}
\label{sec:Kinematics of scalar resonances}

The simplest incarnation of the Higgs mechanism predicts one scalar 
boson with the simplest coupling to the SM fields.  However, there are 
models which go beyond the minimal Higgs model and predict also other scalars
which would couple differently to the SM fields.  To generalize this, one 
can write down the most general way in which a scalar can couple to the SM
feilds.  For bosonic decays one gets the amplitude show in 
equation~\ref{eq:scalarAmp}.  
\begin{equation}
  \mathscr{A}(X\to VV) = v^{-1}(g_1m_v^2\epsilon_1^*\epsilon_2^*+g_2f_{\mu\nu}^{*(1)}f^{*(2),\mu\nu}+g_3f^{*(1),\mu\nu}f_{\mu\alpha}^{*(2)}\frac{q_\nu q^\alpha}{\Lambda^2}+g_4f_{\mu\nu}^{*(1)}\tilde{f^{*(2),\mu\nu}}),
\label{eq:scalarAmp}
\end{equation}
where $f$, $\tilde{f}$ are the field strenght tensor and the conjugate field
strength tensor, $g_i$ are dimensionless couplings, $\epsilon_i$ are the 
polarization vectors of the vector bosons, $\Lambda$ denotes the scale where new
physics could appear, $m_V$ is the mass of the vector boson, and q is the
momentum of the VV-system.  
This amplitude correspond to three independent
lorentz structures and can be rewritten as,
\begin{equation}
\mathscr{A}(X\to VV) = v^{-1}\epsilon_1^{*\mu}\epsilon_2^{*\nu}(a_1g_{\mu\nu}m_X^2+a_2q_\mu q_\nu+a_3\epsilon_{\mu\nu\alpha\beta}q_1^{\alpha}q_2^{\beta}). 
\end{equation}
The SM Higgs couples to the W's and Z's only through the $a_1$ term,
while it couples to the photons through an effective coupling which is a 
combination of the $a_1$ and $a_2$ terms.  A CP-odd scalar, commonly referred 
to as a pseudoscalar, couples to gauge bosons through the $a_3$ term.

The amplitude can be broken into several 
more specific amplitudes, known as helicity amplitudes, corresponding to the 
helicity states of the 
vector bosons, where the quantization axis is taken to be ????.  For a scalar
resonance, there are only three non-zero helicity amplitudes out of the 
nine permutations,

\begin{center}
\begin{subequations}
  \begin{equation}
    A_{00} = -\frac{m_X^2}{v}(a_1\sqrt{1+x}+a_2\frac{m_1m_2}{m_X^2}x,    \end{equation}
  \begin{equation}
    A_{++} = \frac{m_X^2}{v}(a_1+ia_3\frac{m_1m_2}{m_X^2}\sqrt{x}),
    \end{equation}
  \begin{equation}
    A_{--} = \frac{m_X^2}{v}(a_1-ia_3\frac{m_1m_2}{m_X^2}\sqrt{x}),
    \end{equation}
\end{subequations}
\end{center}
where x is defined as
\begin{equation}
x=(\frac{m_X^2-m_1^2-m_2^2}{2m_1m_2})^2-1.
\end{equation}

It turns out there is a convenient basis of 8 variables which can be
used to fully describe $ZZ\to4\ell$ decays in the ZZ rest frame.  These
variables fall into two classes, invariant masses ($m_{ZZ}$, $m_Z$, and 
$m_Z*$),which determine the 
magnitude of the different helicity amplitudes, and angles.  The five 
angles are depicted in figure~\ref{fig:??}.
Each helicity amplitude has a distinct angular distribution while the
magnitude of each helicity amplitude depends on the invariant masses 
of the two Z bosons and the resonance.  Together these combine into the
differential cross section according to~\cite{???}.
As a result, the differential cross section is parameterized in terms of 
the underlying couplings. 
The angular and mass distributions of these for several types of scalar 
models are shown in figure~\ref{fig:ScalarMasses} 
and~\ref{fig:ScalarHelicityAngles}.  The red and blue distributions
correspond to a SM Higgs and pseudoscalar resonances.  The green 
distributions corespond to a scalar model in which the resonance couples
to the vector boson only through the $a_2$ term.  Thus, these three models 
represent the three independent lorentz structures of the most generic
scalar-vector-vector amplitude.  

In principle,
a mixture of these terms can occur.  In fact, there is a small but 
negligible contribution from the $a_2$ exists in the SM from NLO 
electroweak corrections.  In various extensions to the SM, 2 Higgs doublet
models, multiple scalars exist with different CP properties.  It is
even possible that a CP-violating scalar resonance could exist.  The 
distributions in figures~\ref{fig:ScalarMasses} 
and~\ref{fig:ScalarHelicityAngles} clearly demonstrate that with 
enough events, the angular and mass distributions are sufficient to 
determine the couplings of an observed scalar resonance.  

\begin{figure}
\begin{center}
\includegraphics[width=.32\linewidth]{HiggsPhenPlots/spinParityPaper/plots/z1mass_125GeV_spin0_3in1.eps}
\includegraphics[width=.32\linewidth]{HiggsPhenPlots/spinParityPaper/plots/z2mass_125GeV_spin0_3in1.eps}
\label{fig:ScalarMasses}
\caption{Distributions of the Z boson masses.  The smaller of the two masses is
plotted on the right, while the larger of the two masses is plotted on the
left. Markers show simulation of events using JHUGen; lines are projections
of the analytical distribution described above.  Red line/circles correspond
to a SM Higgs, blue lines/diamonds, a pseudoscalar, and green lines/square, 
a CP-even scalar produced from higher dimension operators.}
\end{center}
\end{figure}

\begin{figure}
\begin{center}
\includegraphics[width=.32\linewidth]{HiggsPhenPlots/spinParityPaper/plots/costheta1_125GeV_spin0_3in1.eps}
\includegraphics[width=.32\linewidth]{HiggsPhenPlots/spinParityPaper/plots/costheta2_125GeV_spin0_3in1.eps}
\includegraphics[width=.32\linewidth]{HiggsPhenPlots/spinParityPaper/plots/phi_125GeV_spin0_3in1.eps}
\label{fig:ScalarHelicityAngles}
\caption{Distributions of helicity angles, $\cos\theta_1$ (left), 
$\cos\theta_2$ (middle), and $\Phi$ (right). Markers show simulation of 
events using JHUGen; lines are projections
of the analytical distribution described above.  Red lines/circles correspond
to a SM Higgs, blue lines/diamonds, a pseudoscalar, and green lines/square, 
a CP-even scalar produced from higher dimension operators.}
\end{center}
\end{figure}

Constraining the contribution
from either the $a_2$ or $a_3$ term of the amplitude can be more aptly 
formulated through a reparameterization of the HZZ amplitude; the 
three complex couplings, $a_1$, $a_2$, and $a_3$ can be represented by
four real parameters, two fractions ($f_2$ and $f_3$) and two phase 
($\phi_2$ and $\phi_3$) defined as
\begin{center}
\begin{subequations}
  \begin{equation}
    f_i = \frac{|a_i|^2\sigma_i }{|a_1|^2\sigma_1+|a_2|^2\sigma_2+|a_3|\sigma}
    \end{equation}
  \begin{equation}
    \phi_i = arg(\frac{g_i}{g_1}).
    \end{equation}
\end{subequations}
\end{center}
In the above formula, $\sigma_i$ is the cross section of the process 
corresponding
to $a_i=1$ and $a_{\neq i}=0$.  The fractions represent an effective yield 
resulting from the corresponding term of the amplitude.  In the case where
there is no interference, this interpretation is exact.  These variables
are also straight forward measurables for experiments where rates are 
directly measured, as will be discussed in later sections.  

Similar differential cross sections can be calculated for a generic spin-1 or spin-2 
resonance~\cite{??}.  
Figures~\ref{fig:VectorMasses},~\ref{fig:VectorProdAngles}, 
and~\ref{fig:VectorHelicityAngles} show two choice vector 
resonance models.  
Figures~\ref{fig:TensorMasses},~\ref{fig:TensorProdAngles}, 
and~\ref{fit:TensorHelicityAngles}
show three choice tensor 
resonance models.  Similar to the case of a scalar resonance, sufficient
information is contained in the angular and mass distributions to 
constrain all the parameters of the vector-vector-vector amplitude. 

\begin{figure}
\begin{center}
\includegraphics[width=.32\linewidth]{HiggsPhenPlots/spinParityPaper/plots/z1mass_125GeV_spin1_2in1.eps}
\includegraphics[width=.32\linewidth]{HiggsPhenPlots/spinParityPaper/plots/z2mass_125GeV_spin1_2in1.eps}
\label{fig:VectorMasses}
\caption{Distributions of the Z boson masses.  The smaller of the two masses is
plotted on the right, while the larger of the two masses is plotted on the
left. Markers show simulation of events using JHUGen; lines are projections
of the analytical distribution described above.  Red line/circles correspond
to CP-even vector, blue lines/diamonds, a CP-odd vector.}
\end{center}
\end{figure}

\begin{figure}
\begin{center}
\includegraphics[width=.32\linewidth]{HiggsPhenPlots/spinParityPaper/plots/costhetastar_125GeV_spin1_2in1.eps}
\includegraphics[width=.32\linewidth]{HiggsPhenPlots/spinParityPaper/plots/phistar1_125GeV_spin1_2in1.eps}
\label{fig:VectorProdAngles}
\caption{Distributions of the production angles, $\cos\theta^*$ (left) and 
$\Phi_1$ (right).  Markers show simulation of events using JHUGen; lines 
are projections
of the analytical distribution described above.   Red line/circles correspond
to CP-even vector, blue lines/diamonds, a CP-odd vector.}
\end{center}
\end{figure}

\begin{figure}
\begin{center}
\includegraphics[width=.32\linewidth]{HiggsPhenPlots/spinParityPaper/plots/costheta1_125GeV_spin1_2in1.eps}
\includegraphics[width=.32\linewidth]{HiggsPhenPlots/spinParityPaper/plots/costheta2_125GeV_spin1_2in1.eps}
\includegraphics[width=.32\linewidth]{HiggsPhenPlots/spinParityPaper/plots/phi_125GeV_spin1_2in1.eps}
\label{fig:VectorHelicityAngles}
\caption{Distributions of the helicity angles, $\cos\theta_1$ (left), 
$\cos\theta_2$ (middle), and $\Phi$ (right). Markers show simulation of 
events using JHUGen; lines are projections
of the analytical distribution described above.  Red line/circles correspond
to CP-even vector, blue lines/diamonds, a CP-odd vector.}
\end{center}
\end{figure}

\begin{figure}
\begin{center}
\includegraphics[width=.32\linewidth]{HiggsPhenPlots/spinParityPaper/plots/z1mass_125GeV_spin2_3in1.eps}
\includegraphics[width=.32\linewidth]{HiggsPhenPlots/spinParityPaper/plots/z2mass_125GeV_spin2_3in1.eps}
\label{fig:TensorMasses}
\caption{Distributions of the Z boson masses.  The smaller of the two masses is
plotted on the right, while the larger of the two masses is plotted on the
left. Markers show simulation of events using JHUGen; lines are projections
of the analytical distribution described above.  Red line/circles correspond
to a minimal coupling graviton, blue lines/diamonds, a CP-odd tensor, 
and green lines/square, 
a CP-even tensor produced from higher dimension operators.}
\end{center}
\end{figure}

\begin{figure}
\begin{center}
\includegraphics[width=.32\linewidth]{HiggsPhenPlots/spinParityPaper/plots/costhetastar_125GeV_spin2_3in1.eps}
\includegraphics[width=.32\linewidth]{HiggsPhenPlots/spinParityPaper/plots/phistar1_125GeV_spin2_3in1.eps}
\label{fig:TensorProdAngles}
\caption{Distributions of the production angles, $\cos\theta^*$ (left) and
$\Phi_1$ (right). Markers show simulation of events using JHUGen; lines
are projections
of the analytical distribution described above.  Red line/circles correspond
to a minimal coupling graviton, blue lines/diamonds, a CP-odd tensor, 
and green lines/square, 
a CP-even tensor produced from higher dimension operators.}
\end{center}
\end{figure}

\begin{figure}
\begin{center}
\includegraphics[width=.32\linewidth]{HiggsPhenPlots/spinParityPaper/plots/costheta1_125GeV_spin2_3in1.eps}
\includegraphics[width=.32\linewidth]{HiggsPhenPlots/spinParityPaper/plots/costheta2_125GeV_spin2_3in1.eps}
\includegraphics[width=.32\linewidth]{HiggsPhenPlots/spinParityPaper/plots/phi_125GeV_spin2_3in1.eps}
\label{fig:TensorHelicityAngles}
\caption{Distributions of the helicity angles, $\cos\theta_1$ (left), 
$\cos\theta_2$ (middle), and $\Phi$ (right). Markers show simulation of 
events using JHUGen; lines are projections
of the analytical distribution described above.  Red line/circles correspond
to a minimal coupling graviton, blue lines/diamonds, a CP-odd tensor, 
and green lines/square, 
a CP-even tensor produced from higher dimension operators.}
\end{center}
\end{figure}

\subsection{Variables for Property Measurements}
\label{sec:Spin-parity}

Several extensions
to the SM discussed previously, section~\ref{sec:Introduction}, can result 
in ZZ resonances.  
Consequently, understanding the spin and CP of any new resonace discovered 
at the LHC will be critical to understanding its role in nature.  
Since measuring model parameters requires a description of all backgrounds, 
acceptance effects, and resolution in 8 dimensions, a more efficient way of
constraining resonance properties is to use compact variables to isolate 
specific properties.  Such a variable can be 
built from either the square of the matrix element for two processes, or 
equivalently, the differential cross section, according to 
\begin{equation}
D_{J^P} = \frac{\mathscr{P}_{SMH}}{\mathscr{P}_{SMH}+\mathscr{P}_{J^{P}}}.
\label{eq:KD}
\end{equation}
These type of variable uses the ideal distributions to isolate emphasise
kinematic difference between two choice models.  Since acceptance effects 
will cancel when calculating ratios and resolution effects are relatively 
small, these variable will be close to optimal.  
These variables can either be used either to perform hypothesis separation
or for measuring some continuous model parameters.  

To perform such measurements an accurate description of the detector 
level distribution of D must be modeled.  Simulated Monte Carlo (MC)
events can be used, including all detector simulations, reconstruction
algorithms, and analysis selections, to model the shape of these discriminants.
Thus, MC simulations can effectively be used to model the appropriate 
transfer function for a given analysis.  From here, D can be used either 
as an additional selection variable, or for fitting the shape of data. 
This process of building discriminants from kinematic distributions
using a matrix element calculation paired with MC simulations is known
as the Matrix Element Likelihood Analysis (MELA).

Even with a relatively small number of signal events the MELA technique
can be used to perform hypothesis separation to rule out definite non-SM 
signals.  For example, the variable $D_{0-}$ can be used to isolate
the relevant properties that distinguish a SM Higgs from a purely CP-odd
scalar.  The SM Higgs and pseudoscalar distribution of this variable for 
ideal MC is shown in figure~\ref{??}.  The separation between these two 
models can be quantified using Neyman-Pearson hypothesis testing.  In this 
way, the compatibility of data which respect to either the null hypothesis
(always the SM Higgs hypothesis) or the alternative hypothesis can be 
quantified.  Other models, such as spin-1 or spin-2 models, can be tested 
using variables analogous to $D_{0-}$.  A list of models which will be 
used in section~ref{sec:HZZ4l} are listed in table~\ref{table:alternativeModels} 
along with a description of there couplings.

\begin{table}
\begin{tabular}{cccc}
\hline 
\hline
scenario & X production & $X\to VV$ decay & comments \\
\hline
$0_m^+$ & $gg\to X$ & $g_1\neq 0$ & SM Higgs boson \\ 
$0_h^+$ & $gg\to X$ & $g_2\neq 0$ & scalar with higher-dimension operators \\ 
$0^-$ & $gg\to X$ & $g_4\neq 0$ & pseudoscalar \\ 
$1^+$ & $q\bar{q}\to X$ & $b_2\neq 0$ & exotic pseudovector\\
$1^-$ & $q\bar{q}\to X$ & $b_1\neq 0$ & exotic vector \\ 
$2_m^+$ & $g_1^{2}=g_5^{2}\neq 0$ & $g_1^{2}=g_5^{2}\neq 0$ & graviton-like tensor with minimal couplings\\
$2_h^+$ & $g_4^{2}\neq 0$ & $g_4^{2}\neq 0$ & tensor with higher-dimensional operators\\
$2_h^-$ & $g_8^{2}\neq 0$ & $g_8^{2}\neq 0$ & ``pseudotensor''\\ 
\hline
\hline
\end{tabular}
\label{table:alternativeModels}
\caption{List of alternat signal models to be tested against the SM Higgs 
hypothesis along with a description of the their couplings to ZZ.  Amplitude
parameterizations for spin-0 resonances is given in equation~\ref{eq:???};
parameterizations for spin-1 and spin-2 resonances are given in equations 16 
18, respectively, elsewhere~\cite{???}.}
\end{table}


Certain discriminants have properties which allow them to be efficiently
used to measure model parameters.  Consider the situation in which prior
knowledge allows us to know that a resonance is a scalar in which $f_2=0$.
In this case, $f_3$ can be measured directly using kinematic discriminants,
specifically, $D_{0-}$.  Figure~\ref{fig:fa3comparison} shows this discriminant for
both the SM Higgs (red??), a pseudoscalar (blue??), and a mixed parity 
model corresponding to $f_3=0.5$, $\phi_3=0.0$ (green??), $\pi/2$ (purple??),
$\pi$ (brown??), $3\pi/2$ (magenta??).  All of the mixed parity samples
can be described by a weighted sum of the SM Higgs distribution and the 
pseudoscalar distibution, 
\begin{equation}
\mathscr{P}(D_{0-}|f_3) = |\mathscr{A}_{SMH}|^2 + |\mathscr{A}_{0-}|^2 + 2Re(\mathscr{A}_{SMH}^*\mathscr{A}_{0-}) \simeq (1-f_3)\mathscr{P}_{SMH}(D_{0-})+f_3\mathscr{P}_{0-}.
\label{eq:fa3}
\end{equation}
$P_{SMH}$ and $P_{0-}$ represent the different cross section of the 
SM Higgs model and the pseudoscalar, respectively, 
\begin{equation}
\mathscr{P}_{J^P}(D_{0-}) = \frac{d\Gamma}{dD_{0-}} \sim \int |\mathscr{A}_{J^P}|^2d\Omega.
\end{equation}
Thus, using equation~\ref{eq:fa3} the interference is explicitly neglected.  
Thus, we can gather from figure~\ref{fig:fa3Comparison} that $D_{0-}$ is insensitive to
the interference and the relative phase between $\mathscr{A}_{SMH}$ and
$\mathscr{A}_{0-}$.  A more explicit justification of this procedure can be
realized by drawing toys from the simulation of the full matrix element and
fitting for $f_3$ with equation~\ref{eq:fa3}.  Figure~\ref{fig:???} shows
the result of fitting $f_3$ for several mixed parity models, corresponding 
$f_3=.05,.1,.5$; no bias is found in any of these toys studies.

In contrast, the $D_{0h+}$ discriminant cannot be use measure $f_2$.  
Figure~\ref{fig:???} shows that the interference between the $a_1$
and $a_2$ terms cannot be neglected and depends strongly on the $phi_2$.
This effect means that more advanced techiques which can fit for both the
fraction and the phase simultaneously will be needed to constrain this parameter.

Similar variables can be constructed to help discriminate signal effects 
from SM background events,
\begin{equation}
D_{bkg} = \frac{\mathscr{P}_{sig}}{\mathscr{P}_{sig}+\mathscr{P}_{bkg}}.
\end{equation}
Typically, invariant mass distributions are used in resonances searches. 
As will be shown in chapter~\ref{sec:HZZsearches}, variables similar to 
$D_{bkg}$ have proven to provide a significant increase in sensitivity to 
Higgs-like events if used in conjunction with the relevant invariant
mass distributions.  It should be noted, that these variables are important
for properties as well; understanding properties of signal events first
requires good sensitivity to signal and background events.  

\begin{figure}
\begin{center}
\includegraphics[width=.49\linewidth]{HiggsPhenPlots/D0minusComparison.eps}
\label{fig:fa3Comparison}
\caption{Distributions of $D_{0-}$ for }
\end{center}
\end{figure}

