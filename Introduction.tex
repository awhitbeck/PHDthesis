\chapter{Introduction}
\label{sec:intro}
\chaptermark{Introduction}

The Standard Model (SM) of particle physics \textcolor{red}{(reference)} 
is a mathematical description of the fundamental particles and their 
interactions.  Particles are described by quantized excitations of 
various types of fields.  These fields are the solutions to either the
Klein-Gordon, Dirac, Proca, or Maxwell equation which govern the time
evolution of the solutions for spin-0, spin-1/2, and spin-1 particles.
Although, bound states and possible extensions to the SM can have
larger spin, the standard model doesn't explicitly describe the interaction
of these particles.

The interactions themselves can be derived from the Standard Model lagrangian.
The electroweak interactions of the leptons, for example, are described by
the lagrangian in equation~\ref{eq:GSWlagrangian}.  The probability of the
some initial state, i, evolving into some final state, j, can be 
be expanded in powers of the coupling constants, g and g', according to 
equation equation~\ref{eq:matrixElementSquared}.  This matrix element
squared can also be represented by Feynman diagrams, 
figure~\ref{fyn:eeScattering} 
represents electron-electron scattering at leading order which is 
represented in the SM lagrangian by the second half of 
equation~\ref{eq:QED}.

\begin{center}
\begin{equation}
\mid<ee\mid e^{\frac{g}{2c_w}\bar{\Psi}_e\gamma^{\mu}(I_f^3-2s_w^2Q_f-I_f^3\gamma_5)\Psi_eZ_{\mu}}\mid ee>\mid^2
\label{eq:matrixElementSquared}
\end{equation}
\end{center}

\begin{figure}
\begin{center}
\includegraphics[width=4.0cm]{IntroductionPlots/Electron-scattering.png}
\label{fyn:eeScattering}
\caption{Feynman diagram depicting one of the processes contributing to the leading order
electron-electron scattering amplitude. A similar contribution exist where a Z boson is
exchanged instead of a photon.}
\end{center}
\end{figure}

\begin{center}
\begin{subequations}
%\begin{multiline}
\begin{equation}
\mathscr{L} = \sum_f(\bar{\Psi}_f(i\gamma^{\mu}\partial_{\mu}-m_f)\Psi_f
-eQ_f\bar{\Psi}_f\gamma^{\mu}\Psi A_{\mu})
\label{eq:QED}
\end{equation}
\begin{equation}
+\frac{g}{\sqrt{2}}\sum_i(\bar{a}_L^i\gamma^{\mu}b_L^iW_{\mu}^++\bar{b}_L^i\gamma^{\mu}a_L^iW_{\mu}^-)
\end{equation}
\begin{equation}
+\frac{g}{2c_w}\sum_f(\bar{\Psi}_f\gamma^{\mu}(I_f^3-2s_w^2Q_f-I_f^3\gamma_5)\Psi_fZ_{\mu})
\end{equation}
\begin{equation}
-\frac{1}{4}|\partial_{\mu}A_{\nu}-\partial_{\nu}A_{\mu}-ie(W_{\mu}^-W_{\nu}^+ - W_{\mu}^+W_{\nu}^-)|^2
\end{equation}
\begin{equation}
- \frac{1}{2}|\partial_{\mu}W_{\nu}^+-\partial_{\nu}W_{\mu}^+ - ie(W_{\mu}^+A_{\nu}-W_{\nu}^+A_{\mu})+ig'c_w(W_{\mu}^+Z_{\nu}-W_{\nu}^+Z_{\mu})|^2
\end{equation}
\begin{equation}
-\frac{1}{4}|\partial_{\mu}Z_{\nu}-\partial_{\nu}Z_{\mu}+ig'c_w(W_{\mu}^-W_{\nu}^+-W_{\mu}^+W_{\nu}^-)|^2
\end{equation}
\begin{equation}
-\frac{1}{2}M_{\Phi}^2\Phi^2-\frac{gM_{\Phi}^2}{8M_{W}}\Phi^3-\frac{g'^2M_{\Phi}^2}{32M_W}\Phi^4
\end{equation}
\begin{equation}
+|M_WW_{\mu}^++\frac{g}{2}\Phi W_{\mu}^+|^2+\frac{1}{2}|\partial_{\mu}\Phi+iM_ZZ_{\mu}+\frac{ig}{2c_w}\Phi Z_{\mu}|^2
\end{equation}
\begin{equation}
-\sum_f\frac{g}{2}\frac{m_f}{M_W}\bar{\Psi_f}\Psi_f\Phi
\end{equation}
%\end{multiline}
\label{eq:GSWlagrangian}
\end{subequations}
\end{center}

Starting from this framework, many experimental measurements have been 
accurately predicted.  One famous example is the anomalous magnetic 
moment of the \textcolor{red}{(electron?)}, first calculated by 
Julian Schwinger in \textcolor{red}{19??}
in the context of Quantum Electrodynmics (QED). 

Before the advent of quantum field theory the magnetic moment of both the 
muon and electron were thought to be exactly two, due to the quantized spin 
they carry.  However, the quantum fluctuations predicted by quantum field 
theory and QED, the theory of the electromagnetic interactions, add small 
corrections the muon's and electron's magnetic moment, a major acheivement 
of QED.  Later Schwinger, Glashow, and Salam proposed a similar gauge group
to describe
electroweak interactions, $SU(2)\times U(1)$.  However, this
theory struggled to accomodate charge massive gauge bosons of the weak
current, since the mass terms of
the lagrangian would violate the gauge symmetry.  However, around the same 
time, Higgs, ... had proposed a method for spontaneously breaking guage 
symmetries which could allow for massive gauge fields through the interaction
with a extra gauge field, now commonly referred to as the Higgs field.  
This was finally formulated into a coherent theory of the electroweak 
interactions by Weinberg which is now referred to as the 
Glashow-Salam-Weinberg model. 

\section{Electroweak Symmetry Breaking}
\label{sec:Electroweak Symmetry Breaking}

It was known in non-quantum systems that spontaneously broken symmetry
could provide a theoretical framework for explaining massive gauge 
bosons\cite{}. In early 1960's these ideas were studied in the context
of quantum field theories.  It was shown that a complex scalar field 
whos potential was particularly chosen, could not only spontaneously 
break a gauge symmetry but generate gauge boson masses through the 
interaction of this field with the gauge boson\cite{}.  Most notibly, 
Higgs suggested that this would also predict the presence of a new 
massive scalar particle whos interactions with the gauge bosons 
scaled with their mass\cite{}.  

\subsection{Electroweak Interactions}
\label{sec:Electroweak interactions}

The gauge structure of the weak interactions demonstrated by Glashow 
\textcolor{red}{and Salam(?)}\cite{} and Weinberg showed that the 
electromagentic and weak interactions could be unified and implemented
the Higgs mechanism into was is today know to be the Standard Model (SM)
of electroweak interactions.  Together, this is referred to as the 
Glashow-Salam-Weinberg (GSW) model.

The GSW model had concrete experimentally observable predictions.  Since
the gauge symmetry reprossible for generating the interactions was
$U(1)\otimes SU(2)$ which has a total of four generators, a third guage 
boson should exist.  The GSW model specifically that this boson would 
be neutral and mediate neutrino-electron interactions through the 
diagram in Figure~\ref{}.  These so-called neutral currents were later
discovered in \textcolor{red}{197?} by the \textcolor{red}{--- collaboration}.
Both the charged and neutral weak gauge bosons were later discovered
through means of direct detection\cite{}.  

The experimental verification of the Higgs boson and the ...

Another experimental signature of the GSW model was that a chargeless,
colorless, spinless, massive boson, the Higgs boson, should exist.  
Except for its mass all properties about this particle could be calculated
from first principles, see Section~\ref{sec:HiggsPhen}.  Several accelerators
have been built to confirm its existance, Large Electron-Positron collider (LEP), 
the Tevatron, and most recently the Large Hadron Collider.  

\section{The Higgs boson}
\label{sec:The Higgs boson}

\begin{itemize}
\item Why it is critical to observe and understand the properties of the Higgs
and EWSB.

Both performed direct searches for the SM Higgs boson to either 
discover or rule out the existance of a Higgs boson as specific
masses.  Since the lifetime of a Higgs is \textcolor{red}{universally}
small searches are done looking for its decay products.  

\item Describe LEP searches for the Higgs.

\item Describe Tevatron searches for the Higgs.

\item Precision measurements of the Z boson ...
Global fits suggest that ... 

\item What could be beyond the SM ...

\end{itemize}

%problems with the SM

As described above the Higgs mechanism, as described in the SM, via GSW, 
conveniently solved several problems, the exist of massive gauge bosons,
the apparent disparity between the electromagentic and weak forces, and 
the unitarity of weak boson scattering. Yet, despite its success at 
describing terrestrial experiments, the SM fails 
to explain a number of phenomena observed in the universe.
  
It is thought that more than 95\% of the known universe consists of dark 
matter ($\sim25$\%) and dark energy ($\sim75$\%)\cite{??}.  Since there is 
currently no way to explain either dark matter or dark energy with the SM, 
only about 4\% of the constituents of the universe can be explained by the 
SM.  

The overabundance of matter, as opposed to anti-matter, in the 
universe, a phenomenon known as the baryon assymetry, is thought to not be 
attributable to any known process of the SM.
It was shown by Sakharov\cite{??} that there are three necessary conditions 
a model of baryogenesis must satisfy, baryon-number violation, 
charge-symmetry (C-symmetry) violation, charge-parity-symmetry violation 
(CP-violation), and interaction which are out of thermal equilibrium at early
stages of the universe.  Although it has been shown that the SM does
contain the three necessary conditions for baryogenesis, it is believed
to be insufficient for explaining the degree of baryonic assymetry in the 
visible universe.  As such, additional sources of CP-violation in the SM 
would provide a promising solution to the baryon-assymetry problem.  

The standard model also does not include a quantum description of the 
gravitational force.  It was shown by Randall and Sundrum that extra 
dimensional models with warped space-time metrics can provide a natural 
explaination of the heirarchy of not only the gravitational force and the 
weak force, but also heirarchy between the expected bare higgs mass and 
its physical mass. 

Quantum corrections to the higgs mass have been found to be much 
larger than the physical Higgs mass, if it is to provide the necessary   
cancellations to preserve unitarity in weak boson scattering.  These 
corrections can be offset by the bare Higgs mass but this introduces
what is known as fine tuning and thought to be a fault in the standard
model.  There are a number of proposed solutions to the fine tuning 
problem, some of which could also provide solutions to some of the 
problems noted above.  One such example is Super Symmetry, SUSY.  
Since SUSY predicts that all fermions have a corresponding boson, 
all feynman diagrams which provide quantum corrections to the Higgs 
mass have a canceling partner which reduce the need of fine tuning. 
SUSY is also thought to provide a natural dark matter candidate and
is a prerequisite for string theory, which some believe to be provide
a sound theory of gravity.  Finally, it is possible that for SUSY 
to allow for additional CP-violation in the Higgs sector.  Recent work
has studied this idea in the more generic framework of type-II 2 Higgs
doublet models (2HDM) and found that the amount of additional CP-violation
possible in the Higgs sector could provide a reasonable model for 
baryeogenesis.  

Although much of this is huristic, it provide strong evidence 
that the Higgs sector could be a window to physics beyond the standard
model, either through the discover of multiple scalars, through 
the varification of either CP-violation in Higgs interactions, or 
even through the discovery of Higgs compositeness.  

Today, the muon 
magnetic moment has been calculated and measured to \textcolor{red}{11} 
significant digits and has been used as a high precision test of the SM 
as well as a probe for new physics.   Analogously, the Higgs may become
the next source of high precision tests of the standard model which may
ultimately illuminate the exist of physics beyond the standard model.  

%overview of material in thesis
This thesis will discuss several analyses designed to search for the
new resonances, especially those related to electroweak symmetry 
breaking using tools which have been developed to provide not only 
increased sensitivity to signal events but also measure property of 
observed resonances. 
The sections are organized as follows: chapter~\ref{sec:LHC} will 
discuss the experimental details of the Large Hadron Collider (LHC) and 
the Compact Muon Solenoid (CMS); chapter~\ref{sec:HiggsPhen} will discuss 
the Higgs phenomenology at the LHC; chapter~\ref{sec:HZZsearches} will
present two analyses designed to search for the SM Higgs at high mass
using the $ZZ\to2\ell 2q$ signature and over a broad range of masses
using the $ZZ\to 4\ell$ signature; chapter~\ref{sec:HiggsProp} will discuss
properties measurments of the newly discovered boson using the MELA 
technique as well as prospect for high precision property measurements
at either the LHC or a future $e^+e^-$ collider; finally, chapter will
discuss the interpretation of these results in the context of the 
beyond the SM physics mentioned above.  


