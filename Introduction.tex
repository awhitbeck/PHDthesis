\chapter{Introduction}
\label{sec:intro}
\chaptermark{Introduction}

The Standard Model (SM) of particle physics \textcolor{red}{(reference)} 
is a mathematical description of the fundamental particles and their 
interactions.  Particles are described by quantized excitations of 
various types of fields.  Each field is a solution to either the
Klein-Gordon, Dirac, Proca, or Maxwell equation which govern the time
evolution of spin-0, spin-1/2, and spin-1 fields.
Although bound states and possible extensions to the SM can have
larger spin, the standard model doesn't explicitly describe the interactions
of these particles.

The interactions themselves are encoded in the Standard Model lagrangian.
For example, the electromagnetic interactions of electrons are described by
the lagrangian in equation~\ref{eq:EMlagrangian}.  The probability of the
some initial state, i, evolving into some final state, j, can be 
be expanded in powers of the coupling constants, g and g', according to 
equation equation~\ref{eq:matrixElementSquared}, the modulus squared of the
matrix element.  More often the amplitude of a process at some order in 
the couplings is represented by a Feynman diagram such as the one in 
figure~\ref{fig:eeScattering} representing 
ectron-electron scattering at leading order in a purely electromagnetic
theory.

\begin{center}
\begin{equation}
\mathscr{L}_{EM} = \bar{\psi}\left(i\gamma^{\mu}(\partial_{\mu}+ieA_{\mu}-m\right)\psi - \frac{1}{4}F_{\mu\nu}F^{\mu\nu}
\label{eq:EMlagrangian}
\end{equation}
\end{center}
\vspace{-2.0cm}
\begin{center}
\begin{equation}
\mid<ee\mid e^{\bar{\psi}\left(i\gamma^{\mu}(\partial_{\mu}+ieA_{\mu}-m\right)\psi}\mid ee>\mid^2
\label{eq:matrixElementSquared}
\end{equation}
\end{center}

\begin{figure}
\begin{center}
\unitlength=1mm
\begin{fmffile}{eeScattering}

\begin{fmfgraph*}(40,30) \fmfpen{thick}
  \fmfleft{i1,i2} \fmfright{sp1,sp2}
  \fmf{fermion,label=$e^-$}{i1,v1,sp1} 
  \fmf{fermion,label=$e^-$}{i2,v2,sp2}
  \fmf{photon,label=$\gamma$}{v1,v2}
\end{fmfgraph*}

\end{fmffile}
\end{center}
\label{fig:eeScattering}
\caption{Feynman diagram depicting electron-electron scattering via
the electromagnetic interaction.}
\end{figure}


The interactions of the SM are derived from enforcing local gauge 
symmetries and
thus can be described through a symmetry group.  For example, the 
electromagnetic interactions are known to be generated from a U(1) 
gauge symmetry.  Each symmetry has a corresponding charge which is
conserved and which the gauge mediators, the photon in the case of the
electromagnetic force, couple to.  Thus, understanding the gauge symmetries
which generate the interactions of the standard model and the charges of
each of the particles tells us how particles interact.

Currently, the SM describes three of the four known forces, the
electromagnetic, the weak, and the strong force.  As the names 
suggest, these forces operate with different strengths; the 
strong froce is the strongest and the weak force being the weakest. 
The strong, weak, and electromagnetic interactions are generate from
SU(3), SU(2), and U(1) gauge symmetries, respectively.  The charges of the fundamental
fields known to exist in the standard model are shown in 
table~\ref{table:SMcharges}.  The photon only couples to charged
particles, the W boson couple to weak isospin, $T_3$, and the 
gluons couple to color.  One unique consequence of this charge 
structure is that the photon cannot couple to itself while the 
weak vector boson and the gluons can.  The is also indicitive of
the fact that the SU(2) and SU(3) gauge groups are non-abelian. 

\begin{table}
\begin{center}
\begin{tabular}{l|c|c|c|c}
\hline 
\hline
particle & Q  & $T_3$ & $Y_W$ & colored \\ \hline \hline
$e_L$, $\mu_L$, $\tau_L$  & -1 & -1/2   &  -1 &  no \\ 
$e_R$, $\mu_R$, $\tau_R$  & -1 & 0      &  -2 &  no \\ 
$\nu_L$  & 0   & 1/2 & -1 & no \\ 
$u_L$    & 2/3 & 1/2  & 1/3& yes \\ 
$u_R$    & 2/3 & 0  & 4/3& yes \\ 
$d_L$    & -1/3& -1/2 & 1/3& yes \\ 
$d_R$    & -1/3& 0 & -2/3& yes \\
H        & 0   & 1/2  & -- & no \\
W        & 1   & 1    & -- & no \\
Z        & 0   & --   & -- & no \\
$\gamma$ & 0   & --   & -- & no \\
gluon    & 0   & 0    & 0  & yes \\
\hline
\end{tabular}
\end{center}
\label{table:SMcharges}
\caption{List of standard model particles and their charges. 
Q represents the charge of the $SU(1)_{em}$ gauge symmetry,
$T_3$ the broken SU(2) gauge symmetry, and $Y_W$ the broken 
U(1) gauge symmetry.}
\end{table}

The idea of interactions arising from enforcing a gauge symmetry, 
naively,
produces inconsistancies between theory and experiments.  Even at 
the time when the SU(2) structure of the weak interactions was first
proposed by Glashow~\cite{??}, the W boson was known to be 
massive.  One of the predictions of this guage structure, the 
existance of a neutral weak current, was later found to also 
be massive.  However, mass terms break gauge
invariance.  This internal inconsistancy was a major hurdle for 
having a robust theory of the electroweak interactions.  Thus, 
the guage symmetry must be broken in a specific way in order
to allow the weak vector bosons to be massive, a process known
as electroweak symmetry breaking.  

\section{Electroweak Symmetry Breaking}
\label{sec:Electroweak Symmetry Breaking}

It was known in non-quantum systems that spontaneously broken symmetries
could provide a theoretical framework for explaining massive gauge 
bosons\cite{}. In the early 1960's these ideas were studied in the context
of quantum field theories.  It was shown that a complex scalar field 
whose potential was particularly chosen, could spontaneously 
break a gauge symmetry and generate gauge boson masses through the 
interaction of this field with the gauge boson\cite{}.  Most notibly, 
Higgs suggested that this would also predict the presence of a new 
massive scalar particle\cite{}, what is now referred to as a Higgs boson.

It was shown by Glashow, Weinberg, and Salam that the Higgs mechanism 
could be used to break a $SU(2)\times U(1)$ symmetry to a $U(1)_{em}$
symmetry producing all of the known interactions and massive weak gauge 
bosons, a model known as the Glashow-Salam-Weinberg (GSW) model.
The prediction of a massive neutral charge boson around 90 GeV 
was later discovered indirectly through electron-neutrino scattering.  
Both the W and Z boson were directly discovered at the SppS.

Another experimental signature of the GSW model was that a chargeless,
colorless, spinless, massive boson, the Higgs boson, should exist.  
Except for its mass all properties about this particle could be calculated
from first principles, see Section~\ref{sec:HiggsPhen}.  Several 
accelerators have been built to confirm its existance, Large 
Electron-Positron collider (LEP), the Tevatron, and most recently the 
Large Hadron Collider.  

\section{The Higgs boson}
\label{sec:The Higgs boson}

The exact nature of the
electroweak symmetry breaking is the cornerstone of the SM model.
The experimental verification of the existance the Higgs boson is
paramount to our understanding of the SM.  The lack of a Higgs in 
nature would be a paradigm shifting notion.  
Several large-scale
particle accelerators have searched for the Higgs boson, most 
failing to do so.

The first of which was the Large Electron-Positron (LEP) collider
which accelerated electrons and positron to energies up to 209 GeV.
The four experiments produced many high precision measurements
of SM quantities.   Although a broad range of Higgs masses 
were accessible to LEP experiments, no evidence was found and 
95\% CL exlusion limits were set for all masses up to 114.4~GeV~\cite{??}.
Although no evidence of the Higgs boson was 
found, the high precision measurements made on a number of SM 
quantities could be used to constrain the Higgs mass, assuming it 
were to exist according to the SM.  These constraints suggested 
that a SM Higgs would be more likely in the range 100-150~GeV.  

The Tevatron, built at Fermi National Accelerator Laboratory, 
and its experiments also contributed major efforts towards Higgs
searches.  As a 2~TeV $p\bar{p}$ collider, considerably larger 
Higgs masses were accessible.  However, no evidence was found
and 95\% CL exclusion limits were set for Higgs mass between
$X.XX<m_H<Y.YY~GeV$.  The top quark mass measurement also helped
to refine calculations of the Higgs production cross section and 
branching ratios which include top loops.  

By the time the LHC was delivering beams to CMS, theory
calculations had been refined, direct search limits, and indirect limits
were set.  Figure~\ref{fig:???} summarizes the status of Higgs 
searches at this time.  Since the Higgs mechanism
must unitarize VV scattering, there is a limited mass range for 
which the Higgs makes sense, $m_H<\sim1000~GeV$. This theoretical
upper bound and the experimental lower bound from the LEP direct
search limits suggest that the LHC would suffise to make the final
statement about the exist of the Higgs, nearly 50 years after it
was first proposed.  

\section{Beyond the SM Higgs}

As described above the Higgs mechanism, as described in the SM, via GSW, 
conveniently solved several problems, the exist of massive gauge bosons,
the apparent disparity between the electromagentic and weak forces, and 
the unitarity of weak boson scattering. Yet, despite its success at 
describing terrestrial experiments, the SM fails 
to explain a number of phenomena observed in the universe.
  
It is thought that more than 95\% of the known universe consists of dark 
matter ($\sim25$\%) and dark energy ($\sim75$\%)\cite{??}.  Since there is 
currently no way to explain either dark matter or dark energy with the SM, 
only about 4\% of the constituents of the universe can be explained by the 
SM.  

The overabundance of matter, as opposed to anti-matter, in the 
universe, a phenomenon known as the baryon assymetry, is thought to not be 
attributable to any known process of the SM.
It was shown by Sakharov\cite{??} that there are three necessary conditions 
a model of baryogenesis must satisfy, baryon-number violation, 
charge-symmetry (C-symmetry) violation, charge-parity-symmetry violation 
(CP-violation), and interaction which are out of thermal equilibrium at early
stages of the universe.  Although it has been shown that the SM does
contain the three necessary conditions for baryogenesis, it is believed
to be insufficient for explaining the degree of baryonic assymetry in the 
visible universe.  As such, additional sources of CP-violation in the SM 
would provide a promising solution to the baryon-assymetry problem.  

Quantum corrections to the higgs mass have been found to be much 
larger than the physical Higgs mass, if it is to provide the necessary   
cancellations to preserve unitarity in weak boson scattering.  These 
corrections can be offset by the bare Higgs mass but this introduces
what is known as fine tuning and thought to be a fault in the standard
model.  There are a number of proposed solutions to the fine tuning 
problem, some of which could also provide solutions to some of the 
problems noted above.  One such example is Super Symmetry, SUSY.  
Since SUSY predicts that all fermions have a corresponding boson, 
all feynman diagrams which provide quantum corrections to the Higgs 
mass have a canceling partner which reduce the need of fine tuning. 
SUSY is also thought to provide a natural dark matter candidate and
is a prerequisite for string theory, which some believe to be provide
a sound theory of gravity.  Finally, it is possible that for SUSY 
to allow for additional CP-violation in the Higgs sector.  Recent work
has studied this idea in the more generic framework of type-II 2 Higgs
doublet models (2HDM) and found that the amount of additional CP-violation
possible in the Higgs sector could provide a reasonable model for 
baryeogenesis.  

The standard model also does not include a quantum description of the 
gravitational force.  It was shown by Randall and Sundrum that extra 
dimensional models with warped space-time metrics can provide a natural 
explaination of the heirarchy of not only the gravitational force and the 
weak force, but also heirarchy between the expected bare higgs mass and 
its physical mass. 

Although much of this is huristic, it provide strong evidence 
that the Higgs sector could be a window to physics beyond the standard
model: through the discover of multiple scalars, through 
the varification of CP-violation in Higgs interactions, or 
through the discovery of Higgs compositeness.  

Today, the muon 
magnetic moment has been calculated and measured to \textcolor{red}{11} 
significant digits and has been used as a high precision test of the SM 
as well as a probe for new physics.   Analogously, the Higgs may become
the next source of high precision tests of the standard model which may
ultimately illuminate the exist of physics beyond the standard model.  

\section{Summary}

%overview of material in thesis
This thesis will discuss several analyses designed to search for 
new resonances, especially those related to electroweak symmetry 
breaking using tools which have been developed to provide not only 
increased sensitivity to signal events but also measure properties of 
observed resonances. 
Chapter~\ref{sec:LHC} will 
discuss the experimental details of the Large Hadron Collider (LHC) and 
the Compact Muon Solenoid (CMS). Chapter~\ref{sec:HiggsPhen} will discuss 
the Higgs phenomenology at the LHC.  Chapter~\ref{sec:HZZsearches} will
present two analyses designed to search for the SM Higgs using the 
$ZZ\to2\ell 2q$ signature and using the $ZZ\to 4\ell$ signature. The 
latter will include the discovery and characterization of a new
bosonic resonance. Chapter~\ref{sec:FutureMeasurements} will discuss 
the prospects of precision measurements of Higgs properties at both
the LHC and an $e^+e^-$ collider. Finally, chapter~\ref{sec:Conclusions}
will discuss the interpretation of these results in the context of the 
beyond the SM physics mentioned above.  
