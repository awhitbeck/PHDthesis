
\section{Semi-leptonic decay channel}
\label{sec:HZZ2l2q}

The semileptonic final state of the ZZ channel is studied in two 
different kinematic regions, the low mass region 
($125 < m_{2\ell2q} < 170$~GeV) and the high mass region 
($183 < m_{2\ell2q} < 800$~GeV).  Because of the small ZZ branching ratio
expected from the SM Higgs, the intermediate range 
($170 < m_{2\ell2q} < 183$~GeV) is not considered in this analysis.

\subsection{Event Simulation}
\label{sec:HZZ2l2qSimulation}

The analysis strategy, including selections and data-driven background
estimations, were optimized and validated on MC simulations.  Signal
samples are generated with {\verb+POWHEG+}~\cite{??} and {\verb+JHUGen+}~\cite{??}.  
Inclusive Z 
production is generated with either {\verb+MADGRAPH+} 4.4.12~\cite{??} or
{\verb+ALPGEN+} 2.13~\cite{??}.  Continuum diboson production, ZZ, WW, 
and ZW, samples are generated with {\verb+PYTHIA+} 6.4.22~\cite{???}.  
Top backgrounds are generated with either {\verb+MADGRAPH+} 4.4.12 or 
{\verb+POWHEG+}.  Parton distribution functions are modeled using 
{\verb+CTEQ6+}~\cite{???} at leading order and {\verb+CT10+}~\cite{??} at 
next-to-leading order (NLO).  Parton showering is modeled with {\verb+PYTHIA+}
while detector response is simulated with a CMS specific implementation
of {\verb+GEANT4+}~\cite{??}.  A full list of the MC samples used is shown 
in table~\ref{table:HZZ2l2qMCsamples} along with the cross section for each process.  
MC simulations are corrected for mismodelling of pileup and any relative 
efficiencies found between data and MC through tag and probe 
measurements

\begin{table}
\begin{center}
\begin{tabular}{l|c|c|c}
\hline 
\hline
Name & Generator & $\Gamma$ [GeV] & $\sigma\times\mathscr{B}_{ZZ}\times\mathscr{B}_{2l2q}$ [fb] \\
\hline
SM Higgs  & \verb+POWHEG+ & & \\
$m_H=130-600$ & \verb+JHUGen+ & 0.0081-123 & 125.129-14.7312 \\ 
\hline \hline
Name & Generator & $\Gamma$ [GeV] & $\sigma_{LO}~(\sigma_{NLO})$ [pb] \\ \hline \hline
Z+jets    & \verb+MADGRAPH+ & -- & 2289 (3084) \\
Z+jets    & \verb+MADGRPH+  & -- & 2943        \\
$t\bar{t}$& \verb+PYTHIA+   & -- & 94 (157.5)  \\
$t\bar{t}$& \verb+POWHEG+   & -- & 15.86 (16.7)\\
$ZZ\to$anything & \verb+PYTHIA+ & -- & 4.30 (5.9) \\ 
$WW\to$anything & \verb+PYTHIA+ & -- & 10.4 (18.3) \\ 
$ZW\to$anything & \verb+PYTHIA+ & -- & 27.8 (42.9) \\ 
\hline\hline
\end{tabular}
\caption{}
\label{table:HZZ2l2qMCsamples}
\end{center}
\end{table}

\subsection{Event Reconstruction, Selection, and Categorisation}
\label{sec:HZZ2l2qselection}

Only events which contain two oppositely charge leptons, either electrons or
muons, and two jets are considered in this analysis.  Both leptons
flavors are required to have tranverse momentum, $p_T$, greater than 20~GeV
and 10~GeV for the leading and subleading $p_T$, respectively.  For events 
which are used in the high mass analysis, this constraint is tightened to 
$p_T>40, 20$~GeV.  Only muons (electrons) in the pseudorapidity range 
$|\eta|<2.4 (2.5)$, excluding the gap between the barrel and endcap region, 
are considered.  These selections not only serve as a rudimentary method for
rejecting background but are consistent with the double electron and 
double muon triggers that are used.  Muons are required to be well 
isolated from hadronic activity in the detector by restricting the sum of
transverse momentum from the tracker or transverse energy in the ECAL and
HCAL within a cone of $\Delta R  = \sqrt{(\Delta\eta)^2+(\Delta\phi)^2}<0.3$
to be less than 15\% of the measured $p_T$.  Similar requirements are placed
on electrons although the details depend also on the electron shower shape.

Jets are reconstructed with the particle-flow (PF) algorithm~\cite{???}.
Reconstructed particle  candidates are clustered with the anti-$k_T$ 
algorithm~\cite{???} with a clustering parameter $R=0.5$.  All jets which 
overlap with well-isolated leptons are removed from consideration.  
Jets are required to be in the tracker acceptance, $|\eta<2.4|$, to maximize
the effectiveness of the PF algorithm.  Energy corrections  are applied
to jets to account for systematic instrumental effects including the 
non-linear energy response of the calorimeters.  These corrections are 
derived from in-situ measurements using dijet and $\gamma+$jet control
samples~\cite{???}.  Effects of pileup are mitigated by applying corrections
according to the Fastjet algorithm~\cite{???}.  Some requirement is also 
applied to the energy balance between the charged and neutral hadronic 
content in each jet.  In some cases, jet substructure variables are used
to distinguish on a statistical bases differences between gluon jets and
quark jets. Finally, all jets are required to have $p_T>30$~GeV.

With the basic objects in hand, the $2\ell2q$ system is constructed under
the assumption that all pairs of leptons and quarks are the duaghters of
Z bosons.  Each di-lepton pair must have a combined invariant mass of 
$70 < m_{\ell\ell} < 110$~GeV, thus reducing backgrounds which
don't have an intermediate Z, like $t\bar{t}$ and QCD backgrounds.  In order 
to reduce the overwhelming Z+jets background, the dijet invariant mass of 
the event is required to satisfy $75 < m_{jj} < 105$~GeV.  
Figure~\ref{fig:HZZ2l2qPreselectiOnly} shows the $m_{jj}$ for signal and background. 
Solid histograms represent the expected distributions of background events 
from MC simulation, open histograms represent expected distribuionts of 
a SM Higgs, $m_H=400$~GeV, and points with error bars represent distributions
of the observed data.

Categorizing events based on jet flavor provides a significant 
increase in sensitivity to signal events since b-jets are 
more likely to result from a Z decay than from QCD radiation.  
To isolate jets which are likely to originate from b-quarks, the CMS track 
counting high-efficiency (TCHE) b-tagging algorithm~\cite{???} 
is used.  This algorithm relies on tracks within the jet cone having
large impact parameters, indicating a displaced vertex.  This information
is encompassed in a discriminant which is used
to determine how b-like jets are in shown in figure~\ref{fig:HZZ2l2qPreselectiOnly}.
Using this discriminant, the events are divided into three
categories: those which have at least one jet passing the median 
working point ($\sim65\%$ efficient) and another jet passing the
loose working point ($\sim80\%$ efficient); those which have at 
least one jet passing the loose working point; those which 
have zero jets passing the loose working point.  Although there 
is a non-negligible mistag rate for each of these working points, 
the categories are referred to as 2 b-tag, 1 b-tag, and 
0 b-tag, respectively. The categories are defined such that 
they are mutually exclusive by putting events in the category 
with the most stringent requirements.  The division of events
in each of the three categories is shown if 
figure~\ref{fig:HZZ2l2qPreselectiOnly}.

Since there is a significant amount of $t\bar{t}$ and $tW$ 
events in the 2 b-tag category, events in which the PF candidate
collection has a significant inbalance of transverse energy, 
also known as missing transverse energy, $E_T^{miss}$, are removed.
To quantify this, a likelihood ratio, $\lambda(E_T^{miss})$,
is built comparing two hypothesis, $E_T^{miss}=0$ and 
$E_T^{miss}\neq 0$.  Events in the 2 b-tag category are then 
required to satisfy $2\ln\lambda(E_T^{miss})<10$.  In the 
low mass analysis, we instead require $E_T^{miss}<50$~GeV in
the 2 b-tag category.    

In order to further reduce the overwhelming background, Z+jets,
in the high mass analysis, the MELA technique is employed.  
The five angular variables described in 
chapter~\ref{sec:HiggsPhen} are used.  Above threshold, the 
Z masses provide little discrimination power and are dropped 
for simplicity.  The discrimant used makes use of the 
5D probability distributions, 
$\mathscr{P}(cos\theta^*,cos\theta_1,cos\theta_2,\Phi,\Phi_1|m_{ZZ})$,
according to
\begin{equation}
D = \frac{\mathscr{P}_{Higgs}}{\mathscr{P}_{Higgs}+\mathscr{P}_{Zjets}}.
\label{eq:KD}
\end{equation}
The expected and observed
distributions for these 5 angles for both signal and background
are shown in figure~\ref{fig:HZZ2l2qAngularLD}.

Since the ideal distributions for the dominant background 
can't easily be described in terms of the angular variables, 
the distributions are found 
emperically from MC, including all detector effects, 
using simple functions, asssuming no correlations between 
the 5 angular variables.  The $\cos\theta_1$, $\cos\theta_2$, 
and $\cos\theta^*$ projections are modelled with even polynomials
in the corresponding variable.  This makes use of prior knowledge
that the distributions should be symmetric.  In addition, 
the $\cos\theta_2$ projection includes a fermi-dirac distribution
to model the sharp acceptance effect found with the hadronic
Z near $\cos\theta_2=1$.  The $\Phi$ and $\Phi_1$ projections 
are modelled with a finite fourier series.
Fits to Z+jets MC are performed in slices of $m_{ZZ}$.  A function
is then extrapolated between slices so that $\mathscr{P}_{Zjets}$ 
is continuous in $m_{ZZ}$.  Some examples of these fits are shown
in figure~\ref{fig:HZZ2l2qBackgroundPDF}.  

The signal parameterization must also include detector effects.
The ideal distributions from section~\ref{sec:HiggsPhen} are 
modified with 5D uncorrelated function which is then fit to 
MC to account for any detector effects.  The parameterization 
of detector effects is the same as those used for describing background.
Also as with background, 
these fits are performed in slices of $m_{ZZ}$ and extrapolated
to arbitrary values.  Examples of the signal parameterizations
are shown in figure~\ref{fig:HZZ2l2qSignalPDF}.  

Combining these two density
functions together, the discriminant, D, is shown in
figure~\ref{fig:HZZ2l2qAngularLD}.  The signal events tend to peak more 
towards 1 while the background events tend to peak more towards
zero.  Thus, selecting events greater than some value will
increase the signal purity.  

Because the shape of D changes with $m_{ZZ}$, the optimal cut will 
be $m_{ZZ}$ dependent.
An optimization was run using $\kappa=N_{sig}/\sqrt{N_{bkg}}$ as a figure of 
merrit.  This variable represents an approximation of the expected 
upper limit, UL.  For a simple counting experiment in which the 
expected number of background events is large, $\kappa$ is a good
approximation of the true UL.  The optimization was performed 
separately for each of the three b-tagging categories and the
proposed LD cuts along with the cuts used for other variables
are shown in table~\ref{table:HZZ2l2qCuts}.

The angular variables which are used as input to the angular LD represent 
a set of variables which are only loosely correlated with the final discriminating 
variable, $m_{ZZ}$.  As a result, cutting on this variables does not significantly
alter the shape of the $m_{ZZ}$ distribution.  In constrast, an optimized set of 
cuts on more traditional variables ($p_{T,lepton}$, $p_{T,jet}$, $p_{T,\ell\ell}$, 
$\Delta R_{jets}$), which are highly correlated with $m_{ZZ}$, 
would produce a peaked distribution near the signal region.  This is demonstrated
in figure~\ref{fig:HZZ2l2qShapeBias}, where an optimized cut on both sets of 
variables is applied and the resulting $m_{ZZ}$ is shown.  The preservation of
the $m_{ZZ}$ shape allows for the them to be easily described and used
for statistical interpretation of the final observed distributions.  

\begin{figure}
\begin{center}
\includegraphics[width=.32\linewidth]{HZZ2l2qPlots/CosThetaSproj_475_550.eps}  
\includegraphics[width=.32\linewidth]{HZZ2l2qPlots/CosTheta1proj_475_550.eps}  
\includegraphics[width=.32\linewidth]{HZZ2l2qPlots/CosTheta2proj_475_550.eps}\\
\includegraphics[width=.32\linewidth]{HZZ2l2qPlots/Phiproj_475_550.eps}        
\includegraphics[width=.32\linewidth]{HZZ2l2qPlots/PhiStar1proj_475_550.eps}   
\caption{ Emperical derivation of 5D PDF for Z+jets events.  Points
represent expected distributions of events between $475<m_{ZZ}<550$~GeV from MC simulation, lines represent the final model at the 
median $m_{ZZ}$ value.}
\label{fig:HZZ2l2qBackgroundPDF}
\end{center}
\end{figure}

\begin{figure}
\begin{center}
\includegraphics[width=.32\linewidth]{HZZ2l2qPlots/sigPDF_CosThetaSproj_500.eps}  
\includegraphics[width=.32\linewidth]{HZZ2l2qPlots/sigPDF_CosTheta1proj_500.eps}  
\includegraphics[width=.32\linewidth]{HZZ2l2qPlots/sigPDF_CosTheta2proj_500.eps}\\
\includegraphics[width=.32\linewidth]{HZZ2l2qPlots/sigPDF_Phiproj_500.eps}        
\includegraphics[width=.32\linewidth]{HZZ2l2qPlots/sigPDF_PhiStar1proj_500.eps}   
\caption{ Emperical derivation of 5D PDF for signal events.  Points
represent expected distributions of events for $m_H=500$~GeV from MC
simulations, lines represent the final model at the 
median $m_{ZZ}$ value. }
\label{fig:HZZ2l2qSignalPDF}
\end{center}
\end{figure}

\begin{figure}
\begin{center}
\includegraphics[width=.9\linewidth]{HZZ2l2qPlots/2l2q_OptimalCutShape.eps}
\caption{ Distribution of $m_{ZZ}$ after optimal cut on angular LD (right) and traditional
variables, (left). Maroon histogram represents expected distribution of a 400 GeV SM Higgs,
blue and green histograms represent different SM backgrounds from MC simulations.}
\label{fig:HZZ2l2qShapeBias}
\end{center}
\end{figure}

\begin{table}
\begin{center}
\begin{tabular}{c|c|c|c}
\hline 
\hline

\multicolumn{4}{c}{preselection} \\ \hline
$p_T(\ell^\pm)$    & \multicolumn{3}{c}{leading $p_T>40(20)$~GeV, subleading $p_T>20(10)$~GeV} \\ 
$p_T(jets)$       & \multicolumn{3}{c}{$>30$~GeV} \\ 
$|\eta|(\ell^\pm)$ & \multicolumn{3}{c}{$<2.5(e^\pm), <2.4(\mu^\pm)$} \\ 
$|\eta|(jets)$    &  \multicolumn{3}{c}{$<2.4$} \\ \hline \hline
\multicolumn{4}{c}{final selection} \\ \hline \hline
           & 0 b-tag & 1 b-tag & 2 b-tag \\ \hline
b-tag      & none    & 1 loose & 1 loose \& 1 medium \\ 
D          & $>0.55+0.00025m_{ZZ}$ & $>0.302+0.000656m_{ZZ}$ & $>0.5$ \\
$E_T^{miss}$ & none   & none    & $2\ln\lambda(E_T^{miss})<10$ \\
&&& ($E_T^{miss}<50$~GeV) \\ \hline
$m_{jj}$    & \multicolumn{3}{c}{$\in [75,105]$~GeV} \\
$m_{\ell\ell}$& \multicolumn{3}{c}{$\in [70,110](<80)$~GeV}\\
$m_{ZZ}$    & \multicolumn{3}{c}{$\in [183,800] (\in [125,170])$~GeV} \\ \hline \hline
\end{tabular}
\caption{Table listing analysis selections.  The top portion details 
preselection cuts applied to all objects to be consistent with trigger 
requirements and detector acceptance.  The bottom portion details all 
cuts applied in each of the different b-tag categories to optimize the 
sensitivity to signal events.}
\label{table:HZZ2l2qCuts}
\end{center}
\end{table}

\begin{figure}
\begin{center}
\includegraphics[width=.32\linewidth]{HZZ2l2qPlots/data_loose_mjj.eps}
\includegraphics[width=.32\linewidth]{HZZ2l2qPlots/data_loose_tche.eps}\\
\includegraphics[width=.32\linewidth]{HZZ2l2qPlots/data_loose_tag.eps}
\includegraphics[width=.32\linewidth]{HZZ2l2qPlots/data_loose_met.eps}
\caption{ ... }
\label{fig:HZZ2l2qPreselectiOnly}
\end{center}
\end{figure}

\begin{figure}
\begin{center}
\includegraphics[width=.32\linewidth]{HZZ2l2qPlots/angle_cosThetaStar.eps}
\includegraphics[width=.32\linewidth]{HZZ2l2qPlots/angle_cosTheta1.eps}
\includegraphics[width=.32\linewidth]{HZZ2l2qPlots/angle_cosTheta2.eps}\\
\includegraphics[width=.32\linewidth]{HZZ2l2qPlots/angle_phi.eps}
\includegraphics[width=.32\linewidth]{HZZ2l2qPlots/angle_phi1.eps}
\includegraphics[width=.32\linewidth]{HZZ2l2qPlots/data_loose_ld.eps}
\caption{ ... }
\label{fig:HZZ2l2qAngularLD}
\end{center}
\end{figure}

\subsection{Yields and Kinematics Distributions}
\label{sec:HZZ2l2qyields}

From figures~\ref{fig:HZZ2l2qPreselectiOnly} and~\ref{fig:HZZ2l2qAngularLD}
it is clear that the agreement between data and MC is fairly good.  
Although there are some disagreements in some of the distributions,
these disagreements reflect the complexity that exists in modeling
inclusive Z production.  To ensure that background estimations are
reliable in the more restricted phase space of the final selections,
it is important to have a methodology for measuring background 
shapes and normalizations directly from data.

Data control regions are defined using events passing all
of the final selctions in table~\ref{table:HZZ2l2qCuts} but instead
lie in the regions $60 < m_{jj} < 75$~GeV or $105 < m_{jj} < 130$~GeV. 
These regions are mutually exclusive from the signal region, 
$75<m_{jj}<105$~GeV, and include
only a small contribution from signal events, as evident from 
figure~\ref{fig:HZZ2l2qPreselectiOnly}.  Since the kinematics of
this control region are not expected to be exactly the same as 
the signal 
region, events are reweighted to account for the 
differences between the signal region and the control region. 
The expected number
of background events in a given $m_{ZZ}$ range can be estimated by
\begin{equation}
N_{bkg}(m_{ZZ}) = N_{CR}(m_{ZZ})\times\frac{N_{bkg}^{sim}(m_{ZZ})}{N_{CR}^{sim}(m_{ZZ})}=N_{CR}(m_{ZZ})\times\alpha(m_{ZZ}),
\label{eq:HZZ2l2qAlpha}
\end{equation}
where $N_{bkg}$ is the number of events expected in data in the 
signal region, $N_{CR}$ is the number of events observed in the
data control region, and $N_{CR}^{sim}$, $N_{bkg}^{sim}$ are the 
events measured in the MC control region and signal region, 
respectively.  Thus, $\alpha$ represents the weight for extrapolating
between the signal and control region and is calculated using
MC simulation.
These weights range between 0.75 and 1.2 and have been calculated
with two different MC generators, {\verb+MADGRAPH+} and {\verb+SHERPA+}, 
both give statistically compatible results.  This method allows
for both the expected shape and the normalization of the SM
background to be calculated for each of the three b-tag categories.

Once the expected distributions are calculated, the shape of the
background is fit using an empirical
function.  A crystal ball function multiplied by a fermi-dirac
distribution was found to provide a good description of the 
background in the three different b-tag categories in MC.
The uncertainties of the fit parameters and the statistical 
uncertainties on $\alpha$ are taken as systematic
uncertainties in the final statistical analysis.  
Figure~\ref{fig:HZZ2l2qMassDistributions} shows the expected 
shape and normalization of the $m_{ZZ}$ distribution
taken directly from MC (filled histograms), 
the data-driven estimation of the background shape and normalization
(blue line), and the observed distribution from data (points with 
error bars). Although
the MC generally does a reasonably good job of describing the 
observed distribution, there are some minor systematic effects which 
are corrected for by the data-driven estimation.  The SM Higgs
expectation enhanced by a factor 2 (5) or a Higgs mass of 400 (150)~GeV 
is also shown in yellow.  

\begin{figure}
\begin{center}
\includegraphics[width=.32\linewidth]{HZZ2l2qPlots/lowmass_mzz0btag.eps}
\includegraphics[width=.32\linewidth]{HZZ2l2qPlots/lowmass_mzz1btag.eps}
\includegraphics[width=.32\linewidth]{HZZ2l2qPlots/lowmass_mzz2btag.eps}\\
\includegraphics[width=.32\linewidth]{HZZ2l2qPlots/mzz_0btag.eps}
\includegraphics[width=.32\linewidth]{HZZ2l2qPlots/mzz_1btag.eps}
\includegraphics[width=.32\linewidth]{HZZ2l2qPlots/mzz_2btag.eps}
\caption{The $m_{ZZ}$ invariant mass distribution after final selection in
three categories: 0 b-tag (top), 1 b-tag (middle), and 2 b-tag (bottom). 
The low-mass range, $120<m_{ZZ}<170$~GeV is shown on the left and the
high-mass range, $183<m_{ZZ}<800$~GeV is shown on the right.  Points with
error bars show distributions of data and solid curved lines show the 
prediction of background from the control region extrapolation procedure.
In the low-mass range, the background is estimated from the $m_{ZZ}$ for
each Higgs mass hypothesis and the average expectation is shown.  Solid
histograms depicting the background expectation from simulated events
for the different components are shown.  Also shown is the SM Higgs boson
signal with the mass of 150 (400) GeV and cross section 5 (2) times that 
of the SM Higgs boson, which roughly corrsponds to the expected exclusion 
limits in each category.}
\label{fig:HZZ2l2qMassDistributions}
\end{center}
\end{figure}

While the background shapes and event yields are derived from data, the signal
model is derived from MC simulations.  Signal production cross sections
and branching ratios are taken from the Higgs cross sections working
group~\cite{??} and calculated at NLO.  Signal efficiencies are 
taken from CMS simulations and are
corrected for known differences between data and MC using tag and
probe measurements.  The efficiencies are also extrapolated to 
arbitrary values of $m_H$ using a polynomial fit.  Figures~\ref{fig:efficiencyHighMass}
and~\ref{fig:efficiencyLowMass}
shows the efficiency curves for each of the 6 categories.  The 
efficiencies together with the production cross section and branching
ratio are used to derive the expected event yields.

Signal shapes are modeled using both
{\verb+POWHEG+} to model production at NLO and {\verb+PYTHIA+} to model the 
decay kinematics or {\verb+JHUGen+} to model model both production and
decay at LO.  In order to get a good description of the signal
shape, events are fit in two separate categories.  Those in which 
both the jets used to build the Z are matched to generator level 
quarks from the Higgs decay, and those in which the jets are not 
matched.  The latter category represents event in which the Higgs 
was mis-reconstructed and thus is expected to have a much broader 
distribution.  Matched events are fit with a double crystal ball
function (i.e. a gaussian distribution whose tails are described
by two independent power law distributions).  Unmatched events
are fit with a triangle function convoluted with a crystal ball 
function.  Each signal sample is fit separately and the shape 
parameters are then extrapolated to arbitrary values of $m_H$.  
Examples of the signal shape model are shown for a 130~GeV and
400~GeV Higgs boson for each of the three b-tag categories
separately in figure~\ref{fig:signalShapeModels}.

\begin{figure}
\begin{center}
\includegraphics[width=.32\linewidth]{HZZ2l2qPlots/Parameterizations_SIG_400GeVHiggs_0btag.eps}
\includegraphics[width=.32\linewidth]{HZZ2l2qPlots/Parameterizations_SIG_400GeVHiggs_1btag.eps}
\includegraphics[width=.32\linewidth]{HZZ2l2qPlots/Parameterizations_SIG_400GeVHiggs_2btag.eps}
\includegraphics[width=.32\linewidth]{HZZ2l2qPlots/mZZExtrapolation_130GeV_0btag.eps}
\includegraphics[width=.32\linewidth]{HZZ2l2qPlots/mZZExtrapolation_130GeV_1btag.eps}
\includegraphics[width=.32\linewidth]{HZZ2l2qPlots/mZZExtrapolation_130GeV_2btag.eps}

\caption{Signal shapes models for 400 GeV (top row) and 130 GeV (bottom row) signals for each of the three b-tag categories, 0 b-tag (left), 1 b-tag (middle), and 2 b-tag (right).}
\label{fig:signalShapeModels}
\end{center}
\end{figure}

\begin{figure}
\begin{center}
\includegraphics[width=.32\linewidth]{HZZ2l2qPlots/effFit_ELE_0btag.eps}
\includegraphics[width=.32\linewidth]{HZZ2l2qPlots/effFit_ELE_1btag.eps}
\includegraphics[width=.32\linewidth]{HZZ2l2qPlots/effFit_ELE_2btag.eps}\\
\includegraphics[width=.32\linewidth]{HZZ2l2qPlots/effFit_MU_0btag.eps}
\includegraphics[width=.32\linewidth]{HZZ2l2qPlots/effFit_MU_1btag.eps}
\includegraphics[width=.32\linewidth]{HZZ2l2qPlots/effFit_MU_2btag.eps}
\caption{Signal efficieny parameterizations in each of the 6
different categories of the high mass signal samples.}
\label{fig:efficiencyHighMass}
\end{center}
\end{figure}

\begin{figure}
\begin{center}
\includegraphics[width=.32\linewidth]{HZZ2l2qPlots/effParam_ee0b.eps}
\includegraphics[width=.32\linewidth]{HZZ2l2qPlots/effParam_ee1b.eps}
\includegraphics[width=.32\linewidth]{HZZ2l2qPlots/effParam_ee2b.eps}\\
\includegraphics[width=.32\linewidth]{HZZ2l2qPlots/effParam_mm0b.eps}
\includegraphics[width=.32\linewidth]{HZZ2l2qPlots/effParam_mm1b.eps}
\includegraphics[width=.32\linewidth]{HZZ2l2qPlots/effParam_mm2b.eps}

\caption{Signal efficieny parameterizations in each of the 6
different categories of the low mass signal samples.}
\label{fig:efficiencyLowMass}
\end{center}
\end{figure}

A number of systematic uncertainties are associated with the calculation
of the number of expected event yields.  Many of these result from
limited understanding of reconstruction efficiencies.  The 
muon and electron reconstruction effiencies have been assigned 
uncertainties of 2.7\%, 4.5\%, respectively.  Jet reconstruction
effiency uncertainties range from 1-8\% depending on the Higgs mass
hypothesis.  The efficiency uncertainty of $E_T^{miss}$ cuts range from
3-4\%.  The b-tagging efficiency uncertainties depend both on the 
category as well as the Higgs mass hypothesis and range between 2-11\%. 
The additional jet identification requirements applied in the 0~b-tag
category, including gluon-tagging, is assigned an uncertainty of 4.6\%.
Uncertainties from Higgs production, either through parton distribution
functions, NLO corrections, or VBF modeling are assigned to both the
overall cross section calculation or the effect on acceptance due to
shape differences.  Mismodeling of Higgs $m_{ZZ}$ shape introduces some
additional systematic to the effective amount of event near the signal
peak.  Since the width depends strong on the mass hypthosis, $m_H$, the
uncertainties also depends on $m_H$ according to 
$1.5\times10^{-7}\%\times m_H^3$ [GeV].  Finally, uncetainties from 
luminosity measurements are accounted for in the signal systematics. 
All systematic uncertainties on the signal yields are summarized in 
table~\ref{table:HZZ2l2qSystematics}.

\begin{table}
\begin{center}
\begin{tabular}{l|c|c|c}
\hline 
\hline

source & 0 b-tag & 1 b-tag & 2 b-tag \\ \hline \hline

muon reconstruction & \multicolumn{3}{c}{2.7\%} \\ \hline
electron reconstruction & \multicolumn{3}{c}{4.5\%} \\ \hline
jet reconstruction & \multicolumn{3}{c}{1-8\%} \\ \hline
pile-up & \multicolumn{3}{c}{3-4\%} \\ \hline
$E_T^{miss}$ & -- & -- & 3-4\% \\ \hline
b-tagging & 2-7\% & 3-5\% & 10-11\% \\ \hline
gluon-tagging & 4.6\% & -- & -- \\ \hline
acceptance(HqT) & 2\% & 5\% & 3\% \\ \hline
acceptance(PDF) & \multicolumn{3}{c}{3\%} \\ \hline
acceptance(VBF) & \multicolumn{3}{c}{1\%} \\ \hline
signal cross section (PDF) & \multicolumn{3}{c}{8-10\%} \\ \hline
signal cross section (scale) & \multicolumn{3}{c}{8-11\%} \\ \hline
signal shape    & \multicolumn{3}{c}{$1.5\times10^{-7}\%\times m_{H}^{3}$ [GeV]} \\ \hline
luminosity      & \multicolumn{3}{c}{4.5\%} \\ \hline \hline

\end{tabular}
\caption{Summary of systematic uncertainties on signal normalization.
Most sources give multiplicative uncertainties on the cross section
measurement, except for the expected Higgs boson production cross 
section, which is relevant for the measurement of the ratio to the 
SM expectation.  The ranges indicate dependence on $m_{H}$.}
\label{table:HZZ2l2qSystematics}
\end{center}
\end{table}

\subsection{Results and Summary}
\label{sec:HZZ2l2qxsec}

The expected background event yields, both from MC simulation and from the
data-driven estimations, and signal event yields are compared against
the observed event yields in each of the three b-tag categories in 
table~\ref{table:HZZ2l2qYields}.
Since there are no significant excesses found in any of the observed 
invariant mass spectra, limits on the Higgs cross section are calucated.

\begin{table}
\begin{center}
\begin{tabular}{l|c|c|c}
\hline \hline
 & 0 b-tag & 1 b-tag & 2 b-tag \\ \hline \hline
\multicolumn{4}{c}{$m_{ZZ} \in [125,170]$} \\ \hline
observed yield & 1087 & 360  & 30 \\
expected background (data-driven) & 1050$\pm$54 & 324$\pm$28 & 19$\pm$5 \\ 
expected background (MC)& 1089$\pm$39 & 313$\pm$20 & 24$\pm$4 \\ \hline \hline
\multicolumn{4}{c}{$m_{ZZ} \in [183,800]$} \\ \hline
observed yield & 3036 & 3454  & 285 \\
expected background (data-driven) & 3041$\pm$54 & 3470$\pm$59 & 258$\pm$17 \\ 
expected background (MC)& 3105$\pm$39 & 3420$\pm$41 & 255$\pm$11 \\ \hline \hline
\multicolumn{4}{c}{signal expectation (MC)} \\  \hline
$m_H=150$~GeV & 10.1$\pm$1.5 & 4.1$\pm$0.6 & 1.6$\pm$0.3 \\ \hline
$m_H=250$~GeV & 24.5$\pm$3.5 & 21.7$\pm$3.0 & 8.1$\pm$1.7 \\ \hline
$m_H=350$~GeV & 29.6$\pm$4.3 & 26.0$\pm$3.7 & 11.8$\pm$2.5 \\ \hline
$m_H=450$~GeV & 16.5$\pm$2.4 & 15.8$\pm$2.2 & 7.9$\pm$1.7 \\ \hline
$m_H=550$~GeV & 6.5$\pm$1.0 & 6.5$\pm$0.9 & 3.6$\pm$0.8 \\ \hline
\hline \hline
\end{tabular}
\caption{Observed and expected event yields for $4.6~fb^{-1}$ of
data.  The yields are quoted  in the ranges $125<m_{ZZ}<170$~GeV
or $183<m_{ZZ}<800$~GeV, depending on the Higgs boson hypothesis.
The expected  background is quoted from both the data-driven 
estimations and from MC simulations directly. In the low-mass
range, the background  is estimated from the $m_{ZZ}$ sideband
for each  Higgs mass hypothesis and is not quoted in the table.
The errors on the expected  background  from simulation include
only statistical uncertainties.}
\label{table:HZZ2l2qYields}
\end{center}
\end{table}

A simultaneous fit of the $m_{ZZ}$ distributions for the signal 
cross section in the six different channels is perform using a 
dedicated statistical software package discussed in ref.~\cite{Chatrchyan:2012tx}.  Using the distribution of the $CL_S$ test 
statistic~\cite{Read:2002hq}, 95\% CL limits are calculated.
Expected limits are derived from pseudoexperimental which are 
generated based on expected distributions.  Nuisance parameters 
associated with the different systematic uncertainties are 
randomized when generating toys and profiled in fits.  

For the low
mass analysis, the median expected upper limit on the $H\to ZZ$ 
cross section, solid black line, $1\sigma$ ($2\sigma$) band of 
the expected limit, green (yellow), and the observed upper limit 
as a function of $m_H$ is shown in the left plot of 
figure~\ref{fig:HZZ2l2qLimitsSM4}.  Similarly, the expected and 
observed distribution of upper limits on the $H\to ZZ$ cross 
section are shown for the high mass region in right plot of 
figure~\ref{fig:HZZ2l2qLimitsSM4}.  In both plots of 
figure~\ref{fig:HZZ2l2qLimitsSM4}, the Higgs cross section
within the SM model is shown in red.  
Upper limits on the ratio of the 95\% CL cross section limits 
with respect to the SM Higgs cross section, including all 
theoretical uncertainties, are shown in 
figure~\ref{fig:HZZ2l2qLimits}.  While the low mass region
limits are at best around four times SM Higgs cross sections, 
the high mass region has an expected exclusion for Higgs masses 
between X.XX and Y.YY. The observed data excludes Higgs boson 
masses between X.XX and Y.YY.

\begin{figure}
\begin{center}
\includegraphics[width=.49\linewidth]{HZZ2l2qPlots/plot_low_ul.eps}
\includegraphics[width=.49\linewidth]{HZZ2l2qPlots/plot_ul.eps}
\caption{Observed (solid) and expected (dashed) 95\% CL upper limit on
the ration f the production cross section o the SM expectation for the 
HIggs boson obtained using the $CL_s$ technique.  The 68\% ($1\sigma$) 
and 95\% ($1\sigma$) ranges of expectation for the background-only model
are shown with green and yellow bands, respectively.  The solid line at 
1 indicates the SM expectation.  Left: low-mass range, right: high-mass
range. }
\label{fig:HZZ2l2qLimits}
\end{center}
\end{figure}

%\begin{figure}
%\begin{center}
%\includegraphics[width=.49\linewidth]{HZZ2l2qPlots/plot_low_sigma.eps}
%\includegraphics[width=.49\linewidth]{HZZ2l2qPlots/plot_sigma.eps}
%\caption{Observed (dashed) and expected (solid) 95\% CL upper limit on
%the product of the production cross section and branching fraction for 
%$H\to ZZ$ obtained with the $CL_s$ technique.  The 68\% ($1\sigma$) 
%and 95\% ($2\sigma$) ranges of expectation for the background-only model
%are also shown with green and yellow bands, respectively.  The expected
%product of the SM Higgs production cross section and the branching
%fraction is shown as a solid red curve with a band indicating theoretical
%uncertainties at 68\% CL.  The same expectation in the fourth-generation
%model is shown with a red dashed curve with a band indicating theoretical
%uncertainties.  Left: low-mass range, right: high-mass range.  }
%\label{fig:HZZ2l2qLimitsSM4}
%\end{center}
%\end{figure}

