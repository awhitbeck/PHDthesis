
\section{Semi-leptonic decay channel}

The semileptonic final state of the ZZ channel is studied in two 
different kinematic regions, the low mass region 
($125 < m_{2\ell2q} < 170 GeV$) and the high mass region 
($183 < m_{2\ell2q} < 800 GeV$).  Because of the small ZZ branching ratio
expected from the SM Higgs, the intermediate range 
($170 < m_{2\ell2q} < 183 GeV$) is not considered in this analysis.

\subsection{Event Simulation}
\label{sec:HZZ2l2qSimulation}

The analysis strategy include selections and data-driven background
estimations were optimized and validated on MC simulations.  Signal
samples are generated with {\it POWHEG}~\cite{??} and JHUGen~\cite{??}.  
Inclusive Z 
production is generated with either {\it MADGRAPH} 4.4.12~\cite{??} or
{\it ALPGEN} 2.13~\cite{??}.  Continuum diboson production, ZZ, WW, 
and ZW, samples are generated with {\it PYTHIA} 6.4.22~\cite{???}.  
top backgrounds are generated with either {\it MADGRAPH} 4.4.12 or 
{\it POWHEG}.  Parton distribution functions are modeled using 
{\it CTEQ6}~\cite{???} at leading order and {\it CT10}~\cite{??} at 
next-to-leading order (NLO).  Parton showering is modeled with {\it PYTHIA}
while detector response is simulated with a CMS specific implementation
of {\it GEANT4}~\cite{??}.  A full list of the MC samples used is shown 
in table~\ref{table:??} along with the cross section for each process.  

\subsection{Event Selection and Categorisation}
\label{sec:HZZ2l2qselection}

Only events which contain two oppositely charge leptons, either electrons or
muons, and two jets are considered in this analysis.  Both leptons
flavors are required to have tranverse momentum, $p_T$, greater than 20 GeV
and 10 GeV for the leading and subleading $p_T$, respectively.  For events 
which are used in the high mass analysis, this constraint is tightened to 
$p_T>40, 20 GeV$.  Only muons (electrons) in the pseudorapidity range 
$|\eta|<2.4 (2.5)$, excluding the gap between the barrel and endcap region, 
are considered.  These selections not only serve as a rudimentary method for
rejecting background but are consistent with the double electron and 
double muon triggers that are used.  The efficiency of these triggers are 
measured on data and found to be X.XX.  MC is corrected for any residual 
inefficiency found between MC and data. Muons are required to be well 
isolated from hadronic activity in the detector by restricting the sum of
transverse momentum from the tracker or transverse energy in the ECAL and
HCAL within a cone of $\Delta R  = \sqrt{(\Delta\eta)^2+(\Delta\phi)^2}<0.3$
to be less than 15\% of the measured $p_T$.  Similar requirements are placed
on electrons although the details depend also on the electron shower shape.

Jets are reconstructed with the particle-flow (PF) algorithm~\cite{???}.
Reconstructed particle  candidates are clusted with the anti-$k_T$ 
algorithm~\cite{???} with a distance parameter $R=0.5$.  All jets which 
overlap with well-isolated leptons are then removed from consideration.  
Jets are required to be in the tracker acceptance, $|\eta<2.4|$, to maximize
the effectivness of the PF algorithm.  Energy corrections  are applied
to jets to account for systematic instrumental effects including the 
non-linear energy response of the calorimeters.  These corrections are 
derived from in-situ measurements using dijet and $\gamma+$jet control
samples~\cite{???}. To mitigate the additional energy in the calorimeters 
from pileup, the average energy density times the jet area is subtracted 
from each jet~\cite{???}.  Some requirement is also applied to the energy
balance between the charged and neutral hadronic content in each jet.  
Finally, all jets are required to have $p_T>30 GeV$.

With the basic objects in hand, the $2\ell2q$ system is constructed under
the assumption that all pairs of leptons and quarks are the duaghters of
Z bosons.  Each di-lepton pair must have a combined invariant mass of 
$70 < m_{\ell\ell} < 110$.  This is primary meant to reduce backgrounds which
don't have an intermediate Z, like $t\bar{t}$ and QCD backgrounds.  In order 
to reduce the overwhelming Z+jets background, the dijet invariant mass of 
the event is required to satisfy $75 < m_{jj} < 105$.  Figure~\ref{fig:HZZ2l2qPreselection} shows the $m_{jj}$ for signal and background.  

Catagorizing events based on jet flavor provides a significance 
increase in sensitivity to signal events since two b-jets are 
more likely to result from a Z decay than from QCD radiation.  
To isolate jets which are likely to be b-jets, the CMS track 
counting high-efficiency (TCHE) b-tagging algorithm~\cite{???} 
is used.  This algorithm relies on tracks with large impact 
parameters. {\it explain a little better.} The discriminant used
to determine how b-like jets are in shown in figure~\ref{fig:??}.
Using this discriminant, the events are divided into three
categories: those which have at least one jet passing the median 
working point ($\sim65\%$ efficient) and another jet passing the
loose working point ($\sim80\%$ efficient); those which have at 
least one jet passing the loose working point but; those which 
have zero jets passing the loose working point.  Although there 
is a non-negligible mistag rate for each of these working points, 
the categories are referred to as 2 b-tag, 1 b-tag, and 
0 b-tag, respectively. The categories are defined such that 
they are mutually exclusive by putting events in the category 
with the most stringent requirements.  The division of events
in each of the three categories after preselection requirements
is shown if figure~\ref{fig:???}.

Since there is a significant amount of $t\bar{t}$ and $tW$ 
events in the 2 b-tag category, events are required to have
missing transverse energy, $E_T^{miss}$ which is compatible with 
zero.  The $E_T^{miss}$ is defined to be the modulus of the 
negative vector sum of all the PF candidates in the event.  
To quantify this, a likelihood ratio, $\lambda(E_T^{miss})$ 
is built comparing two hypothesis, $E_T^{miss}=0$ and 
$E_T^{miss}\neq 0$.  Events in the 2 b-tag category are then 
required to satisfy $2\ln\lambda(E_T^{miss})<10$.  In the 
low mass analysis, we instead require $E_T^{miss}<50~GeV$ in
the 2 b-tag category.    

In order to reduce further the overwhelming background, Z+jets,
a kinematic discriminant is built to distinguish between
Higgs-like and Z+jets-like events.  The basis of variables 
that is used as well as the distributions that describe the 
signal events is described in the section~\ref{sec:HiggsPhen}.
The discrimant that is used makes use of the 5D probability 
distributions, 
$\mathscr{P}(cos\theta^*,cos\theta_1,cos\theta_2,\Phi,\Phi_1)$,
to build a discriminant,
\begin{equation}
\mathscr{D} = \frac{\mathscr{P}_{Higgs}}{\mathscr{P}_{Higgs}+\mathscr{P}_{Zjets}}.
\label{eq:KD}
\end{equation}

Since the dominant background can't easily be described in 
terms of angular distributions, the distributions are found 
emperically from MC using approximate functions.  Fits to 
Z+jets MC are performed in slices of $m_{ZZ}$.  A function
is then extrapolated between slices so that $\mathscr{P}_{Zjets}$ 
is continuous in $m_{ZZ}$.  Some examples of these fits are shown
in figure~\ref{fig:???}.  Since the MC includes detector effects, 
the signal parameterization should also include detector effects.
The ideal distributions from section~\ref{sec:HiggsPhen} are 
modified with 5D uncorrelated function which is then fit to 
MC to account for any detector effects.  As with background, 
these fits are performed in slices of $m_{ZZ}$ and extrapolated
to arbitrary values.  Examples of the signal parameterizations
are shown in figure~\ref{fig:???}.

{\it angular likelihood discriminant description!!!}

\begin{figure}
\begin{center}
\includegraphics[width=.49\linewidth]{HZZ2l2qPlots/data_loose_mjj.pdf}
\includegraphics[width=.49\linewidth]{HZZ2l2qPlots/data_loose_tche.pdf}\\
\includegraphics[width=.49\linewidth]{HZZ2l2qPlots/data_loose_tag.pdf}
\includegraphics[width=.49\linewidth]{HZZ2l2qPlots/data_loose_met.pdf}
\label{fyn:HZZ2l2qPreselectiony}
\caption{ ... }
\end{center}
\end{figure}

\subsection{Yields and Kinematics Distributions}
\label{sec:HZZ2l2qyields}

\begin{figure}
\begin{center}
\includegraphics[width=.49\linewidth]{HZZ2l2qPlots/angle_cosThetaStar.pdf}
\includegraphics[width=.49\linewidth]{HZZ2l2qPlots/angle_cosTheta1.pdf}\\
\includegraphics[width=.49\linewidth]{HZZ2l2qPlots/angle_cosTheta2.pdf}
\includegraphics[width=.49\linewidth]{HZZ2l2qPlots/angle_phi.pdf}\\
\includegraphics[width=.49\linewidth]{HZZ2l2qPlots/angle_phi1.pdf}
\includegraphics[width=.49\linewidth]{HZZ2l2qPlots/data_loose_ld.pdf}
\label{fyn:HZZ2l2qAngularLD}
\caption{ ... }
\end{center}
\end{figure}

\begin{figure}
\begin{center}
\includegraphics[width=.49\linewidth]{HZZ2l2qPlots/lowmass_mzz0btag.pdf}
\includegraphics[width=.49\linewidth]{HZZ2l2qPlots/mzz_0btag.pdf}\\
\includegraphics[width=.49\linewidth]{HZZ2l2qPlots/lowmass_mzz1btag.pdf}
\includegraphics[width=.49\linewidth]{HZZ2l2qPlots/mzz_1btag.pdf}\\
\includegraphics[width=.49\linewidth]{HZZ2l2qPlots/lowmass_mzz2btag.pdf}
\includegraphics[width=.49\linewidth]{HZZ2l2qPlots/mzz_2btag.pdf}
\label{fyn:HZZ2l2qMassDistributions}
\caption{ ... }
\end{center}
\end{figure}

\subsection{Results}
\label{sec:HZZ2l2qxsec}

\begin{figure}
\begin{center}
\includegraphics[width=.49\linewidth]{HZZ2l2qPlots/plot_low_ul.pdf}
\includegraphics[width=.49\linewidth]{HZZ2l2qPlots/plot_ul.pdf}\\
\includegraphics[width=.49\linewidth]{HZZ2l2qPlots/plot_low_sigma.pdf}
\includegraphics[width=.49\linewidth]{HZZ2l2qPlots/plot_sigma.pdf}\\
\label{fyn:HZZ2l2qLimits}
\caption{ ... }
\end{center}
\end{figure}

