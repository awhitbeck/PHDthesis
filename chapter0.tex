%% FRONTMATTER
\begin{frontmatter}

% generate title
\maketitle

\begin{abstract}
Understanding the exact mechanism of electroweak symmetry breaking
through the discovery and characterization of the Higgs boson
is one of the primary goals of the Large Hadron Collider (LHC).
Two searches for a Higgs boson decaying to a pair of Z bosons
with subsequent decays to either $2\ell2q$ or $4\ell$ are 
presented using data
recorded with the Compact Muon Solenoid (CMS). 
The discovery and characterization of a Higgs-like resonance
using a new set of tools is reported.  
The foundations of 
such tools are developed and prospects for their use
in other Higgs channels and at future colliders are addressed.
Although the Standard
Model (SM) of electroweak interactions has been extremely
successful in describing a number of phenomena, there are still
questions to be addressed pertaining to its naturalness and its
possible connection to beyond the SM physics.  
Results are interpreted in the context of possible
extensions to the SM and their effect on our understanding of
the universe.
\vspace{.3cm}
\noindent Primary Reader: Andrei Gritsan\\
Secondary Reader:

\end{abstract}

\begin{acknowledgment}

I would like to thank Andrei Gritsan for accepting me as a student.
I am lucky to have been a part of developing the great ideas that
have resulted from his research program and have learned an 
immense amount physics and how to approach research problems.  
I have been fortunate to take on a leading role in
my field and to represent my collaboration on more than one 
occasion as an ambassador to the greater scientific community. 
This would not have been possible with his encouragement and
guidance.

I would also like to thank everyone involved with CMS and the
LHC.  It has been a remarkable experience to be 
a part of the collaboration and see what can be done when
thousands of people put their minds to one big idea.  
I would also like to give special thanks to all the 
CMS research groups: te Higgs PAG, HZZ subgroup, and tracker 
alignment group.  
I am eternally grateful for those who have supported me
in my continue academic career: Chiara, Joe, Andrey, and Yves. 

\end{acknowledgment}

%\begin{dedication} 

%\end{dedication}

% generate table of contents
\tableofcontents

% generate list of tables
\listoftables

% generate list of figures
\listoffigures

\end{frontmatter}
